\documentclass[a4paper,oneside, 10pt]{book}

\usepackage[utf8]{inputenc}
\usepackage[cm-default]{fontspec}
\usepackage{xunicode}
\usepackage{xltxtra}

\usepackage{xgreek, hyperref}

\setmainfont[Mapping=tex-text]{CMU Serif}

\usepackage{graphicx}
\begin{document}
\chapter*{ ν. 4516/2018 }\section* { Version  0 } 
\subsection*{ Άρθρο 1 }
\paragraph { 0. }  Κυρώνονται και έχουν την ισχύ, που ορίζει το άρθρο 28 παρ. 1 του Συντάγματος, αφενός το Τροποποιητικό Πρωτόκολλο, της Συμφωνίας μεταξύ της Ευρωπαϊκής Κοινότητας και του Πριγκιπάτου του Λιχτενστάιν που προβλέπει μέτρα ισοδύναμα με τα θεσπιζόμενα στην Οδηγία 2003/48/ ΕΚ του Συμβουλίου για τη φορολόγηση των υπό μορφή τόκων εισοδημάτων από αποταμιεύσεις, που έχει κυρωθεί με τις διατάξεις του ν. 3365/ 2005 (Α΄ 161), και αφετέρου οι κοινές δηλώσεις των συμβαλλόμενων μερών, τα κείμενα των οποίων σε πρωτότυπο στην ελληνική γλώσσα, όπως έχουν δημοσιευθεί στην Επίσημη Εφημερίδα της Ευρωπαϊκής Ένωσης (ΕΕ ΕΕ L 339/3), έχουν ως εξής:     Τεύχος Α’ 19/07.02.2018       Τεύχος Α’ 19/07.02.2018       Τεύχος Α’ 19/07.02.2018       Τεύχος Α’ 19/07.02.2018       Τεύχος Α’ 19/07.02.2018       Τεύχος Α’ 19/07.02.2018       Τεύχος Α’ 19/07.02.2018       Τεύχος Α’ 19/07.02.2018       Τεύχος Α’ 19/07.02.2018       Τεύχος Α’ 19/07.02.2018       Τεύχος Α’ 19/07.02.2018       Τεύχος Α’ 19/07.02.2018       Τεύχος Α’ 19/07.02.2018       Τεύχος Α’ 19/07.02.2018       Τεύχος Α’ 19/07.02.2018       Τεύχος Α’ 19/07.02.2018       Τεύχος Α’ 19/07.02.2018       Τεύχος Α’ 19/07.02.2018       Τεύχος Α’ 19/07.02.2018       Τεύχος Α’ 19/07.02.2018       Τεύχος Α’ 19/07.02.2018       Τεύχος Α’ 19/07.02.2018       Τεύχος Α’ 19/07.02.2018       Τεύχος Α’ 19/07.02.2018       Τεύχος Α’ 19/07.02.2018       Τεύχος Α’ 19/07.02.2018   
\subsection*{ Άρθρο 2 }
\paragraph { 0. }  1.α. Ως «Αρμόδια Αρχή» που ενεργεί ως εξουσιοδοτημένος αντιπρόσωπος του Υπουργού Οικονομικών, δυνάμει του στοιχείου δ΄ της παραγράφου 1 του άρθρου 1 της Συμφωνίας και του Παραρτήματος ΙΙΙ στοιχείο η΄ αυτής, όπως τροποποιείται με τις παραγράφους 2 και 3 του άρθρου 1 του κυρούμενου Τροποποιητικού Πρωτοκόλλου, και ιδίως για την εφαρμογή της Συμφωνίας, όπως τροποποιείται με τις παραγράφους 2 και 3 του άρθρου 1 του κυρούμενου Τροποποιητικού Πρωτοκόλλου σε ό,τι αφορά θέματα που εμπίπτουν στις αρμοδιότητες της Ανεξάρτητης Αρχής Δημοσίων Εσόδων του ν. 4389/2016 (Α΄ 94), ορίζεται η αρμόδια αρχή της Ανεξάρτητης Αρχής Δημοσίων Εσόδων, κατά το άρθρο 5 του ν. 4170/2013 (Α΄ 163) και το άρθρο 17 παρ. 1 του ν. 4389/2016 με την επιφύλαξη της επόμενης περίπτωσης β΄.  β. Αρμόδια Αρχή δυνάμει του στοιχείου δ΄ της παραγράφου 1 του άρθρου 1 της Συμφωνίας και του Παραρτήματος ΙΙΙ στοιχείο η΄ αυτής, όπως τροποποιείται με τις παραγράφους 2 και 3 του άρθρου 1 του κυρούμενου Τροποποιητικού Πρωτοκόλλου, για την εφαρμογή των άρθρων 8 και 9 της Συμφωνίας, είναι ο Υπουργός Οικονομικών κατόπιν εισήγησης του Διοικητή της Ανεξάρτητης Αρχής Δημοσίων Εσόδων.  2. Ο κατάλογος για τις «Συμμετέχουσες Δικαιοδοσίες» δυνάμει του ορισμού της παραγράφου 5 της Ενότητας Δ΄ του Τμήματος VIII του Παραρτήματος Ι της Συμφωνίας, όπως τροποποιείται με την παράγραφο 3 του άρθρου 1 του κυρούμενου Τροποποιητικού Πρωτοκόλλου, δημοσιεύεται και κοινοποιείται από την Αρμόδια Αρχή της παραγράφου 1 περίπτωση α΄ του παρόντος στην Αρμόδια Αρχή του Λιχτενστάιν.  3.α. Τα Δηλούντα Ελληνικά Χρηματοπιστωτικά Ιδρύματα που προσδιορίζονται βάσει των ορισμών των περιπτώσεων ε΄ και ζ΄ της παραγράφου 1 του άρθρου 1 και της Ενότητας Α΄ του Τμήματος VIII του Παραρτήματος Ι της Συμφωνίας, όπως τροποποιείται με τις παραγράφους 2 και 3 του άρθρου 1 του κυρούμενου Τροποποιητικού Πρωτοκόλλου, υποχρεούνται να υποβάλουν ηλεκτρονικά στην Αρμόδια Αρχή της παραγράφου 1 περίπτωση α΄ του παρόντος, τις πληροφορίες που αφορούν κάθε Δηλωτέο Λογαριασμό Λιχτενστάιν δυνάμει της περίπτωσης ι΄ της παραγράφου 1 του άρθρου 1 και της παραγράφου 2 του άρθρου 2 της Συμφωνίας, όπως τροποποιείται με την παράγραφο 2 του άρθρου 1 του κυρούμενου Τροποποιητικού Πρωτοκόλλου, σύμφωνα με τις υποχρεώσεις υποβολής στοιχείων και δέουσας επιμέλειας, καθώς και να διασφαλίζουν την αποτελεσματική και σύμφωνη εφαρμογή αυτών με βάση τα οριζόμενα στα Παραρτήματα Ι και ΙΙ της Συμφωνίας, όπως τροποποιείται με την παράγραφο 3 του άρθρου 1 του κυρούμενου Τροποποιητικού Πρωτοκόλλου, και στις κατ’ εξουσιοδότηση του άρθρου τέταρτου του παρόντος νόμου προβλεπόμενες αποφάσεις.  β. Τα Δηλούντα Ελληνικά Χρηματοπιστωτικά Ιδρύματα μπορεί να χρησιμοποιούν παρόχους υπηρεσιών για να τηρήσουν τις υποχρεώσεις τους, σχετικά με την υποβολή στοιχείων και τη δέουσα επιμέλεια, που ισχύουν  για τα ίδια, δυνάμει της πρόβλεψης της Ενότητας Δ΄ του Τμήματος ΙΙ του Παραρτήματος Ι της Συμφωνίας, όπως τροποποιείται με την παράγραφο 3 του άρθρου 1 του κυρούμενου Τροποποιητικού Πρωτοκόλλου συμπεριλαμβανομένων των υποχρεώσεων που απορρέουν από τις διατάξεις της Συμφωνίας, όπως τροποποιείται με το κυρούμενο Τροποποιητικό Πρωτόκολλο και τις διατάξεις της εσωτερικής νομοθεσίας σχετικά με την τήρηση του φορολογικού απορρήτου, της εμπιστευτικότητας των πληροφοριών και των διατάξεων του ν. 2472/1997 (A΄ 50), εφόσον συντρέχουν σωρευτικά οι ακόλουθες υποχρεώσεις:  αα. Τα Δηλούντα Ελληνικά Χρηματοπιστωτικά Ιδρύματα υποχρεώνουν τους τρίτους παρόχους υπηρεσιών να τηρούν αντίγραφα των εγγράφων και των πληροφοριών που αποκτώνται από αυτούς.  ββ. Τα Δηλούντα Ελληνικά Χρηματοπιστωτικά Ιδρύματα δεν βασίζονται στις πληροφορίες για τα Δηλωτέα Πρόσωπα, όπως καθορίζονται από τους τρίτους παρόχους υπηρεσιών για τα οποία γνωρίζουν ή είναι εν τοις πράγμασι σε θέση να γνωρίζουν ότι οι ως άνω πληροφορίες είναι αναξιόπιστες ή ανακριβείς.  Για τις ανωτέρω υποχρεώσεις συνεχίζουν να ευθύνονται τα Δηλούντα Χρηματοπιστωτικά Ιδρύματα και οι ενέργειες των τρίτων παρόχων υπηρεσιών καταλογίζονται για την εφαρμογή αστικών, διοικητικών και ποινικών κυρώσεων στα Δηλούντα Χρηματοπιστωτικά Ιδρύματα. γ. Τα Δηλούντα Ελληνικά Χρηματοπιστωτικά Ιδρύματα μπορούν να εφαρμόζουν τις διαδικασίες δέουσας επιμέλειας για Νέους Λογαριασμούς σε Προϋπάρχοντες Λογαριασμούς και τις διαδικασίες δέουσας επιμέλειας για Λογαριασμούς Υψηλής Αξίας σε Λογαριασμούς Χαμηλότερης Αξίας δυνάμει της πρόβλεψης της Ενότητας Ε΄ του Τμήματος ΙΙ του Παραρτήματος Ι της Συμφωνίας, όπως τροποποιείται με την παράγραφο 3 του άρθρου 1 του κυρούμενου Τροποποιητικού Πρωτοκόλλου. Στην περίπτωση που επιλέξουν να εφαρμόζουν τις διαδικασίες δέουσας επιμέλειας για Νέους Λογαριασμούς σε Προϋπάρχοντες Λογαριασμούς, παραμένουν σε ισχύ οι κανόνες που ισχύουν κατά τα λοιπά στους Προϋπάρχοντες Λογαριασμούς.  δ. Αν δεν βρεθεί ένδειξη κατά την έρευνα σε αρχείο εγγράφων και δεν σταθεί δυνατό να εξασφαλιστεί η αυτοπιστοποίηση ή το αποδεικτικό έγγραφο δυνάμει των προβλέψεων της παραγράφου 5 της Ενότητας Β΄ και της περίπτωσης γ΄ της παραγράφου 5 της Ενότητας Γ΄ του Τμήματος ΙΙΙ του Παραρτήματος Ι της Συμφωνίας, όπως τροποποιείται με την παράγραφο 3 του άρθρου 1 του κυρούμενου Τροποποιητικού Πρωτοκόλλου, τα Δηλούντα Ελληνικά Χρηματοπιστωτικά Ιδρύματα υποχρεούνται να δηλώσουν τον εν λόγω λογαριασμό στην Αρμόδια Αρχή της παραγράφου 1 περίπτωση α΄ του παρόντος ως μη τεκμηριωμένο λογαριασμό.  ε. Τα Δηλούντα Ελληνικά Χρηματοπιστωτικά Ιδρύματα υποχρεούνται να θεσπίσουν ολοκληρωμένα και αποτελεσματικά εσωτερικά συστήματα και διαδικασίες συμμόρφωσης προς τις υποχρεώσεις που απορρέουν από τη Συμφωνία και τα Παραρτήματα Ι και ΙΙ αυτής, όπως τροποποιείται με το κυρούμενο Τροποποιητικό Πρωτό    κολλο, καθώς και στις προβλεπόμενες αποφάσεις κατ’ εξουσιοδότηση του άρθρου τέταρτου του παρόντος. Δεν θεωρείται ότι έχουν επιδείξει τη δέουσα επιμέλεια συμμορφώσεως, αν δεν έχουν συμμορφωθεί τουλάχιστον με τις απαιτήσεις του προηγούμενου εδαφίου.  4. Η ισχύς των διατάξεων του παρόντος άρθρου αρχί ζει από την 1η Ιανουαρίου 2016
\subsection*{ Άρθρο 3 }
\paragraph { 1. } Κάθε είδος ανταλλαγής πληροφοριών δυνάμει των διατάξεων των άρθρων 2, 3 και 5 της Συμφωνίας, όπως τροποποιείται με την παράγραφο 2 του άρθρου 1 του κυρούμενου Τροποποιητικού Πρωτοκόλλου υπόκειται στις διατάξεις του ν. 2472/1997. Η ανταλλαγή πληροφοριών, συμπεριλαμβανομένων των διοικητικών ερευνών, γίνεται κατόπιν ειδικά αιτιολογημένης απόφασης ως προς την αναγκαιότητα και αναλογικότητα των δεδομένων, τόσο από την αιτούσα αρχή όσο και από τη λαμβάνουσα αρχή, σύμφωνα με τις διατάξεις του άρθρου 4 του ν. 2472/1997 λαμβανομένων υπόψη και όσων ορίζονται στο άρθρο 6 της Συμφωνίας του προηγούμενου εδαφίου
\paragraph { 2. } Με την επιφύλαξη των διατάξεων της παραγράφου 4 του παρόντος, για τους σκοπούς του άρθρου 5 της Συμφωνίας, όπως τροποποιείται με την παράγραφο 2 του άρθρου 1 του κυρούμενου Τροποποιητικού Πρωτοκόλλου και ύστερα από ειδικά αιτιολογημένο αίτημα της αρμόδιας αρχής κατά το άρθρο 5 παρ. 1 του ν. 4170/2013 προς την Αρχή Προστασίας Δεδομένων Προσωπικού Χαρακτήρα μπορεί να αποφασίζεται εξαίρεση της άσκησης των δικαιωμάτων των άρθρων 11 και 12 του ν. 2472/1997
\paragraph { 3. } Σύμφωνα με την παράγραφο 2 του άρθρου 6 της Συμφωνίας, όπως τροποποιείται με την παράγραφο 2 του άρθρου 1 του κυρούμενου Τροποποιητικού Πρωτοκόλλου και τα οριζόμενα στα άρθρα δεύτερο και τρίτο του παρόντος νόμου και στις κατ’ εξουσιοδότηση του άρθρου τέταρτου προβλεπόμενες αποφάσεις, τα Δηλούντα Χρηματοπιστωτικά Ιδρύματα και οι αρμόδιες αρχές του Υπουργείου Οικονομικών και της Ανεξάρτητης Αρχής Δημοσίων Εσόδων δυνάμει των διατάξεων του π.δ. 111/2014 (Α΄ 178) και του ν. 4389/2016, όπως εκάστοτε ισχύουν, και των σχετικών με το οργανόγραμμα κανονιστικών αποφάσεων του Διοικητή της Ανεξάρτητης Αρχής Δημοσίων Εσόδων, θεωρούνται «υπεύθυνοι επεξεργασίας» κατά την έννοια και για τους σκοπούς του ν. 2472/1997, καθένας για κάθε επεξεργασία που διενεργεί. Οι υπεύθυνοι επεξεργασίας, κατά το προηγούμενο εδάφιο, οφείλουν να ειδοποιούν κάθε Δηλωτέο Πρόσωπο σε περίπτωση παραβίασης ασφάλειας που αφορά τα δεδομένα του, όταν η παραβίαση αυτή ενδέχεται να έχει αρνητικές επιπτώσεις στην προστασία των προσωπικών του δεδομένων ή του ιδιωτικού του βίου
\paragraph { 4. } Κάθε Δηλούν Ελληνικό Χρηματοπιστωτικό Ίδρυμα, κατά την έννοια του άρθρου δεύτερου παράγραφος 3 περίπτωση α΄ του παρόντος νόμου, για τους σκοπούς του άρθρου 6 της Συμφωνίας, όπως τροποποιείται με  την παράγραφο 2 του άρθρου 1 του κυρούμενου Τροποποιητικού Πρωτοκόλλου υποχρεούται να:  (α) ενημερώνει κάθε ενδιαφερόμενο Δηλωτέο Πρόσωπο ότι οι πληροφορίες που το αφορούν και περιλαμβάνονται στο άρθρο 2 της Συμφωνίας, όπως τροποποιείται με τις παραγράφους 2 και 3 του άρθρου 1 του κυρούμενου Τροποποιητικού Πρωτοκόλλου, συλλέγονται και διαβιβάζονται σύμφωνα με τα ειδικότερα οριζόμενα στη Συμφωνία, όπως τροποποιείται με το κυρούμενο Τροποποιητικό Πρωτόκολλο,  (β) παρέχει σε αυτό το πρόσωπο κάθε πληροφορία της οποίας δικαιούται να λαμβάνει γνώση δυνάμει των διατάξεων του ν. 2472/1997 εντός ευλόγου χρονικού διαστήματος, ώστε το πρόσωπο να ασκήσει τα δικαιώματά του ως προς την προστασία δεδομένων και, σε κάθε περίπτωση, πριν το ενδιαφερόμενο Δηλούν Ελληνικό Χρηματοπιστωτικό Ίδρυμα υποβάλλει τις πληροφορίες που ορίζονται στο άρθρο δεύτερο παράγραφος 3 περίπτωση α΄ του παρόντος νόμου στην Αρμόδια Αρχή της παραγράφου 1 περίπτωση α΄ του άρθρου δεύτερου του παρόντος
\paragraph { 5. } Με την επιφύλαξη των διατάξεων της φορολογικής νομοθεσίας περί παραγραφής, οι πληροφορίες που υφίστανται επεξεργασία δυνάμει της Συμφωνίας, και ιδίως της παραγράφου 2 του άρθρου 6, όπως τροποποιείται με το κυρούμενο Τροποποιητικό Πρωτόκολλο, διατηρούνται μόνο για το χρονικό διάστημα που απαιτείται για την επιδίωξη των σκοπών της εν λόγω Συμφωνίας και, σε κάθε περίπτωση, σύμφωνα με τις διατάξεις περί παραγραφής για κάθε υπεύθυνο επεξεργασίας.   6. Η ισχύς των διατάξεων του παρόντος άρθρου αρ χίζει από την 1η Ιανουαρίου 2016
\subsection*{ Άρθρο 4 }
\paragraph { 1. } Με κοινή απόφαση του Υπουργού Οικονομικών και του Διοικητή της Ανεξάρτητης Αρχής Δημοσίων Εσόδων μπορεί να τίθενται ή να εξειδικεύονται κανόνες, να καθορίζονται διοικητικές διαδικασίες και να λαμβάνονται τα αναγκαία συνακόλουθα μέτρα αυτών για τη διασφάλιση της αποτελεσματικής και σύμφωνης εφαρμογής των υποχρεώσεων υποβολής στοιχείων και δέουσας επιμέλειας που περιλαμβάνονται στη Συμφωνία και τα Παραρτήματα Ι και ΙΙ αυτής, όπως τροποποιείται με το κυρούμενο Τροποποιητικό Πρωτόκολλο από τα Δηλούντα Ελληνικά Χρηματοπιστωτικά Ιδρύματα. Ειδικότερα με την παραπάνω απόφαση μπορεί να ορίζονται τα ακόλουθα:  α. κανόνες ώστε να αποτρέπονται οι πρακτικές Χρηματοπιστωτικών Ιδρυμάτων, προσώπων ή ενδιαμέσων οι οποίες αποσκοπούν στην καταστρατήγηση των διαδικασιών υποβολής στοιχείων και δέουσας επιμέλειας, β. κανόνες που απαιτούν από τα Δηλούντα Ελληνικά Χρηματοπιστωτικά Ιδρύματα να τηρούν αρχεία σχετικά με τα μέτρα που λαμβάνουν και τα τυχόν αποδεικτικά στοιχεία στα οποία στηρίζονται για την τήρηση των ανωτέρω διαδικασιών, καθώς και να λαμβάνουν κατάλληλα μέτρα για την πρόσβαση στα αρχεία αυτά,  Τεύχος Α’ 19/07.02.2018    γ. διοικητικές διαδικασίες ώστε να διασφαλίζεται ότι εξακολουθεί να είναι μικρός ο κίνδυνος χρήσης των Οντοτήτων που λογίζονται ως απαλλασσόμενοι πραγματικοί δικαιούχοι ή θεωρούμενα ως συμμορφούμενα Αλλοδαπά Χρηματοπιστωτικά Ιδρύματα, ανά περίπτωση, και των λογαριασμών που εξαιρούνται από τους Χρηματοοικονομικούς Λογαριασμούς δυνάμει των οριζομένων στο Παράρτημα ΙΙ της Συμφωνίας, όπως τροποποιείται με το κυρούμενο Τροποποιητικό Πρωτόκολλο, και  δ. κάθε άλλο συναφές με τα παραπάνω θέματα
\paragraph { 2. } Με απόφαση του Διοικητή της Ανεξάρτητης Αρχής Δημοσίων Εσόδων μπορεί να τίθενται ή να εξειδικεύονται κανόνες, να καθορίζονται διοικητικές διαδικασίες και να λαμβάνονται τα αναγκαία συνακόλουθα μέτρα αυτών για τη διασφάλιση της τήρησης των ειδικότερων όρων που περιλαμβάνονται στη Συμφωνία με τα Παραρτήματά της, όπως τροποποιείται με το κυρούμενο Τροποποιητικό Πρωτόκολλο, και ειδικότερα:  α. διοικητικές διαδικασίες και μέτρα ώστε να εξακριβώνεται η έγκαιρη συμμόρφωση των Δηλούντων Ελληνικών Χρηματοπιστωτικών Ιδρυμάτων προς τις διαδικασίες υποβολής στοιχείων και δέουσας επιμέλειας,  β. κάθε άλλο συναφές θέμα σχετικά με την εφαρμογή της αυτόματης ανταλλαγής πληροφοριών της Συμφωνίας, συμπεριλαμβανομένων των θεμάτων του άρθρου δεύτερου της παραγράφου 3 της περίπτωσης α΄ του παρόντος, καθώς και των θεμάτων αξιολόγησης και αποτελεσματικής ανταλλαγής πληροφοριών
\end{document}