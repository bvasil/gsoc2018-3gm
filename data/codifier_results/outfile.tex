\documentclass[a4paper,oneside, 10pt]{book}

\usepackage[utf8]{inputenc}
\usepackage[cm-default]{fontspec}
\usepackage{xunicode}
\usepackage{xltxtra}

\usepackage{xgreek, hyperref}

\setmainfont[Mapping=tex-text]{CMU Serif}

\usepackage{graphicx}
\begin{document}
\chapter*{ ν. 4488/2018 }\section* { Version  0 } 
\subsection*{ Άρθρο 1 }
\paragraph { 1. } Στο τέλος της παρ. 1 του άρθρου 5 του ν. 4387/2016(Α΄ 85) προστίθεται εδάφιο, με έναρξη ισχύος από1.1.2017, ως εξής:«Οι ρυθμίσεις του προηγούμενου εδαφίου εφαρμόζονται και σε πρόσωπα που διορίζονται σε θέσειςμετακλητών υπαλλήλων με σχέση δημοσίου δικαίου.Εάν ο διορισμός των προσώπων αυτών έχει γίνει μέχρικαι 31.12.2016 εφαρμόζεται ως προς την εργοδοτικήεισφορά η 111482/0092/30.11.2016 κοινή υπουργικήαπόφαση (Β΄ 4005).».
\paragraph { 2. } Μετά το δεύτερο εδάφιο της παρ. 6 του άρθρου 7του ν. 4387/2016 προστίθεται εδάφιο ως εξής:«Η προϋπόθεση συμπλήρωσης δεκαπέντε (15) ετώνασφάλισης για την καταβολή της εθνικής σύνταξης δενισχύει για όσους θεμελιώνουν δικαίωμα σύνταξης με τησυμπλήρωση χρόνου ασφάλισης μικρότερου των δεκαπέντε (15) ετών. Στην περίπτωση αυτή το ποσό τηςεθνικής σύνταξης δεν μπορεί να υπολείπεται αυτού πουαντιστοιχεί στα δεκαπέντε (15) έτη ασφάλισης.».
\paragraph { 3. } Στο τέλος της παρ. 3 του άρθρου 8 του ν. 4387/2016προστίθενται εδάφια ως εξής:«Αν δεν προκύπτει χρόνος ασφάλισης (πραγματικός,πλασματικός, χρόνος προαιρετικής ασφάλισης ή άλλοςπου λογίζεται ως συντάξιμος), τουλάχιστον πέντε (5)ετών από την 1.1.2002 μέχρι την έναρξη καταβολής τηςσύνταξης του υπαλλήλου - λειτουργού του Δημοσίου ήτου στρατιωτικού, τότε για τον υπολογισμό των συντάξιμων αποδοχών αναζητείται χρόνος ασφάλισης (πραγματικός, πλασματικός, χρόνος προαιρετικής ή άλλος πουλογίζεται ως συντάξιμος) και κατά το πριν την 1.1.2002χρονικό διάστημα και μέχρι τη συμπλήρωση συνολικάέως πέντε (5) ετών ασφάλισης.Για συντάξεις με έναρξη καταβολής από 1.1.2021, ανδεν προκύπτει χρόνος ασφάλισης (πραγματικός, πλασματικός, χρόνος προαιρετικής ασφάλισης ή άλλος πουλογίζεται ως συντάξιμος), τουλάχιστον δέκα (10) ετώναπό την 1.1.2002 μέχρι την έναρξη καταβολής της σύνταξης του υπαλλήλου - λειτουργού του Δημοσίου ή τουστρατιωτικού, τότε για τον υπολογισμό των συντάξιμωναποδοχών αναζητείται χρόνος ασφάλισης (πραγματικός,πλασματικός, χρόνος προαιρετικής ή άλλος που λογίζεται ως συντάξιμος) και κατά το πριν την 1.1.2002 χρονικόδιάστημα και μέχρι τη συμπλήρωση έως δέκα (10) ετώνασφάλισης.».
\paragraph { 4. } Στο τέλος της παρ. 5 του άρθρου 8 του ν. 4387/2016προστίθεται εδάφιο ως εξής:«Το συνολικό ακαθάριστο ποσό της ανταποδοτικήςσύνταξης, όπως αυτό προκύπτει σύμφωνα με το παρόνάρθρο, δεν μπορεί να υπερβαίνει το ακαθάριστο ποσότων συντάξιμων αποδοχών, όπως αυτές ορίζονται στιςπαραγράφους 2 και 3.».
\subsection*{ Άρθρο 2 }
\paragraph { 1. } Το δεύτερο εδάφιο της παρ. 6 του άρθρου 11 τουπ.δ. 169/2007 (Α΄ 210) αντικαθίσταται ως εξής:«Επίσης θεωρείται ως συντάξιμος χρόνος ο χρόνοςτης άδειας άνευ αποδοχών ανατροφής παιδιών ηλικίας μέχρι έξι (6) ετών, όπως αυτός ισχύει κάθε φορά μεβάση τις οικείες διοικητικές διατάξεις και ο χρόνος τηςάδειας άνευ αποδοχών της παρ. 1 του άρθρου 51 τουν. 3528/2007 (Α΄ 26), με την προϋπόθεση της καταβολήςαπό τον υπάλληλο των προβλεπόμενων ασφαλιστικώνεισφορών.».
\paragraph { 2. } Στο τέλος της περίπτωσης β΄ της παρ. 2 του άρθρου56 του π.δ. 169/2007 προστίθεται εδάφιο, ως εξής:«Το συνολικό ποσοστό της ως άνω μείωσης, κατά τημεταβατική περίοδο σταδιακής αύξησης των ορίωνηλικίας συνταξιοδότησης της παρ. 9 του άρθρου 1 τουν. 4336/2015 (Α΄ 94), δεν μπορεί να υπερβαίνει το 30%.».
\paragraph { 3. } Η παρ. 1 του άρθρου 63 του π.δ. 169/2007 αντικαθίσταται ως εξής:«1. Η καταβολή της σύνταξης αναστέλλεται για όσοδιάστημα ο δικαιούχος εκτίει περιοριστική της ελευθερίας ποινή μεγαλύτερη του ενός (1) έτους και εφόσοντο αδίκημα για το οποίο καταδικάστηκε στρέφεται κατάτου Δημοσίου ή νομικών προσώπων δημοσίου δικαίου.Στην περίπτωση αυτή, η σύνταξη μεταβιβάζεται στουςτυχόν δικαιοδόχους αυτής σύμφωνα με τις εκάστοτεισχύουσες διατάξεις για συνταξιοδότηση λόγω θανάτου.Τα οικονομικά αποτελέσματα της παρούσας διάταξηςαρχίζουν από την 1η ημέρα του επόμενου μήνα από τηδημοσίευση του παρόντος νόμου. Κάθε αντίθετη διάταξη καταργείται.».
\paragraph { 4. } Στο τέλος της περίπτωσης ι΄ της παρ. 2 του άρθρου9 του π.δ. 169/2007 προστίθεται εδάφιο ως εξής:«Οι διατάξεις του τρίτου και τέταρτου εδαφίου δενέχουν εφαρμογή για όσα από τα αναφερόμενα σε αυτέςπρόσωπα υπάγονται για τον υπολογισμό της σύνταξήςτους στις διατάξεις του ν. 4387/2016.».
\subsection*{ Άρθρο 3 }
\paragraph { 1. } Από 1.1.2017 οι συντάξιμες αποδοχές, επί των οποίων υπολογίζονται ασφαλιστικές εισφορές, για όσουςυπηρετούν ή προσλαμβάνονται στο Δημόσιο με σχέσηεργασίας δημοσίου δικαίου και αμείβονται σύμφωναμε τις διατάξεις του ν. 4472/2017 (Α΄ 74), είναι οι μηνιαίες τακτικές αποδοχές της παρ. 10 του άρθρου 153 τουν. 4472/2017, με εξαίρεση την προσωπική διαφορά τουάρθρου 155 του ν. 4472/2017.
\paragraph { 2. } Οι συντάξιμες αποδοχές που προσδιορίζονται στηνπροηγούμενη παράγραφο έχουν εφαρμογή και στονυπολογισμό των ασφαλιστικών εισφορών των προσώπων αυτών για επικουρική ασφάλιση, εφάπαξ παροχήκαι υγειονομική περίθαλψη.
\subsection*{ Άρθρο 4 }
\paragraph { 1. } Το πρώτο και το δεύτερο εδάφιο της παρ. 1 τουάρθρου 60 του π.δ. 169/2007 αντικαθίστανται ως εξής:«1. Το δικαίωμα στη σύνταξη είναι απαράγραπτο, ταδε οικονομικά αποτελέσματα που γεννώνται από τηνάσκηση του δικαιώματος αυτού ανατρέχουν στην ημερομηνία έναρξης καταβολής της σύνταξης, όπως αυτήπροσδιορίζεται κατά περίπτωση από τις οικείες συνταξιοδοτικές διατάξεις του Δημοσίου.».
\paragraph { 2. } Οι ρυθμίσεις της παραγράφου 1 έχουν εφαρμογή γιατα πρόσωπα της περίπτωσης γ΄ της παρ. 1 του άρθρου6 του ν. 4387/2016.
\subsection*{ Άρθρο 5 }
\paragraph { 1. } Η περίπτωση β΄ του άρθρου 62 και η παρ. 1 τουάρθρου 64 του π.δ. 169/2007 καταργούνται.
\paragraph { 2. } Από την έναρξη ισχύος του παρόντος νόμου καταργούνται οι παράγραφοι 1, 2, 3, 4, 5 και 9 του άρθρου 3του ν. 3075/2002 (Α΄ 297).ΚΕΦΑΛΑΙΟ Β΄
\subsection*{ Άρθρο 6 }
\paragraph { 1. } Στο τέλος της παρ. 3 του άρθρου 28 του ν. 4387/2016προστίθενται εδάφια ως εξής:«Αν δεν προκύπτει χρόνος ασφάλισης (πραγματικός,πλασματικός, χρόνος προαιρετικής ασφάλισης ή άλλοςπου λογίζεται ως συντάξιμος), τουλάχιστον πέντε (5)ετών από την 1.1.2002 και έως την υποβολή της αίτησηςσυνταξιοδότησης, τότε για τον υπολογισμό των συντάξιμων αποδοχών αναζητείται χρόνος ασφάλισης (πραγματικός, πλασματικός, χρόνος προαιρετικής ή άλλος πουλογίζεται ως συντάξιμος) και κατά το πριν την 1.1.2002χρονικό διάστημα και μέχρι τη συμπλήρωση έως πέντε(5) ετών ασφάλισης.Για αιτήσεις συνταξιοδότησης με έναρξη καταβολήςαπό 1.1.2021, αν δεν προκύπτει χρόνος ασφάλισης(πραγματικός, πλασματικός, χρόνος προαιρετικής ασφάλισης ή άλλος που λογίζεται ως συντάξιμος), τουλάχιστον δέκα (10) ετών από την 1.1.2002 έως την έναρξητης συνταξιοδότησης, τότε για τον υπολογισμό τωνσυντάξιμων αποδοχών αναζητείται χρόνος ασφάλισης(πραγματικός, πλασματικός, χρόνος προαιρετικής ή άλλος που λογίζεται ως συντάξιμος) και κατά το πριν την1.1.2002 χρονικό διάστημα και μέχρι τη συμπλήρωσηέως δέκα (10) ετών ασφάλισης.».
\paragraph { 2. } Η παρ. 9 του άρθρου 29 του ν. 4387/2016 καταργείται από την έναρξη ισχύος της. Η παράγραφος 10 τουιδίου άρθρου αναριθμείται σε 9.
\subsection*{ Άρθρο 7 }
\paragraph { 0. } Μετά το τρίτο εδάφιο της παρ. 1 του άρθρου 30 τουν. 4387/2016, όπως συμπληρώθηκε με το άρθρο 24 τουν. 4445/2016 (Α΄ 236), διαγράφεται, από τότε που ίσχυσε,η φράση «πλην των προσώπων που υπάγονται στις διατάξεις των περιπτώσεων ε΄ και στ΄ της παρ. 2 του άρθρου46 του ν. 2084/1992 (Α΄ 165). Τα τελευταία πρόσωπα δικαιούνται την προσαύξηση του πρώτου εδαφίου για καθεμία ποσοστιαία μονάδα (1%) εισφοράς που υπερβαίνειτο κάτωθι συνολικό ποσοστό εισφορών ασφαλισμένουκαι εργοδότη: α) 27% για τα πρόσωπα της περίπτωσης ε΄της παρ. 2 του άρθρου 46 του ν. 2084/1992, β) 23,6% γιατα πρόσωπα της περίπτωσης στ΄ της παρ. 2 του άρθρου46 του ν. 2084/1992.».
\subsection*{ Άρθρο 8 }
\paragraph { 0. } Το τελευταίο εδάφιο της παρ. 1 του άρθρου 31 τουν. 4387/2016 αντικαθίσταται ως εξής:«Μέχρι την έκδοση της απόφασης αυτής εξακολουθούν να ισχύουν οι διατάξεις του Κανονισμού Ασθενείαςτου οικείου φορέα κύριας ασφάλισης. Ειδικά για τουςασφαλισμένους του Ταμείου Συντάξεων Προσωπικού -ΗΣΑΠ (ΤΣΠ-ΗΣΑΠ), του Ταμείου Συντάξεων Προσωπικούτης Εθνικής Τραπέζης της Ελλάδος (ΤΣΠ-ΕΤΕ), καθώς καιΤεύχος Α’ 137/13.09.2017του Κλάδου Σύνταξης του Ταμείου Ασφάλισης Προσωπικού - ΟΤΕ (ΤΑΠ-ΟΤΕ), του Κλάδου Ασφάλισης Προσωπικού - ΔΕΗ (ΚΑΠ-ΔΕΗ), του Ταμείου Ασφάλισης Προσωπικού Ασφαλιστικής Εταιρείας «Η ΕΘΝΙΚΗ» (ΤΑΠΑΕ-Ε) καιτου Ταμείου Ασφάλισης Προσωπικού - ΕΤΒΑ (ΤΑΠ-ΕΤΒΑ)που εντάχτηκαν στον Κλάδο Σύνταξης του πρ. ΙΚΑ-ΕΤΑΜ,αρμόδιο όργανο για το χαρακτηρισμό των ατυχημάτωνως εργατικών ή μη και των επαγγελματικών ασθενειώνορίζεται ο Προϊστάμενος του Υποκαταστήματος Μισθωτών του ΕΦΚΑ στην αρμοδιότητα του οποίου ανήκει οτόπος απασχόλησης ή η έδρα της επιχείρησης με τηνοποία συνδέεται με εργασιακή σχέση ο ασφαλισμένος.Για εκκρεμείς υποθέσεις χαρακτηρισμού των ατυχημάτων ως εργατικών ή μη των ενταχθέντων στο πρ. ΙΚΑ -ΕΤΑΜ Ταμείων και Κλάδων του προηγούμενου εδαφίου,πλην του Ταμείου Συντάξεων Προσωπικού της ΕθνικήςΤραπέζης της Ελλάδος (ΤΣΠ-ΕΤΕ), που δεν έχουν χαρακτηριστεί μέχρι τη δημοσίευση του παρόντος νόμου, αρμόδιο όργανο ορίζεται ο Προϊστάμενος ΠεριφερειακήςΔιεύθυνσης Ασφάλισης και Παροχών Υπαλλήλων Τραπεζών και Επιχειρήσεων Κοινής Ωφέλειας του ΕΦΚΑ.».
\subsection*{ Άρθρο 9 }
\paragraph { 0. } Στο τέλος της περίπτωσης δ΄ της παρ. 1 του άρθρου 34του ν. 4387/2016 προστίθενται εδάφια ως εξής:«Η αναγνώριση του χρόνου της παρ. 5 του άρθρου 53του ν. 3518/2006 (Α΄ 272) εξακολουθεί να είναι σε ισχύ.Οι ασφαλισμένοι καταβάλλουν μηνιαία εισφορά ύψους20% επί του 70% του προβλεπόμενου, κατά την υποβολή της αίτησης, κατώτατου βασικού μισθού άγαμουμισθωτού άνω των 25 ετών. Με απόφαση του Υπουργού Εργασίας, Κοινωνικής Ασφάλισης και ΚοινωνικήςΑλληλεγγύης καθορίζεται η διαδικασία καταβολής τηςανωτέρω εισφοράς και κάθε άλλη αναγκαία λεπτομέρειαγια την εφαρμογή της ρύθμισης.».
\subsection*{ Άρθρο 10 }
\paragraph { 1. } Μετά το δεύτερο εδάφιο της περίπτωσης στ΄ τηςπαρ. 3 του άρθρου 38 του ν. 4387/2016 προστίθεταιπερίπτωση ζ΄, με έναρξη ισχύος από 1.1.2017, ως εξής:«ζ. Για πρόσωπα που διορίζονται σε θέσεις μετακλητών υπαλλήλων με σχέση δημοσίου ή ιδιωτικού δικαίου. Εάν ο διορισμός των ανωτέρω προσώπων έχει γίνειμέχρι 31.12.2016 εφαρμόζεται ως προς την εργοδοτικήεισφορά η 111482/0092/30.11.2016 κοινή υπουργικήαπόφαση (Β΄ 4005).
\paragraph { 2. } Τα ανωτέρω εφαρμόζονται και για τα πρόσωπα πουεπέλεξαν να διατηρήσουν το ασφαλιστικό - συνταξιοδοτικό καθεστώς κύριας, επικουρικής ασφάλισης, εφάπαξ παροχής και υγειονομικής περίθαλψης, στο οποίουπάγονταν πριν από το διορισμό τους στις θέσεις αυτές,σύμφωνα με την περίπτωση α΄της παρ. 11 του άρθρου 4του ν. 4151/2013 (Α΄ 103). Στην περίπτωση αυτή η εφεξήςυπηρεσία τους στις ανωτέρω θέσεις θεωρείται ότι έχειδιανυθεί στο προγενέστερο ασφαλιστικό - συνταξιοδοτικό καθεστώς.».
\subsection*{ Άρθρο 11 }
\paragraph { 1. } Στο τέλος της παρ. 2 του άρθρου 39 του ν. 4387/2016και της παραγράφου 1 του άρθρου 40 προστίθενται εδάφια ως εξής:«Οι ασφαλισμένοι με αίτησή τους, που υποβάλλεταιστον ΕΦΚΑ οποτεδήποτε, μπορούν να επιλέξουν ταως άνω ποσοστά να υπολογίζονται επί ανώτερης βάσης υπολογισμού από εκείνη που προκύπτει βάσει τουμηνιαίου εισοδήματός τους, όπως αυτό καθορίζεταισύμφωνα με τις διατάξεις της παρούσας παραγράφου.Στην περίπτωση αυτή το ύψος της βάσης υπολογισμού,ο κλάδος υπέρ του οποίου θα εισφέρει, καθώς και τοχρονικό διάστημα εφαρμογής της επιλέγεται από τουςασφαλισμένους με την ως άνω αίτησή τους, με την επιφύλαξη για το ανώτατο όριο ασφαλιστέου μηνιαίου εισοδήματος. Η εφαρμογή της νέας βάσης υπολογισμούαρχίζει από την πρώτη του επόμενου μήνα υποβολήςτης αίτησης και παύει να ισχύει και πριν τη παρέλευσητου ορισθέντος σύμφωνα με τα ανωτέρω χρονικού διαστήματος αυτοδικαίως, οποτεδήποτε προκύψει ανώτερηβάση υπολογισμού βάσει του μηνιαίου εισοδήματος σεσχέση με την επιλεγείσα, καθώς και από τον επόμενομήνα από την ανάκληση της αίτησης ή την υποβολή νέαςαίτησης εκ μέρους του ασφαλισμένου. Με απόφαση τουΥπουργού Εργασίας, Κοινωνικής Ασφάλισης και Κοινωνικής Αλληλεγγύης καθορίζεται κάθε άλλη αναγκαίαλεπτομέρεια για την εφαρμογή της διάταξης αυτής.».
\paragraph { 2. } Στο τέλος της παρ. 2 του άρθρου 39 του ν. 4387/2016,όπως τροποποιείται με το παρόν, μετά τη φράση «με τηνεπιφύλαξη για το ανώτατο όριο ασφαλιστέου μηνιαίουεισοδήματος.» προστίθεται η φράση «Στην περίπτωσηαυτή δεν εφαρμόζεται η ρύθμιση του άρθρου 98 τουπαρόντος νόμου.».
\paragraph { 3. } Στο τέλος της παρ. 9 του άρθρου 39 του ν. 4387/2016προστίθενται εδάφια ως εξής:«Σε περίπτωση αμφισβήτησης της υπαγωγής ενόςπροσώπου στη ρύθμιση της παραγράφου αυτής, μπορούν να υποβληθούν από οποιονδήποτε συμβαλλόμενοαντιρρήσεις ενώπιον του ΕΦΚΑ. Με απόφαση του Υπουργού Εργασίας, Κοινωνικής Ασφάλισης και ΚοινωνικήςΑλληλεγγύης καθορίζονται η εφαρμοστέα διαδικασίααντιρρήσεων και ο τρόπος έκδοσης των σχετικών αποφάσεων.».
\subsection*{ Άρθρο 12 }
\paragraph { 1. } Στο τέλος της περίπτωσης ε΄ της παρ. 1 του άρθρου52 του ν. 4387/2016 προστίθεται εδάφιο ως εξής:«Τα θέματα αυτά, πλην της σύστασης οργανικών μονάδων, μπορούν να ρυθμίζονται και με απόφαση τουΥπουργού Εργασίας, Κοινωνικής Ασφάλισης και Κοινωνικής Αλληλεγγύης, μετά από πρόταση του Διοικητή τουΕΦΚΑ.»
\paragraph { 2. } Στο τέλος της περίπτωσης θ΄ της παρ. 1 του άρθρου52 του ν. 4387/2016 προστίθεται εδάφιο ως εξής:«Με απόφαση του Υπουργού Εργασίας, ΚοινωνικήςΑσφάλισης και Κοινωνικής Αλληλεγγύης, μετά από πρόταση του Διοικητή του ΕΦΚΑ, μπορεί να ρυθμίζονται θέματα λειτουργίας του, στο πλαίσιο εφαρμογής του προεδρικού διατάγματος, για την οργανωτική του διάρθρωσηκαι την αποτελεσματική διαχείριση των εκκρεμοτήτων,και ιδίως ο χρόνος έναρξης λειτουργίας των οργανικώντου μονάδων και ζητήματα εσωτερικής λειτουργίας.».
\subsection*{ Άρθρο 13 }
\paragraph { 1. } Η περίπτωση θ΄ της παρ. 4 του άρθρου 57 τουν. 4387/2016 αντικαθίσταται, από τότε που ίσχυσε, ωςεξής:«θ. Έναν (1) ειδικό επιστήμονα με τον αναπληρωτήτου, εξειδικευμένο σε θέματα κοινωνικής ασφάλισης καιπροστασίας που ορίζεται από τον Υπουργό Εργασίας,Κοινωνικής Ασφάλισης και Κοινωνικής Αλληλεγγύης.Απαραίτητο προσόν για τη θέση αυτή αποτελεί η αποδεδειγμένη ακαδημαϊκή ή επαγγελματική εξειδίκευση σεθέματα κοινωνικής ασφάλισης ή η ιδιότητα του μέλουςτου Νομικού Συμβουλίου του Κράτους, με εξειδίκευσησε θέματα κοινωνικής ασφάλισης.».
\paragraph { 2. } Το τελευταίο εδάφιο της παρ. 6 του άρθρου 57 τουν. 4387/2016 αντικαθίσταται, από τότε που ίσχυσε, ωςεξής:«Με απόφαση του Διοικητή ορίζονται κατ’ ανώτατοαριθμό δύο (2) υπάλληλοι του ΕΦΚΑ, με τους αναπληρωτές τους, οι οποίοι εκτελούν χρέη γραμματέα του ΔΣ.».
\subsection*{ Άρθρο 14 }
\paragraph { 0. } Στην παρ. 6 του άρθρου 71 του ν. 4387/2016, μετά τηφράση «πλην εκείνων της Γενικής Διεύθυνσης Χορήγησης Συντάξεων Δημοσίου Τομέα», προστίθεται η φράση«του ΝΑΤ και του ΟΓΑ».
\subsection*{ Άρθρο 15 }
\paragraph { 1. } Το άρθρο 104 του ν. 4387/2016 αντικαθίσταται απότότε που ίσχυσε ως εξής:«1. Αχρεωστήτως καταβληθείσες εισφορές στον ΕΦΚΑσυμψηφίζονται με πάσης φύσεως καθυστερούμενεςοφειλές, ρυθμισμένες ή μη, των δικαιούχων προς τονΕΦΚΑ και τους τρίτους φορείς, για τους οποίους ο ΕΦΚΑσυνεισπράττει εισφορές. Αν δεν υπάρχουν οφειλές ή ανύστερα από τον συμψηφισμό προκύπτει υπόλοιπο ποσό,αυτό επιστρέφεται άτοκα στους δικαιούχους ως εξής:α. Στις περιπτώσεις μισθωτών, ύστερα από αίτηση τωνδικαιούχων.β. Στις περιπτώσεις ελεύθερων επαγγελματιών και αυτοαπασχολούμενων, η επιστροφή γίνεται μετά την ετήσιαεκκαθάριση των οφειλόμενων ασφαλιστικών εισφορών,σύμφωνα με τις 61501/3398/30.12.2016 (Β΄ 4330) και61502/3399/30.12.2016 (Β΄ 4330) υπουργικές αποφάσεις.Τα πρόσωπα αυτά μπορούν, με αίτησή τους, να ζητήσουντο υπερβάλλον ποσό να παραμείνει στον ΕΦΚΑ ως πιστωτικό υπόλοιπο, συμψηφιζόμενο με τις επόμενες εισφορές.
\paragraph { 2. } Οι εισφορές υπέρ τρίτων φορέων που συνεισπράττονται από τον ΕΦΚΑ επιστρέφονται από αυτόν, σύμφωνα με τη διαδικασία της παραγράφου 1 και βαρύνουν τοναντίστοιχο φορέα, υπέρ του οποίου έγινε η είσπραξη.
\paragraph { 3. } Με απόφαση του Υπουργού Εργασίας, ΚοινωνικήςΑσφάλισης και Κοινωνικής Αλληλεγγύης καθορίζεται ηδιαδικασία συμψηφισμού και επιστροφής των εισφορώνκαι κάθε άλλο αναγκαίο ζήτημα για την εφαρμογή τηςδιάταξης αυτής.
\paragraph { 4. } Κάθε αντίθετη με το παρόν διάταξη καταργείται.»2. Οι διατάξεις της προηγούμενης παραγράφου εφαρμόζονται και στις εκκρεμείς κατά την έναρξη ισχύος τουπαρόντος αιτήσεις.
\subsection*{ Άρθρο 16 }
\paragraph { 0. } Η υποπερίπτωση βα΄ της περίπτωσης β΄ της παρ. 5του άρθρου 42 του ν. 4052/2012 (Α΄ 41) αντικαθίσταταιαπό 13.5.2016, ως εξής:«βα. Το τμήμα της σύνταξης που αντιστοιχεί στο χρόνοασφάλισης που έχει πραγματοποιηθεί έως και 31.12.2014υπολογίζεται με βάση ποσοστό αναπλήρωσης, το οποίογια κάθε έτος ασφάλισης αντιστοιχεί σε ποσοστό 0,45%επί των συντάξιμων αποδοχών κάθε ασφαλισμένου πουυπεβλήθησαν σε εισφορές υπέρ επικουρικής ασφάλισης.Ως συντάξιμες αποδοχές νοούνται:βαα. Για τους μισθωτούς, ο μέσος όρος μηνιαίων αποδοχών του ασφαλισμένου από το έτος 2002 έως και τοέτος 2014. Ο μέσος αυτός όρος υπολογίζεται ως το πηλίκον της διαίρεσης του συνόλου των μηνιαίων αποδοχώντου ασφαλισμένου διά του χρόνου ασφάλισής του κατάτο ανωτέρω χρονικό διάστημα. Ως σύνολο μηνιαίων αποδοχών που έλαβε ο ασφαλισμένος νοείται το άθροισματων μηνιαίων αποδοχών που υπόκεινται σε εισφορέςυπέρ επικουρικής ασφάλισης, για το ανωτέρω χρονικόδιάστημα. Για τον υπολογισμό των συντάξιμων αποδοχών λαμβάνονται υπόψη οι αποδοχές του ασφαλισμένου για κάθε ημερολογιακό έτος, αναπροσαρμοζόμενεςσύμφωνα με την παρ. 4 του άρθρου 8 του ν. 4387/2016.βαβ. Για τους αυτοαπασχολούμενους και τους ελεύθερους επαγγελματίες, το εισόδημα, το οποίο υπόκειταισε εισφορές υπέρ επικουρικής ασφάλισης του ασφαλισμένου από το έτος 2002 έως και το έτος 2014. Ως εισόδημα νοείται το ποσό που θα αποτελούσε το ασφαλιστέο μηνιαίο εισόδημα αν εκλαμβανόταν ως μηνιαίαεισφορά το ποσό που πράγματι καταβλήθηκε για κάθεμήνα ασφάλισης κατά το ανωτέρω χρονικό διάστημα.Στο ποσό της ασφαλιστικής εισφοράς που πράγματικαταβλήθηκε για κάθε ασφαλισμένο συνυπολογίζεται,όπου υπήρχε, και η ασφαλιστική εισφορά που έχει καταβληθεί από τον εργοδότη. Για τους ασφαλισμένουςμε ποσό εισφοράς υπέρ επικουρικής ασφάλισης, πουπροκύπτει ανάλογα με την αξία ή την ποσότητα επί τωναγοραζομένων ή πωλουμένων προϊόντων, ο μέσος όροςμηνιαίων τεκμαρτών αποδοχών που προκύπτουν απότην αναγωγή των πραγματικά καταβληθεισών μηνιαίωνασφαλιστικών εισφορών, των ετών 2002 έως και 2014,θεωρώντας ως ποσοστό εισφοράς το 6%. Για τον υπολογισμό των συντάξιμων αποδοχών λαμβάνεται υπόψητο εισόδημα του ασφαλισμένου για κάθε ημερολογιακόέτος, αναπροσαρμοζόμενο σύμφωνα με την παρ. 4 τουάρθρου 8 του ν. 4387/2016.Τεύχος Α’ 137/13.09.2017βαγ. Αν για τον προσδιορισμό των συντάξιμων αποδοχών δεν προκύπτουν ασφαλιστικά στοιχεία από πραγματικό ή πλασματικό χρόνο ασφάλισης ή από προαιρετικήασφάλιση, για χρονικό διάστημα τουλάχιστον πέντε (5)ετών από το έτος 2002 έως το έτος 2014, τότε για τονυπολογισμό των συντάξιμων αποδοχών του τμήματοςτης επικουρικής σύνταξης που αντιστοιχεί στο χρόνοασφάλισης έως και το έτος 2014 αναζητούνται τα ασφαλιστικά στοιχεία και κατά το πριν το έτος 2002 χρονικόδιάστημα μέχρι τη συμπλήρωση συνολικά πέντε (5) ετών.βαδ. Για το χρόνο ασφάλισης που αναγνωρίζεται πλασματικά, ύστερα από την καταβολή του προβλεπόμενουποσού εξαγοράς, ως συντάξιμες αποδοχές ορίζεται τοποσό που θα αποτελούσε τον ασφαλιστέο μηνιαίο μισθό-εισόδημα, αν εκλαμβανόταν ως μηνιαία εισφοράτο ποσό που καταβλήθηκε για την εξαγορά κάθε μήναασφάλισης. Οι πλασματικοί χρόνοι που αναγνωρίστηκανχωρίς εξαγορά δεν συνυπολογίζονται για τον υπολογισμό του ποσού του ανωτέρω τμήματος της σύνταξης.».
\subsection*{ Άρθρο 17 }
\paragraph { 0. } Στο πρώτο εδάφιο της παρ. 1 του άρθρου 62 τουν. 4170/2013 (Α΄ 163), όπως ισχύει, μετά τις λέξεις «τηνΟικονομική Αστυνομία,» προστίθενται οι λέξεις «το σύνολο των υπηρεσιών του Κέντρου Είσπραξης Ασφαλιστικών Οφειλών του ΕΦΚΑ».
\subsection*{ Άρθρο 18 }
\paragraph { 0. } Στο άρθρο 15 του ν. 4174/2013 (Α΄ 170) προστίθεταιπαράγραφος 6, ως εξής:«6. Οι διατάξεις των προηγούμενων παραγράφωνεφαρμόζονται ανάλογα και για τις πληροφορίες πουζητούνται από τις υπηρεσίες του Κέντρου ΕίσπραξηςΑσφαλιστικών Οφειλών (ΚΕΑΟ) στο πλαίσιο των αρμοδιοτήτων τους, για τον έλεγχο της εισπραξιμότητας τωνοφειλών και τον εντοπισμό πηγών αποπληρωμής τωναπαιτήσεων των ασφαλιστικών οργανισμών.».
\subsection*{ Άρθρο 19 }
\paragraph { 0. } Το πρώτο εδάφιο και η περίπτωση α΄ της παρ. 1 τουάρθρου 26 του π.δ. 422/1981 (Α΄ 114) και ο τίτλος αυτούαντικαθίστανται, από 1.1.2017, ως εξής:«Άρθρο 26Κρατήσεις επί των συντάξιμων αποδοχών1. α) Η πάγια μηνιαία κράτηση των μετόχων του Μετοχικού Ταμείου Πολιτικών Υπαλλήλων (MTΠΥ) ορίζεταισε ποσοστό 4,5% επί των συντάξιμων αποδοχών, όπωςαυτές προσδιορίζονται για την κύρια σύνταξη.».
\subsection*{ Άρθρο 20 }
\paragraph { 1. } Οι αυτοαπασχολούμενοι που είναι εγγεγραμμένοι ήθα εγγραφούν στο Τεχνικό Επιμελητήριο Ελλάδας (ΤΕΕ)υπάγονται στην ασφάλιση του ΕΦΚΑ, σύμφωνα με τιςσχετικές νομοθετικές ρυθμίσεις του πρώην Τομέα Σύνταξης Μηχανικών και Εργοληπτών Δημοσίων Έργων(ΤΣΜΕΔΕ) του ΕΤΑΑ και του Ενιαίου Ταμείου ΕπικουρικήςΑσφάλισης και Εφάπαξ Παροχών (ΕΤΕΑΕΠ), σύμφωνα μετις διατάξεις του άρθρου 76 του ν. 4387/2016, από τηνημερομηνία έναρξης άσκησης του επαγγέλματος στηναρμόδια ΔΟΥ και μέχρι τη διακοπή της επαγγελματικήςδραστηριότητας και τη διαγραφή από τη ΔΟΥ.
\paragraph { 2. } Οι δικηγόροι που είναι εγγεγραμμένοι ή θα εγγραφούν στους οικείους δικηγορικούς συλλόγους καιασκούν ελεύθερο επάγγελμα υπάγονται στην ασφάλισητου ΕΦΚΑ, σύμφωνα με τις σχετικές νομοθετικές ρυθμίσεις του πρώην Τομέα Ασφάλισης Νομικών (ΤΑΝ) τουΕΤΑΑ και του Ενιαίου Ταμείου Επικουρικής Ασφάλισηςκαι Εφάπαξ Παροχών (ΕΤΕΑΕΠ), σύμφωνα με τις διατάξεις του άρθρου 76 του ν. 4387/2016, από την ημερομηνία έναρξης άσκησης του επαγγέλματος στην αρμόδιαΔΟΥ και μέχρι τη διακοπή της επαγγελματικής δραστηριότητας και τη διαγραφή από τη ΔΟΥ.
\paragraph { 3. } Η ισχύς των παραγράφου 1 και 2 του παρόντος άρθρου αρχίζει την 1.1.2017. Ασφαλιστικές εισφορές πουέχουν καταβληθεί και αφορούν σε περίοδο ασφάλισηςαπό 1.1.2017 έως την ισχύ της ρύθμισης συμψηφίζονταιή επιστρέφονται, πλην των ασφαλιστικών εισφορών γιαυγειονομική περίθαλψη.
\paragraph { 4. } Τα πρόσωπα των παραγράφων 1 και 2 που έχουνυπαχθεί στην ασφάλιση του ΕΦΚΑ και του ΕΤΕΑΕΠ μέχριτην ισχύ του παρόντος νόμου και για τα οποία προκύπτει,κατ’ εφαρμογή της παρούσας ρύθμισης, διακοπή τηςασφάλισής τους στον ΕΦΚΑ και του ΕΤΕΑΕΠ, μπορούνπροαιρετικά να συνεχίσουν την ασφάλισή τους για το σύνολο των κλάδων ασφάλισης στους οποίους υπάγοντανμέχρι τη διακοπή της υποχρεωτικής τους ασφάλισης.Στην περίπτωση αυτή η μηνιαία ασφαλιστική εισφορά υπολογίζεται με βάση το κατώτατο όριο μηνιαίουεισοδήματος για τους ασφαλισμένους άνω 5ετίας τουάρθρου 39 του ν. 4387/2016.Για την προαιρετική συνέχιση της ασφάλισης, σύμφωνα με τα ανωτέρω υποβάλλεται δήλωση του ασφαλισμένου εντός προθεσμίας τριών (3) μηνών από την έναρξη ισχύος του παρόντος. Η προαιρετική συνέχιση τηςασφάλισης διακόπτεται, ύστερα από αίτηση του ασφαλισμένου, από την πρώτη ημέρα του επόμενου μήνα απότην υποβολή της αίτησης. Νέα αίτηση για προαιρετικήασφάλιση δεν μπορεί να υποβληθεί.
\subsection*{ Άρθρο 21 }
\paragraph { 1. } Η αυτοδίκαιη περιέλευση της ακίνητης περιουσίαςτων ενταχθέντων στο ΕΤΕΑ και ήδη ΕΤΕΑΕΠ Ταμείων,Τομέων, Κλάδων και Λογαριασμών που προβλέπεται στιςπαραγράφους 2, 7 και 8 του άρθρου 45, καθώς και στιςπαραγράφους 5 και 6 του άρθρου 45Α του ν. 4052/2012,όπως προστέθηκε με το άρθρο 82 του ν. 4387/2016:α) σημειώνεται στα αρμόδια υποθηκοφυλακεία χωρίςτη καταβολή τέλους ή δικαιώματος υπέρ Δημοσίου,ΤΑΧΔΙΚ, άμισθου υποθηκοφυλακείου ή οποιουδήποτεάλλου φυσικού ή νομικού προσώπου και β) δηλώνεται,κατ’ εξαίρεση των οριζομένων στο ν. 2308/1995 (Α΄ 114),στα γραφεία κτηματογράφησης και τα κτηματολογικάγραφεία, απρόθεσμα, χωρίς την καταβολή ανταποδοτικού, παγίου ή αναλογικού, τέλους ή οποιουδήποτε άλλουφόρου, τέλους ή δικαιώματος υπέρ της ΕΚΧΑ ΑΕ, τουΔημοσίου ή άλλου νομικού προσώπου.
\paragraph { 2. } Η διαπιστωτική πράξη, που προβλέπεται στην περίπτωση στ΄ της παρ. 1 του άρθρου 29 του ν. 4325/2015(Α΄ 47) για την περιέλευση ακίνητης περιουσίας στοΕΤΕΑ, καθώς και οποιαδήποτε άλλη διαπιστωτική πράξηπροβλέπεται ειδικώς από διάταξη νόμου για την περιέλευση ακίνητης περιουσίας στο ΕΤΕΑΕΠ ή στα ενταχθέντα σε αυτό Ταμεία, Τομείς και Κλάδους, μεταγράφεταιστα αρμόδια υποθηκοφυλακεία χωρίς την καταβολή τέλους ή δικαιώματος υπέρ Δημοσίου, ΤΑΧΔΙΚ, άμισθουυποθηκοφυλακείου ή οποιουδήποτε άλλου φυσικού ήνομικού προσώπου. Τα σχετικά δικαιώματα δηλώνονται,κατ’ εξαίρεση των οριζομένων στο ν. 2308/1995, σταγραφεία κτηματογράφησης και τα κτηματολογικά γραφεία, απρόθεσμα, χωρίς την καταβολή ανταποδοτικού,παγίου ή αναλογικού τέλους ή οποιουδήποτε άλλουφόρου, τέλους ή δικαιώματος υπέρ της ΕΚΧΑ ΑΕ, τουΔημοσίου ή άλλου νομικού προσώπου.
\subsection*{ Άρθρο 22 }
\paragraph { 1. } Για τα πρόσωπα της παρούσας παραγράφου καταβάλλονται υπέρ του ΕΦΚΑ και του ΕΤΕΑΕΠ ασφαλιστικέςεισφορές για κύρια ασφάλιση, υγειονομική περίθαλψη,επικουρική ασφάλιση και εφάπαξ παροχή, εφόσον προκύπτει υποχρέωση ασφάλισής τους στους σχετικούς κλάδους ασφάλισης, στις εξής περιπτώσεις:α. Δημάρχων που είχαν αποκτήσει την ιδιότητα πριναπό την ισχύ του ν. 4093/2012 (Α΄ 222) και ο ανωτέρωχρόνος ασφάλισης θεωρείται ότι έχει διανυθεί στηνασφάλιση του Δημοσίου ή του ΕΤΕΑΕΠ (πρώην τομείςΤΑΔΚΥ για επικουρική ασφάλιση και ΤΠΔΚΥ για εφάπαξπαροχή).Για όσους από τους δημάρχους αυτούς εμπίπτουν στιςρυθμίσεις της παρ. 1 του άρθρου 93 του ν. 3852/2010(Α΄ 87), ο ανωτέρω χρόνος ασφάλισης δύναται να θεωρηθεί ως χρόνος ασφάλισης στο προγενέστερο ασφαλιστικό - συνταξιοδοτικό τους καθεστώς ασφάλισης.Αν ασκούν παράλληλα επαγγελματική δραστηριότηταέχουν εφαρμογή οι ρυθμίσεις της παρ. 1 του άρθρου 17και της παρ. 1 του άρθρου 36 του ν. 4387/2016.Αναφορικά με την υγειονομική τους περίθαλψη, οι δήμαρχοι της παρούσας περίπτωσης που εμπίπτουν στιςρυθμίσεις της παρ. 1 του άρθρου 93 του ν. 3852/2010(Α΄ 87) ή συνεχίζουν να ασκούν παράλληλα και επαγγελματική δραστηριότητα διατηρούν το προγενέστεροκαθεστώς υγειονομικής περίθαλψης. Αν δεν είχαν προγενέστερη ασφάλιση υπάγονται για υγειονομική περίθαλψη στον τομέα του πρώην ΤΥΔΚΥ του ΕΦΚΑ.β. Δημάρχων που αποκτούν για πρώτη φορά την ιδιότητα μετά την ισχύ του ν. 4093/2012 και ο ανωτέρωχρόνος ασφάλισης θεωρείται ότι έχει διανυθεί στο προγενέστερο ασφαλιστικό - συνταξιοδοτικό τους καθεστώς(κύριας και επικουρικής ασφάλισης, εφάπαξ παροχής) ήσε εκείνο που προκύπτει κατ’ εφαρμογή των παραγράφων 1, 2 και 3 του άρθρου 2 του ν. 3865/2010.Αν τα πρόσωπα της παρούσας περίπτωσης συνεχίζουνπαράλληλα να ασκούν την προηγούμενη επαγγελματικήτους δραστηριότητα έχουν εφαρμογή οι ρυθμίσεις τηςπαρ. 2 του άρθρου 36 του ν. 4387/2016 ή οι ρυθμίσειςτης περίπτωσης β΄ της παρ. 2 του άρθρου 5 και της περίπτωσης β΄ της παρ. 2 του άρθρου 38 του ν. 4387/2016και ο ανωτέρω χρόνος ασφάλισης θεωρείται ότι έχει διανυθεί στο προγενέστερο ασφαλιστικό - συνταξιοδοτικότους καθεστώς (κύριας και επικουρικής ασφάλισης καιεφάπαξ παροχής). Εάν παράλληλα ασκούν διαφορετικήτης προηγούμενης επαγγελματική δραστηριότητα έχουνεφαρμογή οι ρυθμίσεις της παρ. 1 του άρθρου 17 και τηςπαρ. 1 του άρθρου 36 του ν. 4387/2016.Αναφορικά με την υγειονομική τους περίθαλψη, οι δήμαρχοι της παρούσας περίπτωσης, που εμπίπτουν στιςρυθμίσεις της παρ. 1 του άρθρου 93 του ν. 3852/2010(Α΄ 87) ή συνεχίζουν να ασκούν παράλληλα και επαγγελματική δραστηριότητα, διατηρούν το προγενέστεροκαθεστώς υγειονομικής περίθαλψης. Αν δεν είχαν προγενέστερη ασφάλιση υπάγονται για υγειονομική περίθαλψη στον τομέα του πρώην ΤΥΔΚΥ του ΕΦΚΑ.γ. Αιρετών οργάνων α΄ και β΄ βαθμού, πλην δημάρχωνκαι περιφερειαρχών, καθώς και του Συμπαραστάτη τουΔημότη και της Επιχείρησης και του Συμπαραστάτη τουΠολίτη και της Επιχείρησης, που εμπίπτουν στις ρυθμίσεις της παρ. 1 του άρθρου 93 και της παρ. 1 του άρθρου182 του ν. 3852/2010 ή στις ρυθμίσεις του άρθρου 51 τουν. 4144/2013 (Α΄ 88). Ο ανωτέρω χρόνος ασφάλισης θεωρείται ότι έχει διανυθεί στο προγενέστερο ασφαλιστικό -συνταξιοδοτικό τους καθεστώς (κύριας και επικουρικήςασφάλισης, εφάπαξ παροχής) ή σε εκείνο που προκύπτεικατ’ εφαρμογή των ρυθμίσεων της παρ. 2 του άρθρου51 του ν. 4144/2013.Αν τα πρόσωπα της παρούσας περίπτωσης ασκούνπαράλληλα επαγγελματική δραστηριότητα, έχουνεφαρμογή οι ρυθμίσεις της παρ. 1 του άρθρου 17 καιτης παρ. 1 του άρθρου 36 του ν. 4387/2016 και ως αιρετάόργανα υποχρεούνται να καταβάλλουν ασφαλιστικές εισφορές στον ΕΦΚΑ για κύρια ασφάλιση και υγειονομικήπερίθαλψη, σύμφωνα με τις παραγράφους 2 και 3 τουπαρόντος άρθρου.Αναφορικά με την υγειονομική τους περίθαλψη, τααιρετά όργανα της παρούσας περίπτωσης που εμπίπτουν στις ρυθμίσεις της παρ. 1 του άρθρου 93 και τηςπαρ. 1 του άρθρου 182 του ν. 3852/2010 ή συνεχίζουννα ασκούν παράλληλα και επαγγελματική δραστηριότητα, διατηρούν το προγενέστερο καθεστώς υγειονομικήςπερίθαλψης ή του ΙΚΑ - ΕΤΑΜ σύμφωνα με την παρ. 2 τουάρθρου 51 παρ. 2 του ν. 4144/2013.δ. Περιφερειαρχών που εμπίπτουν στις ρυθμίσεις τωνάρθρων 119 και 182 του ν. 3852/2010 και θεωρείται ότισυνεχίζουν να υπάγονται στην ασφάλιση του προγενέστερου ασφαλιστικού - συνταξιοδοτικού καθεστώς(κύριας και επικουρικής ασφάλισης, υγειονομικής περίθαλψης, και εφάπαξ παροχής).ε. Βουλευτών που είχαν αποκτήσει την ιδιότητα πριναπό την ισχύ του ν. 4093/2012 και ο ανωτέρω χρόνοςΤεύχος Α’ 137/13.09.2017ασφάλισης θεωρείται ότι έχει διανυθεί στην ασφάλισητου Δημοσίου.Αν τα πρόσωπα της παρούσας περίπτωσης ασκούν παράλληλα επαγγελματική δραστηριότητα, έχουν εφαρμογή οι ρυθμίσεις της παρ. 1 του άρθρου 17 και της παρ. 1του άρθρου 36 του ν. 4387/2016.Αναφορικά με την υγειονομική τους περίθαλψη, ανασκούν παράλληλα επαγγελματική δραστηριότητα, διατηρούν το προγενέστερο καθεστώς υγειονομικής περίθαλψης και εάν δεν είχαν προγενέστερη ασφάλιση υπάγονται στον ΕΦΚΑ (πρώην ΟΠΑΔ του πρώην ΙΚΑ - ΕΤΑΜ).στ. Βουλευτών που έχουν αποκτήσει για πρώτη φοράτην ιδιότητα μετά από την ισχύ του ν. 4093/2012 και οανωτέρω χρόνος ασφάλισης θεωρείται ότι έχει διανυθείστο προγενέστερο ασφαλιστικό - συνταξιοδοτικό τουςκαθεστώς (κύριας και επικουρικής ασφάλισης, εφάπαξπαροχής) ή σε εκείνο που προκύπτει κατ’ εφαρμογή τωνπαραγράφων 1, 2 και 3 του άρθρου 2 του ν. 3865/2010.Αν τα πρόσωπα της παρούσας περίπτωσης συνεχίζουνπαράλληλα να ασκούν την προηγούμενη επαγγελματικήτους δραστηριότητα, έχουν εφαρμογή οι ρυθμίσεις τηςπαρ. 2 του άρθρου 36 του ν. 4387/2016 ή οι ρυθμίσειςτης περίπτωσης β΄ της παρ. 2 του άρθρου 5 και της περίπτωσης β΄ της παρ. 2 του άρθρου 38 του ν. 4387/2016και ο ανωτέρω χρόνος ασφάλισης θεωρείται ότι έχει διανυθεί στο προγενέστερο ασφαλιστικό - συνταξιοδοτικότους καθεστώς (κύριας και επικουρικής ασφάλισης καιεφάπαξ παροχής).Εάν παράλληλα ασκούν διαφορετική της προηγούμενης επαγγελματική δραστηριότητα έχουν εφαρμογή οιρυθμίσεις της παρ. 1 του άρθρου 17 και της παρ. 1 τουάρθρου 36 του ν. 4387/2016.Αναφορικά με την υγειονομική τους περίθαλψη, ανασκούν παράλληλα επαγγελματική δραστηριότητα, διατηρούν το προγενέστερο καθεστώς υγειονομικής περίθαλψης και, εάν δεν είχαν προγενέστερη ασφάλιση,υπάγονται στον ΕΦΚΑ (πρώην ΟΠΑΔ του ΙΚΑ - ΕΤΑΜ).ζ. Προσώπων που διορίζονται σε θέσεις διοίκησης φορέων του Δημοσίου ή του ευρύτερου δημόσιου τομέαή ανεξάρτητων διοικητικών αρχών, τα οποία διατηρούντο προγενέστερο ασφαλιστικό - συνταξιοδοτικό τουςκαθεστώς, σύμφωνα με το άρθρο 6 του ν. 2703/1999(Α΄ 72) ή με άλλες διατάξεις που παραπέμπουν σε αυτό.Τα πρόσωπα της παρούσας περίπτωσης συνεχίζουν ναυπάγονται στην ασφάλιση του προγενέστερου ασφαλιστικού - συνταξιοδοτικού καθεστώς (κύριας και επικουρικής ασφάλισης, υγειονομικής περίθαλψης και εφάπαξπαροχής).Τα ανωτέρω εφαρμόζονται και στα πρόσωπα πουεμπίπτουν στην παρ. 8 του άρθρου 8 του ν. 2227/1994(Α΄ 129) και στην περίπτωση α΄ της παρ. 11 του άρθρου5 του ν. 2703/1999 (Α΄ 72). Οι περιπτώσεις β΄ και γ΄ τηςπαρ. 11 του άρθρου 5 του ν. 2703/1999 καταργούνται.Αν τα πρόσωπα της παρούσας περίπτωσης συνεχίζουνπαράλληλα να ασκούν την προηγούμενη επαγγελματικήτους δραστηριότητα, έχουν εφαρμογή οι ρυθμίσεις τηςπαρ. 2 του άρθρου 36 του ν. 4387/2016 ή οι ρυθμίσειςτης περίπτωσης β΄ της παρ. 2 του άρθρου 5 και της περίπτωσης β΄ της παρ. 2 του άρθρου 38 του ν. 4387/2016και θεωρείται ότι ο χρόνος ασφάλισης έχει διανυθεί στοπρογενέστερο ασφαλιστικό - συνταξιοδοτικό τους καθεστώς (για κύρια και επικουρική ασφάλιση, υγειονομικήπερίθαλψη και εφάπαξ παροχή).Εάν παράλληλα ασκούν διαφορετική της προηγούμενης επαγγελματική δραστηριότητα έχουν εφαρμογή οιρυθμίσεις της παρ. 1 του άρθρου 17 και της παρ. 1 τουάρθρου 36 του ν. 4387/2016.
\paragraph { 2. } Για τα πρόσωπα της παραγράφου 1 καταβάλλονται:α. Για κύρια ασφάλιση, μηνιαία εισφορά στον ΕΦΚΑ,σύμφωνα με τις παραγράφους 1 και 2 του άρθρου 5του ν. 4387/2016, εφαρμοζόμενης της 111482/0092/30.11.2016 κοινής υπουργικής απόφασης (Β΄ 4005) γιαόσα από τα πρόσωπα της παρ. 1 έχουν διοριστεί μέχρι31.12.2016.β. Για υγειονομική περίθαλψη, μηνιαία εισφορά στονΕΦΚΑ, σύμφωνα με την παρ. 1 του άρθρου 41 τουν. 4387/2016.γ. Για επικουρική ασφάλιση καταβάλλεται στον Κλάδο Επικουρικής Ασφάλισης του ΕΤΕΑΕΠ η μηνιαία εισφορά έμμισθων ασφαλισμένων του άρθρου 97 τουν. 4387/2016, υπολογιζόμενη επί της βάσης, όπως αυτήπροσδιορίζεται στην παράγραφο 3 για την κύρια ασφάλιση.Αν τα πρόσωπα της παρούσας περίπτωσης ασκούν παράλληλα επαγγελματική δραστηριότητα έχει εφαρμογήη παρ. 2 του άρθρου 96 του ν. 4387/2016.δ. Για εφάπαξ παροχή καταβάλλεται στον Κλάδο Εφάπαξ Παροχών του ΕΤΕΑΕΠ η μηνιαία εισφορά έμμισθωνασφαλισμένων του άρθρου 35 του ν. 4387/2016, υπολογιζόμενη επί της βάσης, όπως αυτή προσδιορίζεται στηνπαράγραφο 3 του παρόντος για την κύρια ασφάλιση.Αν τα πρόσωπα της παρούσας περίπτωσης ασκούνπαράλληλα επαγγελματική δραστηριότητα, έχουν εφαρμογή οι ρυθμίσεις της παρ. 1 του άρθρου 17 και τωνπαραγράφων 1 και 2 του άρθρου 36 του ν. 4387/2016.
\paragraph { 3. } Η μηνιαία εισφορά για κύρια ασφάλιση, ασφαλισμένου και εργοδότη, υπολογίζεται επί της αντιμισθίας γιατα αιρετά όργανα ή της βουλευτικής αποζημίωσης γιατους βουλευτές ή επί των μηνιαίων αποδοχών ή αποζημίωσης που λαμβάνουν τα λοιπά πρόσωπα της παραγράφου 1. Αν τα πρόσωπα της παραγράφου 1 έχουν επιλέξεινα λαμβάνουν τις αποδοχές της οργανικής τους θέσηςκαι όχι τις προβλεπόμενες για τη θέση που κατέχουν αποδοχές, η ως άνω ασφαλιστική εισφορά υπολογίζεται επίτων συντάξιμων αποδοχών της οργανικής τους θέσηςκαι για υγειονομική περίθαλψη επί των πάσης φύσεωςαποδοχών της οργανικής τους θέσης.Ως προς το ανώτατο όριο αποδοχών επί του οποίουυπολογίζεται η ασφαλιστική εισφορά, έχουν εφαρμογήοι ρυθμίσεις της περίπτωσης α΄ της παρ. 2 του άρθρου5 του ν. 4387/2016.Η εισφορά ασφαλισμένου βαρύνει το πρόσωπο πουέχει διοριστεί στις θέσεις της παραγράφου 1 του παρόντος.Η εργοδοτική εισφορά βαρύνει το Δήμο ή την Περιφέρεια ή την Βουλή των Ελλήνων ή τους φορείς του Δημοσίου ή του ευρύτερου δημόσιου τομέα ή των ανεξάρτητων διοικητικών αρχών, στους οποίους έχουν διοριστείτα πρόσωπα της παραγράφου 1 του παρόντος.
\paragraph { 4. } Αν δεν υφίσταται προγενέστερο ασφαλιστικό καθεστώς για τα πρόσωπα των περιπτώσεων δ΄ και ζ΄ τηςπαραγράφου 1, καταβάλλονται εισφορές για κύρια ασφάλιση και υγειονομική περίθαλψη, σύμφωνα με τις παραγράφους 2 και 3 και ο χρόνος αυτός θεωρείται ότι έχειδιανυθεί στην ασφάλιση του ΕΦΚΑ (πρώην ΙΚΑ - ΕΤΑΜ).
\paragraph { 5. } Στις περιπτώσεις που εφαρμόζονται οι ρυθμίσεις τηςπαρ. 1 του άρθρου 17 και της παραγράφου 1 και 2 τουάρθρου 36 του ν. 4387/2016, η ασφαλιστική εισφοράπου έχει καταβληθεί από τα ανωτέρω πρόσωπα για τηθέση που κατέχουν θεωρείται εισφορά μισθωτού. Για τηνάσκηση επαγγελματικής δραστηριότητας καταβάλλονται ασφαλιστικές εισφορές σύμφωνα με τα άρθρα 36,38, 39, 40 και 41 του ν. 4387/2016.
\paragraph { 6. } Αν στις θέσεις που εμπίπτουν στο πεδίο εφαρμογήςτης παραγράφου 1 διορίζονται συνταξιούχοι του ΕΦΚΑή του Δημοσίου, εφαρμόζεται η παράγραφος 4. Εξαιρούνται τα ανωτέρω πρόσωπα που έχουν παραιτηθείτης αμοιβής που προβλέπεται για τη θέση που κατέχουν.
\paragraph { 7. } Για τα πρόσωπα που εμπίπτουν στις ρυθμίσεις τουάρθρου 25 του ν. 4255/2014 (Α΄ 89) εφαρμόζονται οιρυθμίσεις του παρόντος άρθρου και ο χρόνος ασφάλισης θεωρείται ότι έχει διανυθεί στο προγενέστεροασφαλιστικό - συνταξιοδοτικό καθεστώς (κύριας καιεπικουρικής ασφάλισης, υγειονομικής περίθαλψης καιεφάπαξ παροχής). Για την ασφάλισή τους εφαρμόζονται τα προβλεπόμενα στις παραγράφους 2 και 3 τουπαρόντος άρθρου.
\paragraph { 8. } Η ισχύς του παρόντος άρθρου αρχίζει την 1.1.2017.Αν από την εφαρμογή του παρόντος άρθρου προκύψουνδιαφορές στα ποσά των ασφαλιστικών εισφορών πουπρέπει να καταβληθούν από την ανωτέρω ημερομηνία,εφαρμόζονται τα εξής:α. Εάν το ποσό της ασφαλιστικής εισφοράς που έχεικαταβληθεί είναι υψηλότερο από εκείνο που προκύπτειαπό την εφαρμογή του παρόντος άρθρου, το επιπλέονποσό συμψηφίζεται, μέχρι εξάλειψής του, με μελλοντικέςκαταβολές ασφαλιστικών εισφορών.β. Εάν το ποσό της ασφαλιστικής εισφοράς που έχεικαταβληθεί είναι χαμηλότερο από εκείνο που προκύπτειαπό την εφαρμογή του παρόντος άρθρου, το επιπλέονποσό που πρέπει να καταβληθεί για την τακτοποίησητων εισφορών των προηγούμενων μηνών καταβάλλεται σε δόσεις, ο αριθμός των οποίων είναι ίσος με τοναριθμό των μηνών για τους οποίους προκύπτει διαφοράστο ύψος των εισφορών, χωρίς την επιβολή προστίμωνκαθυστέρησης εξόφλησης.
\paragraph { 9. } Οι διατάξεις της παρ. 1 του άρθρου 51 του ν. 4144/2013εφαρμόζονται από 1.1.2017 και στους ασφαλισμένους τουΕΦΚΑ βάσει των γενικών, ειδικών ή καταστατικών διατάξεων του πρώην ΟΓΑ, εφαρμοζομένων των ρυθμίσεων τηςπερίπτωσης γ΄ της παραγράφου 1 του παρόντος άρθρου.
\subsection*{ Άρθρο 23 }
\paragraph { 0. } 1.α. Γενικές και ειδικές διατάξεις που προβλέπουνδιακοπή ή περικοπή της σύνταξης αναπηρίας ή τηςσύνταξης λόγω θανάτου και των προνοιακών ή άλλωνεπιδομάτων όταν ο δικαιούχος αναλαμβάνει εργασία ήαυτοαπασχολείται, δεν έχουν εφαρμογή στους δικαιούχους που πάσχουν αναπηρίας, η οποία οφείλεται σεψυχική πάθηση ή νοητική υστέρηση ή συμπαθολογίαψυχικής πάθησης και νοητικής υστέρησης, με ποσοστό50% και άνω, εφόσον η ανάληψη μισθωτής απασχόλησης ή η αυτοαπασχόληση ενδείκνυται για λόγους ψυχοκοινωνικής αποκατάστασης και κοινωνικής επανένταξηςκαι η κρίση αυτή πιστοποιείται με γνωμάτευση μονάδαςψυχικής υγείας, η οποία θα ισχύει για τρία (3) έτη, τουαντίστοιχου Τομέα Ψυχικής Υγείας, σύμφωνα με τα οριζόμενα στο ν. 2716/1999 (Α΄ 96).β. Η ανάληψη μισθωτής απασχόλησης ή η αυτοαπασχόληση των ανωτέρω προσώπων για λόγους ψυχοκοινωνικής αποκατάστασης και κοινωνικής επανένταξης,δεν επηρεάζει την σχετική κρίση αξιολόγησης αναπηρίας περί ανικανότητας για κάθε βιοποριστική εργασία,κατά τη διαδικασία πιστοποίησης αναπηρίας από ταΚΕΠΑ.2. Η περίπτωση α΄ της παραγράφου 1 δεν εφαρμόζεταισε συνταξιούχους που αναλαμβάνουν μόνιμη ή με σχέσηιδιωτικού δικαίου αορίστου χρόνου εργασία σε φορέατης Γενικής Κυβέρνησης.3. Οι συνταξιούχοι της παραγράφου 1 υποχρεούνταιπριν αναλάβουν εργασία ή αυτοαπασχοληθούν να τοδηλώσουν στον ΕΦΚΑ, στο ΕΤΕΑΕΠ και στις αρμόδιεςυπηρεσίες πρόνοιας.4. Με απόφαση του Υπουργού Εργασίας, ΚοινωνικήςΑσφάλισης και Κοινωνικής Αλληλεγγύης μπορεί να καθορίζεται κάθε λεπτομέρεια εφαρμογής του παρόντοςάρθρου.
\subsection*{ Άρθρο 24 }
\paragraph { 0. } Με απόφαση του Υπουργού Εργασίας, ΚοινωνικήςΑσφάλισης και Κοινωνικής Αλληλεγγύης επιτρέπεταιμέχρι 31.12.2017 και κατά παρέκκλιση των κειμένωνδιατάξεων, η μεταφορά υπαλλήλων μεταξύ των ΕΦΚΑ,ΕΤΕΑΕΠ, ΟΓΑ και ΝΑΤ, με μεταφορά της οργανικής τουςθέσης, την ίδια εργασιακή σχέση, κλάδο ή ειδικότητα,βαθμό και μισθολογικό κλιμάκιο, με συνεκτίμηση τηςαίτησής τους.Ομοίως επιτρέπεται για αιτήσεις που υποβάλλονταιμέχρι 31.12.2017 και κατά παρέκκλιση των κειμένων διατάξεων, η μεταφορά υπαλλήλων των ΕΦΚΑ, ΕΤΕΑΕΠ, ΟΓΑκαι ΝΑΤ σε κενές οργανικές θέσεις της Κεντρικής Υπηρεσίας του Υπουργείου Εργασίας, Κοινωνικής Ασφάλισηςκαι Κοινωνικής Αλληλεγγύης, με την ίδια εργασιακή σχέση, κλάδο ή ειδικότητα, βαθμό και μισθολογικό κλιμάκιο.Αν δεν υπάρχουν κενές οργανικές θέσεις, οι υπάλληλοιμεταφέρονται με την οργανική τους θέση και υποχρεούνται να υπηρετήσουν για μία τουλάχιστον πενταετία στηΓενική Διεύθυνση Κοινωνικής Ασφάλισης.
\subsection*{ Άρθρο 25 }
\paragraph { 0. } Επιτρέπεται η μετάταξη του προσωπικού ασφαλείαςπου υπηρετεί στον ΕΦΚΑ σε θέση άλλου κλάδου ή ειδικόΤεύχος Α’ 137/13.09.2017τητας της ίδιας ή ανώτερης κατηγορίας ή εκπαιδευτικήςβαθμίδας, με την ίδια σχέση εργασίας και ανεξαρτήτωςορίου ηλικίας. Αν δεν υπάρχει κενή θέση, η μετάταξηγίνεται με μεταφορά της θέσης που κατέχει ο μετατασσόμενος σε κλάδο ή ειδικότητα της υπηρεσίας του,αντίστοιχης εκπαιδευτικής βαθμίδας του τίτλου σπουδών που κατέχει, ύστερα από γνώμη του υπηρεσιακούσυμβουλίου. Κατά τα λοιπά εφαρμόζεται αναλογικά ηκείμενη περί μετατάξεων νομοθεσία.
\subsection*{ Άρθρο 26 }
\paragraph { 1. } Ορίζονται στον ΕΦΚΑ ως «Προϊστάμενη Αρχή» ηΔιεύθυνση Στέγασης και Τεχνικών Υπηρεσιών του ΕΦΚΑ,ως «Διευθύνουσα Υπηρεσία» το Τμήμα Κατασκευών τουΕΦΚΑ και ως «Τεχνικό Συμβούλιο» το Τεχνικό Συμβούλιοτου ΙΚΑ - ΕΤΑΜ, σύμφωνα με το ν. 1418/1984 (Α΄ 23).Κατά τα λοιπά, εφαρμόζεται αναλογικά το π.δ. 170/1987(Α΄ 84). Με απόφαση του Υπουργού Εργασίας, Κοινωνικής Ασφάλισης και Κοινωνικής Αλληλεγγύης ρυθμίζονται ειδικότερα θέματα για την εφαρμογή του παρόντος,ιδίως ο ορισμός των αποφαινόμενων οργάνων σε περίπτωση τροποποίησης της οργανωτικής διάρθρωσηςτου ΕΦΚΑ.
\paragraph { 2. } Η ισχύς του παρόντος άρθρου αρχίζει την 1.1.2017.
\subsection*{ Άρθρο 27 }
\paragraph { 0. } Προκειμένου για τα πρόσωπα που υπάγονται στηνασφάλιση του ΕΦΚΑ, η καταβαλλόμενη σύνταξη αναστέλλεται για όσο διάστημα ο συνταξιούχος εκτίει περιοριστική της ελευθερίας ποινή μεγαλύτερη του ενός(1) έτους και εφόσον το αδίκημα για το οποίο καταδικάσθηκε στρέφεται κατά ασφαλιστικού φορέα. Στην περίπτωση αυτή, η σύνταξη μεταβιβάζεται στους τυχόνδικαιοδόχους αυτής, σύμφωνα με τις εκάστοτε ισχύουσες διατάξεις για τη συνταξιοδότηση λόγω θανάτου.Τα οικονομικά αποτελέσματα της παρούσας διάταξηςαρχίζουν από την 1η ημέρα του επόμενου μήνα από τηδημοσίευση του παρόντος νόμου. Κάθε αντίθετη διάταξη καταργείται.
\subsection*{ Άρθρο 28 }
\paragraph { 0. } Το άρθρο 66 του ν. 4144/2013 (Α΄ 88) αντικαθίσταταιως εξής:«Άρθρο 66Στις περιπτώσεις λήξης του συνταξιοδοτικού δικαιώματος λόγω αναπηρίας και εφόσον εκκρεμεί στις αρμόδιες υγειονομικές επιτροπές ΚΕΠΑ ιατρική κρίση, χωρίςυπαιτιότητα των ασφαλισμένων, το δικαίωμα συνταξιοδότησής τους λόγω αναπηρίας παρατείνεται για έξι (6)μήνες, με το ίδιο ποσό που ελάμβαναν οι συνταξιούχοι πριν από τη λήξη του δικαιώματος, εφόσον κατά τοπροηγούμενο χρονικό διάστημα είχαν κριθεί από τηνυγειονομική επιτροπή με ποσοστό αναπηρίας τουλάχιστον 67%. Εάν κριθεί από τις υγειονομικές επιτροπές τωνΚΕΠΑ ότι αυτοί δεν φέρουν συντάξιμο ποσοστό αναπηρίας ή φέρουν μικρότερο ποσοστό αναπηρίας από τοπρογενεστέρως κριθέν, οι αχρεωστήτως καταβληθείσεςπαροχές αναζητούνται άτοκα, δια συμψηφισμού, με μηνιαία παρακράτηση 20% από τις τυχόν χορηγούμενεςσυνταξιοδοτικές παροχές, ενώ στην περίπτωση που δενχορηγούνται παροχές, αναζητούνται σύμφωνα με τιςδιατάξεις του ΚΕΔΕ.Η ανωτέρω ρύθμιση εφαρμόζεται και για όλα τα επιδόματα που χορηγούνται λόγω αναπηρίας, καθώς καιγια τις συντάξεις με αιτία την αναπηρία, ενώ παράλληλαπαρατείνεται και η ιατροφαρμακευτική περίθαλψη όσωνεμπίπτουν στο παρόν άρθρο. Η ισχύς του άρθρου αυτούαρχίζει την 1.7.2017.».
\subsection*{ Άρθρο 29 }
\paragraph { 0. } Το άρθρο 30 του ν. 1469/1984 (Α΄ 111) αντικαθίσταταιως εξής:«1. Ο/η ανασφάλιστος/η διαζευγμένος/η σύζυγοςδικαιούται να διατηρήσει, ως άμεσα ασφαλισμένος, τοδικαίωμα παροχών ασθένειας σε είδος από τον ΕΟΠΥΥή άλλο φορέα κοινωνικής ασφάλισης ή το Δημόσιο, πουείχε κατά το χρόνο λύσης του γάμου ο έτερος σύζυγος,εφόσον συντρέχουν σωρευτικά οι εξής προϋποθέσεις:α) Ο γάμος λύθηκε μετά τη συμπλήρωση του 25ουέτους της ηλικίας του, καιβ) δεν καλύπτεται άμεσα ή έμμεσα για παροχές ασθένειας σε είδος από τον ΕΟΠΥΥ ή άλλο ασφαλιστικό φορέα ή το Δημόσιο.2. Για τη διατήρηση του δικαιώματος των παροχώνασθένειας σε είδος, σύμφωνα με την παράγραφο 1, ο ενδιαφερόμενος καταβάλλει μηνιαία ασφαλιστική εισφοράυπέρ υγειονομικής περίθαλψης ως εξής:α) Στον ΕΦΚΑ, για παροχές σε είδος από τον ΕΟΠΥΥ, σεποσοστό 6,45% επί του ποσού που αντιστοιχεί στον εκάστοτε βασικό μισθό άγαμου μισθωτού άνω των 25 ετών,β) στο Δημόσιο σε ποσοστό 6,45% επί του ποσού πουαντιστοιχεί στον εκάστοτε βασικό μισθό άγαμου μισθωτού άνω των 25 ετών,γ) στους λοιπούς ασφαλιστικούς φορείς, την προβλεπόμενη από τους οικείους κανονισμούς εισφορά ασφαλισμένου και εργοδότη υπέρ υγειονομικής περίθαλψηςπου υπολογίζεται επί του ποσού που αντιστοιχεί στονεκάστοτε βασικό μισθό άγαμου μισθωτού άνω των 25ετών.3. Το ασφαλιστικό δικαίωμα ασκείται από τον ενδιαφερόμενο μέσα σε ένα (1) έτος από την ημερομηνίαέκδοσης της οριστικής απόφασης διαζυγίου.».
\subsection*{ Άρθρο 30 }
\paragraph { 0. } προϋποθέσεις που ορίζει κοινή υπουργική απόφαση τωνΥπουργών Εργασίας, Κοινωνικής Ασφάλισης και Κοινωνικής Αλληλεγγύης, Εσωτερικών και Υγείας.ΜΕΡΟΣ Β΄
\subsection*{ Άρθρο 31 }
\paragraph { 1. } Στo τέλος της περίπτωσης α΄ της παρ. 2 του άρθρου2 του ν. 3996/2011 (Α΄ 170) προστίθενται εδάφια ως εξής:«Ειδικά, κατά τη διενέργεια ελέγχου, σύμφωνα με τιςυποπεριπτώσεις αα΄, ββ΄, γγ΄ και δδ΄, οι ευρισκόμενοιστο χώρο εργασίας υποχρεούνται να επιδεικνύουν τηναστυνομική τους ταυτότητα ή άλλο αποδεικτικό τηςταυτοπροσωπίας έγγραφο, εφόσον τους ζητηθεί απότους διενεργούντες τον έλεγχο Επιθεωρητές Εργασίας.Στον εργοδότη που αρνείται την κατά τα προηγούμεναεδάφια είσοδο και πρόσβαση ή την παροχή των ως άνωστοιχείων ή πληροφοριών ή παρέχει ανακριβείς πληροφορίες ή στοιχεία επιβάλλονται οι διοικητικές κυρώσειςτης υποπαραγράφου 1Α του άρθρου 24.».
\paragraph { 2. } Το τελευταίο εδάφιο της περίπτωσης δ΄ της παρ. 2του άρθρου 2 του ν. 3996/2011 αντικαθίσταται ως εξής:«Στον εργοδότη ή οποιονδήποτε τρίτο αρνείται τηνκατά τα προηγούμενα εδάφια είσοδο και πρόσβαση ήτην παροχή στοιχείων ή πληροφοριών ή παρέχει ανακριβείς πληροφορίες ή στοιχεία επιβάλλονται οι διοικητικέςκυρώσεις της υποπαραγράφου 1Α του άρθρου 24.».
\subsection*{ Άρθρο 32 }
\paragraph { 0. } Η περίπτωση α΄ της παρ. 3 του άρθρου 23 τουν. 3996/2011 αντικαθίσταται ως εξής:«3.α. Αν κρίνει ότι υπάρχουν παραβάσεις που εγκυμονούν άμεσο ή σοβαρό κίνδυνο για την ασφάλεια καιτην υγεία των εργαζομένων, διακόπτει προσωρινά τηλειτουργία της επιχείρησης ή εκμετάλλευσης ή τμήματόςτης ή στοιχείου του εξοπλισμού της, για χρονικό διάστημα μέχρι της πλήρους συμμόρφωσης του εργοδότηκαι της άρσης των παραβάσεων που διαπιστώθηκαν.Αν η επιχείρηση, μετά την προσωρινή διακοπή ή τηνεπιβολή άλλων διοικητικών κυρώσεων, εξακολουθεί ναπαραβαίνει συστηματικά τις διατάξεις της νομοθεσίας μεάμεσο ή σοβαρό κίνδυνο για τους εργαζόμενους εισηγείται στον Υπουργό Εργασίας, Κοινωνικής Ασφάλισηςκαι Κοινωνικής Αλληλεγγύης την οριστική διακοπή τηςλειτουργίας της.».
\subsection*{ Άρθρο 33 }
\paragraph { 0. } Η υποπαράγραφος 1Β του άρθρου 24 του ν. 3996/2011αντικαθίσταται ως εξής:«Β.α. Προσωρινή διακοπή της λειτουργίας συγκεκριμένης παραγωγικής διαδικασίας ή τμήματος ή τμημάτων ήτου συνόλου της επιχείρησης ή εκμετάλλευσης για χρονικό διάστημα μέχρι τριών (3) ημερών, εφόσον επιβληθούν σε βάρος του εργοδότη τρεις (3) πράξεις επιβολήςπροστίμου για παραβάσεις της εργατικής νομοθεσίαςπου χαρακτηρίζονται, σύμφωνα με την υπουργική απόφαση 2063/Δ1632/2011 (Β΄ 266), ως «υψηλής» ή «πολύυψηλής» σοβαρότητας, και οι οποίες διαπιστώνονταιαθροιστικά σε τρεις (3) διενεργούμενους ελέγχους σεδιαφορετικές ημερομηνίες, μέσα σε χρονικό διάστημαδύο (2) ετών.β. Προσωρινή διακοπή της λειτουργίας συγκεκριμένηςπαραγωγικής διαδικασίας ή τμήματος ή τμημάτων ή τουσυνόλου της επιχείρησης ή εκμετάλλευσης για χρονικόδιάστημα τεσσάρων (4) μέχρι πέντε (5) ημερών, εφόσονεπιβληθούν σε βάρος του εργοδότη τέσσερις (4) ή περισσότερες πράξεις επιβολής προστίμου για παραβάσειςτης εργατικής νομοθεσίας που χαρακτηρίζονται, σύμφωνα με την υπουργική απόφαση 2063/Δ1632/2011, ως«υψηλής» ή «πολύ υψηλής» σοβαρότητας, και οι οποίεςδιαπιστώνονται αθροιστικά σε τέσσερις (4) διενεργούμενους ελέγχους σε διαφορετικές ημερομηνίες, μέσα σεχρονικό διάστημα δύο (2) ετών.γ. Προσωρινή διακοπή της λειτουργίας συγκεκριμένηςπαραγωγικής διαδικασίας ή τμήματος ή τμημάτων ή τουσυνόλου της επιχείρησης ή εκμετάλλευσης για χρονικόδιάστημα μέχρι τριών (3) ημερών, εφόσον επιβληθούνσε βάρος του εργοδότη δύο (2) ή περισσότερες πράξειςεπιβολής προστίμου για αδήλωτη εργασία και οι οποίεςδιαπιστώνονται αθροιστικά σε δύο (2) διενεργούμενουςελέγχους σε διαφορετικές ημερομηνίες, μέσα σε χρονικόδιάστημα δύο (2) ετών. Σε περιπτώσεις που η επιχείρησηδιατηρεί τμήματα, εκμεταλλεύσεις ή υποκαταστήματασε διαφορετικούς τόπους, το μέτρο της προσωρινήςδιακοπής λειτουργίας λαμβάνεται για το αντίστοιχοτμήμα ή εκμετάλλευση ή υποκατάστημα, του τελευταίου χρονολογικά διενεργούντος ελέγχου που επισύρειπράξη επιβολής προστίμου «υψηλής» ή «πολύ υψηλής»σοβαρότητας.Οι κυρώσεις των περιπτώσεων α΄ έως γ΄ επιβάλλονταιμε αιτιολογημένη πράξη του Ειδικού Επιθεωρητή ή μεαιτιολογημένη πράξη του Διευθυντή της αρμόδιας Περιφερειακής Διεύθυνσης Επιθεώρησης του ΣΕΠΕ, ύστερααπό αιτιολογημένη εισήγηση του αρμόδιου ΕπιθεωρητήΕργασίας.Με απόφαση του Υπουργού Εργασίας, ΚοινωνικήςΑσφάλισης και Κοινωνικής Αλληλεγγύης, ύστερα απόαιτιολογημένη εισήγηση του Διευθυντή της αρμόδιαςΠεριφερειακής Διεύθυνσης Επιθεώρησης του ΣΕΠΕ, μπορεί να επιβληθεί στον εργοδότη προσωρινή διακοπή γιαδιάστημα μεγαλύτερο από πέντε (5) ημέρες ή και οριστική διακοπή της λειτουργίας συγκεκριμένης παραγωγικήςδιαδικασίας ή τμήματος ή τμημάτων ή του συνόλου τηςεπιχείρησης ή εκμετάλλευσης.Όταν συντρέχουν οι προϋποθέσεις της περίπτωσης α΄της παραγράφου 3 του άρθρου 23, δεν απαιτείται πρόσκληση για παροχή εξηγήσεων. Η διακοπή επιβάλλεταιαπό τον Ειδικό Επιθεωρητή ή τον αρμόδιο ΕπιθεωρητήΕργασίας που διενεργεί τον έλεγχο, ύστερα από αιτιολογημένη καταγραφή στο δελτίο ελέγχου των παραβάσεων, που κατά την κρίση τους συνιστούν άμεσο ή σοβαρόκίνδυνο για την ασφάλεια και υγεία των εργαζομένων.Η εκτέλεση της διοικητικής κύρωσης της προσωρινήςή οριστικής διακοπής γίνεται από την αρμόδια αστυνομική αρχή.Τεύχος Α’ 137/13.09.2017Ο χρόνος προσωρινής ή οριστικής διακοπής λογίζεταιως κανονικός χρόνος εργασίας ως προς όλα τα δικαιώματα των εργαζομένων.».
\subsection*{ Άρθρο 34 }
\paragraph { 0. } θεται δεύτερο εδάφιο ως εξής:«Με τις ίδιες ποινές τιμωρείται και ο εργοδότης πουπαραβιάζει την πράξη ή την απόφαση περί προσωρινήςή οριστικής διακοπής λειτουργίας της συγκεκριμένηςπαραγωγικής διαδικασίας ή τμήματος ή τμημάτων ή τουσυνόλου της επιχείρησης ή εκμετάλλευσης ή στοιχείουτου εξοπλισμού της, που του έχει επιβληθεί ως διοικητικήκύρωση για παραβιάσεις της εργατικής νομοθεσίας.».
\subsection*{ Άρθρο 35 }
\paragraph { 0. } θεται τρίτο εδάφιο ως εξής:«Αν από τη σχετική έκθεση διαπιστώνεται η μη αναγραφή εργαζομένου στον ισχύοντα πίνακα προσωπικού,σύμφωνα με το άρθρο 16 του ν. 2874/2000 (Α΄ 286), επιβάλλονται άνευ ετέρου από τον Προϊστάμενο του κατάτόπο αρμόδιου Τμήματος Επιθεώρησης ΕργασιακώνΣχέσεων του Σώματος Επιθεώρησης Εργασίας οι διοικητικές κυρώσεις του άρθρου 1 της υπουργικής απόφασης27397/122/2013 (B΄ 2062).».
\subsection*{ Άρθρο 36 }
\paragraph { 0. } Η παρ. 1 του άρθρου 80 του ν. 4144/2013 αντικαθίσταται ως εξής:«1.α. Ο εργοδότης υποχρεούται να καταχωρεί στοπληροφοριακό σύστημα του Υπουργείου Εργασίας,Κοινωνικής Ασφάλισης και Κοινωνικής Αλληλεγγύης«ΕΡΓΑΝΗ» κάθε αλλαγή ή τροποποίηση του ωραρίου ήτης οργάνωσης του χρόνου εργασίας των εργαζομένων,το αργότερο έως και την ημέρα αλλαγής ή τροποποίησηςτου ωραρίου ή της οργάνωσης του χρόνου εργασίας,και σε κάθε περίπτωση πριν την ανάληψη υπηρεσίαςαπό τους εργαζομένους, καθώς και την υπερεργασία καιτη νόμιμη κατά την ισχύουσα νομοθεσία υπερωριακήαπασχόληση πριν την έναρξη πραγματοποίησής της.β. Αν διαπιστωθεί από τα αρμόδια ελεγκτικά όργανατροποποίηση ωραρίου εργασίας εργαζομένου ή υπερωριακή του απασχόληση χωρίς αυτή να έχει γνωστοποιηθεί κατά τα ανωτέρω πριν την έναρξη πραγματοποίησήςτης, επιβάλλονται με πράξη του αρμόδιου οργάνου σεβάρος του εργοδότη κυρώσεις, σύμφωνα με τα άρθρα24 και 28 του ν. 3996/2011.γ. Με απόφαση του Υπουργού Εργασίας, ΚοινωνικήςΑσφάλισης και Κοινωνικής Αλληλεγγύης καθορίζεται ηδιαδικασία καταχώρισης, τα στοιχεία που γνωστοποιούνται και κάθε άλλη αναγκαία λεπτομέρεια για την εφαρμογή του παρόντος.δ. Η παρ. 1 του άρθρου 55 του ν. 4310/2014 (Α΄ 258)καταργείται και οι υπόλοιπες παράγραφοι του άρθρουαναριθμούνται αντίστοιχα.ε. Η ισχύς της παρούσας παραγράφου αρχίζει από τηνέκδοση της απόφασης της περίπτωσης γ΄.».
\subsection*{ Άρθρο 37 }
\paragraph { 0. } Το άρθρο 33 του ν. 1836/1989 (Α΄ 79) αντικαθίσταταιως εξής:«1. Εργοδότης που εκτελεί οικοδομική εργασία ή τεχνικό έργο, υποχρεούται να αναγγέλλει ηλεκτρονικάστο Ολοκληρωμένο Πληροφοριακό Σύστημα του ΣΕΠΕ(ΟΠΣ-ΣΕΠΕ) το απασχολούμενο προσωπικό, πριν απότην έναρξη κάθε ημερήσιας απασχόλησης. Σε περίπτωση μη συμμόρφωσης με τα ανωτέρω επιβάλλονταιοι διοικητικές κυρώσεις του άρθρου 1 της υπουργικήςαπόφασης 27397/122/2013 (B΄ 2062).2. Αντίγραφο του εντύπου της ως άνω αναγγελίας γιακάθε ημέρα απασχόλησης τηρείται στον τόπο εκτέλεσης του έργου με ευθύνη του εργοδότη και τίθεται στηδιάθεση των ελεγκτικών οργάνων, προς έλεγχο, οσάκιςζητείται. Σε περίπτωση μη τήρησης της ως άνω υποχρέωσης επιβάλλονται οι διοικητικές κυρώσεις του άρθρου24 του ν. 3996/2011 (Α΄ 170).3. Για τους σκοπούς του παρόντος, ως εργοδότηςθεωρείται ο εργολάβος και ο υπεργολάβος που έχουναναλάβει την εκτέλεση της οικοδομικής εργασίας ή τουτεχνικού έργου.4. Με απόφαση του Υπουργού Εργασίας, ΚοινωνικήςΑσφάλισης και Κοινωνικής Αλληλεγγύης καθορίζεται ηδιαδικασία της αναγγελίας, τα στοιχεία που περιλαμβάνει, καθώς και κάθε άλλη αναγκαία λεπτομέρεια για τηνεφαρμογή του παρόντος.5. Η ισχύς του παρόντος άρθρου αρχίζει από την έκδοση της απόφασης της παραγράφου 4.».
\subsection*{ Άρθρο 38 }
\paragraph { 1. } Ο εργοδότης υποχρεούται να αναγγέλλει, με ηλεκτρονική υποβολή των σχετικών εντύπων που προβλέπονται στην υπουργική απόφαση 295/2014 (Β΄ 2390)στο πληροφοριακό σύστημα του Υπουργείου Εργασίας, Κοινωνικής Ασφάλισης και Κοινωνικής Αλληλεγγύης«ΕΡΓΑΝΗ», κάθε περίπτωση οικειοθελούς αποχώρησηςμισθωτού ή καταγγελίας σύμβασης εργασίας αορίστουχρόνου ή λήξης σύμβασης εργασίας ορισμένου χρόνουή έργου το αργότερο τέσσερις (4) εργάσιμες ημέρες απότην ημέρα αποχώρησης του μισθωτού ή καταγγελίας τηςσύμβασης εργασίας αορίστου χρόνου ή λήξης της σύμβασης εργασίας ορισμένου χρόνου ή έργου, αντίστοιχα.
\paragraph { 2. } Η αναγγελία οικειοθελούς αποχώρησης του μισθωτού θα πρέπει να συνοδεύεται υποχρεωτικά είτε απόηλεκτρονικά σαρωμένο έντυπο υπογεγραμμένο από τονεργοδότη και τον εργαζόμενο είτε από εξώδικη δήλωσητου εργοδότη προς τον εργαζόμενο, με την οποία τονενημερώνει ότι έχει χωρήσει οικειοθελής αποχώρησήτου και ότι αυτή θα αναγγελθεί στο πληροφοριακό σύστημα «ΕΡΓΑΝΗ». Στην τελευταία περίπτωση, η εξώδικηδήλωση του εργοδότη επιδίδεται στον εργαζόμενο τοαργότερο τέσσερις (4) εργάσιμες ημέρες από την οικειοθελή του αποχώρηση και η αναγγελία γίνεται την επόμενη εργάσιμη ημέρα από την επίδοση της εξώδικηςδήλωσης. Αν ο εργοδότης δεν τηρήσει εμπρόθεσμα τιςυποχρεώσεις αναγγελίας οικειοθελούς αποχώρησης, συμπεριλαμβανομένης της υποβολής των συνοδευτικώνεγγράφων της παρούσας, η σύμβαση εργασίας θεωρείται ότι λύθηκε με άτακτη καταγγελία του εργοδότη.
\subsection*{ Άρθρο 39 }
\paragraph { 0. } Α) 1. Στην παρ. 2 του άρθρου 73 του ν. 4412/2016(Α΄ 147) μετά την περίπτωση β΄ προστίθεται περίπτωση γ΄ ως εξής:«γ) γνωρίζει ή μπορεί να αποδείξει με τα κατάλληλαμέσα ότι έχουν επιβληθεί σε βάρος του οικονομικούφορέα, μέσα σε χρονικό διάστημα δύο (2) ετών πριναπό την ημερομηνία λήξης της προθεσμίας υποβολήςπροσφοράς ή αίτησης συμμετοχής: αα) τρεις (3) πράξειςεπιβολής προστίμου από τα αρμόδια ελεγκτικά όργανατου Σώματος Επιθεώρησης Εργασίας για παραβάσεις τηςεργατικής νομοθεσίας που χαρακτηρίζονται, σύμφωναμε την υπουργική απόφαση 2063/Δ1632/2011 (Β΄ 266),όπως εκάστοτε ισχύει, ως «υψηλής» ή «πολύ υψηλής»σοβαρότητας, οι οποίες προκύπτουν αθροιστικά απότρεις (3) διενεργηθέντες ελέγχους, ή ββ) δύο (2) πράξειςεπιβολής προστίμου από τα αρμόδια ελεγκτικά όργανατου Σώματος Επιθεώρησης Εργασίας για παραβάσειςτης εργατικής νομοθεσίας που αφορούν την αδήλωτηεργασία, οι οποίες προκύπτουν αθροιστικά από δύο (2)διενεργηθέντες ελέγχους. Οι υπό αα΄ και ββ΄ κυρώσειςπρέπει να έχουν αποκτήσει τελεσίδικη και δεσμευτικήισχύ. Ο λόγος αποκλεισμού δεν εφαρμόζεται όταν η εκτιμώμενη αξία της σύμβασης, χωρίς ΦΠΑ, είναι ίση ή κατώτερη από το ποσό των είκοσι χιλιάδων (20.000) ευρώ.».2. Στην παρ. 2 του άρθρου 80 του ν. 4412/2016 προστίθεται περίπτωση γ΄ ως εξής:«γ) για την περίπτωση γ΄ της παραγράφου 2 του άρθρου 73, πιστοποιητικό από τη Διεύθυνση Προγραμματισμού και Συντονισμού της Επιθεώρησης ΕργασιακώνΣχέσεων, από το οποίο να προκύπτουν οι πράξεις επιβολής προστίμου που έχουν εκδοθεί σε βάρος του οικονομικού φορέα σε χρονικό διάστημα δύο (2) ετών πριναπό την ημερομηνία λήξης της προθεσμίας υποβολήςπροσφοράς ή αίτησης συμμετοχής.».3. Στο άρθρο 305 του ν. 4412/2016 προστίθεται παράγραφος 3 ως εξής:«3. Ο λόγος αποκλεισμού της περίπτωσης γ΄ της παραγράφου 2 του άρθρου 73 περιλαμβάνεται πάντοτε στουςκανόνες και τα κριτήρια της παραγράφου 1.».4. Στην παρ. 1 του άρθρου 335 του ν. 4412/2016 προστίθεται τελευταίο εδάφιο ως εξής:«Οι αναθέτοντες φορείς επιβάλλουν τον όρο ότι κατάτην εκτέλεση της σύμβασης ο ανάδοχος υποχρεούταινα τηρεί τις διατάξεις της νομοθεσίας περί υγείας καιασφάλειας των εργαζομένων και πρόληψης του επαγγελματικού κινδύνου.».5. Η ισχύς των διατάξεων των παραγράφων 1 έως 4ξεκινά δύο (2) μήνες μετά τη δημοσίευση του παρόντοςνόμου και δεν καταλαμβάνει διαδικασίες που έχουν ήδηξεκινήσει, κατά την έννοια των άρθρων 61, 120, 290 και330 του ν. 4412/2016.6. Η παρ. 2 του άρθρου 24 του ν. 3996/2011 (Α΄ 170)καταργείται και οι υπόλοιπες παράγραφοι του άρθρουαναριθμούνται αντίστοιχα.Β) Η περίπτωση γ΄ της παρ. 2 του άρθρου 68 τουν. 3863/2010 (Α΄ 115) αντικαθίσταται ως εξής:«γ) Η αναθέτουσα αρχή αποκλείει από τη σύναψη τηςσύμβασης τις υποψήφιες εταιρείες παροχής υπηρεσιών καθαρισμού ή/και φύλαξης εάν έχουν επιβληθεί σεβάρος τους, μέσα σε χρονικό διάστημα δύο (2) ετώνπριν από τη λήξη της προθεσμίας υποβολής της προσφοράς, τρεις (3) πράξεις επιβολής προστίμου από τααρμόδια ελεγκτικά όργανα του Σώματος ΕπιθεώρησηςΕργασίας για παραβάσεις της εργατικής νομοθεσίας πουχαρακτηρίζονται, σύμφωνα με την υπουργική απόφαση 2063/Δ1632/2011 (Β΄ 266), όπως εκάστοτε ισχύει,ως «υψηλής» ή «πολύ υψηλής» σοβαρότητας, οι οποίεςπροκύπτουν αθροιστικά, από τρεις (3) διενεργηθέντεςελέγχους ή δύο (2) πράξεις επιβολής προστίμου από τααρμόδια ελεγκτικά όργανα του Σώματος ΕπιθεώρησηςΕργασίας για παραβάσεις της εργατικής νομοθεσίας πουαφορούν την αδήλωτη εργασία, οι οποίες προκύπτουναθροιστικά από δύο (2) διενεργηθέντες ελέγχους. Η αναθέτουσα αρχή μπορεί να αποκλείσει από τη σύναψη τηςσύμβασης τις υποψήφιες εταιρείες παροχής υπηρεσιώνκαθαρισμού ή/και φύλαξης: αα) οι οποίες έχουν κηρυχθείέκπτωτες κατ’ εφαρμογή της παραγράφου 7 του παρόντος μέσα σε χρονικό διάστημα τριών (3) ετών πριν απότην ημερομηνία λήξης της προθεσμίας υποβολής τηςπροσφοράς ή ββ) στις οποίες έχει επιβληθεί η κύρωσητης προσωρινής διακοπής της λειτουργίας συγκεκριμένης παραγωγικής διαδικασίας ή τμήματος ή τμημάτωνή του συνόλου της επιχείρησης ή εκμετάλλευσης κατ’εφαρμογή της παρ. 1Β του άρθρου 24 του ν. 3996/2011(Α΄ 170) μέσα σε χρονικό διάστημα τριών (3) ετών πριναπό την ημερομηνία λήξης της προθεσμίας υποβολήςτης προσφοράς.».
\subsection*{ Άρθρο 40 }
\paragraph { 1. } Οι δυνητικοί δικαιούχοι αποκλείονται από την ένταξη σε πρόγραμμα ή την υπαγωγή σε καθεστώτα ενίσχυσης, που χρηματοδοτούνται από ενωσιακούς ή εθνικούςπόρους, εάν έχουν επιβληθεί σε βάρος τους, μέσα σεχρονικό διάστημα δύο (2) ετών πριν από την ημερομηνίαλήξης της προθεσμίας υποβολής αίτησης συμμετοχής:α) τρεις (3) πράξεις επιβολής προστίμου από τα αρμόδια ελεγκτικά όργανα του Σώματος Επιθεώρησης Εργασίας για παραβάσεις της εργατικής νομοθεσίας πουχαρακτηρίζονται, σύμφωνα με την υπουργική απόφαση2063/Δ1632/2011 (Β΄ 266), όπως εκάστοτε ισχύει, ως«υψηλής» ή «πολύ υψηλής» σοβαρότητας, οι οποίες προκύπτουν αθροιστικά από τρεις (3) διενεργηθέντες ελέγχους, ή β) δύο (2) πράξεις επιβολής προστίμου από τααρμόδια ελεγκτικά όργανα του Σώματος ΕπιθεώρησηςΕργασίας για παραβάσεις της εργατικής νομοθεσίας πουαφορούν την αδήλωτη εργασία, οι οποίες προκύπτουναθροιστικά από δύο (2) διενεργηθέντες ελέγχους.Τεύχος Α’ 137/13.09.2017Οι υπό α΄ και β΄ κυρώσεις πρέπει να έχουν αποκτήσειτελεσίδικη και δεσμευτική ισχύ.
\paragraph { 2. } Ο έλεγχος για την επιβολή των κυρώσεων της παραγράφου 1 πραγματοποιείται μέσω των πληροφοριακώνσυστημάτων των φορέων διαχείρισης και του ΣώματοςΕπιθεώρησης Εργασίας. Μέχρι τη διασύνδεση και διαλειτουργικότητα των πληροφοριακών συστημάτων, η οποίαδιαπιστώνεται με απόφαση των Υπουργών Οικονομίαςκαι Ανάπτυξης και Εργασίας, Κοινωνικής Ασφάλισηςκαι Κοινωνικής Αλληλεγγύης, ο δυνητικός δικαιούχοςκατά την υποβολή της αίτησης συμμετοχής υποβάλλειυπεύθυνη δήλωση, στην οποία δηλώνει ότι δεν έχουνεπιβληθεί σε βάρος τους οι κυρώσεις της παραγράφου 1.
\paragraph { 3. } Οι προσκλήσεις των φορέων διαχείρισης περιέχουνυποχρεωτικά όρους για τη μη ύπαρξη των κυρώσεων τηςπαραγράφου 1, καθώς και για την τήρηση των διατάξεωντης νομοθεσίας περί υγείας και ασφάλειας των εργαζομένων και πρόληψης του επαγγελματικού κινδύνου.
\paragraph { 4. } Με απόφαση των Υπουργών Οικονομίας και Ανάπτυξης και Εργασίας, Κοινωνικής Ασφάλισης και ΚοινωνικήςΑλληλεγγύης καθορίζονται οι κυρώσεις και η διαδικασία επιβολής τους, στην περίπτωση που τα αρμόδιαελεγκτικά όργανα του ΣΕΠΕ επιβάλλουν κυρώσεις γιαπαραβάσεις που χαρακτηρίζονται ως «υψηλής» ή «πολύυψηλής» σοβαρότητας, σύμφωνα με την υπουργική απόφαση 2063/Δ1632/2011 (Β΄ 266), όπως εκάστοτε ισχύει,ή για παραβάσεις που αφορούν την αδήλωτη εργασίαμετά την ένταξη σε πρόγραμμα ή την υπαγωγή σε καθεστώτα ενίσχυσης. Με την ίδια απόφαση καθορίζονταιοι διαδικασίες για την επιβεβαίωση της ορθότητας τωνυπεύθυνων δηλώσεων της παραγράφου 2 και κάθε άλλοσυναφές με την εφαρμογή του παρόντος θέμα.
\paragraph { 5. } Η ισχύς του παρόντος αρχίζει από την έκδοση τωναποφάσεων των παραγράφων 2 και 4.
\subsection*{ Άρθρο 41 }
\paragraph { 0. } Στο άρθρο 16 του ν. 2874/2000 (Α΄ 286) προστίθεταιπαράγραφος 8 ως εξής:«8. Ο εργοδότης υποχρεούται να εφοδιάζει τους εργαζόμενους με αντίγραφο της κατάστασης προσωπικού ήαπόσπασμα αυτής όταν απασχολούνται εκτός της έδραςή του παραρτήματος της επιχείρησης.».
\subsection*{ Άρθρο 42 }
\paragraph { 0. } Στο άρθρο 4 του π.δ. 240/2006 (Α΄ 252) προστίθεταιπαράγραφος 5 ως εξής:«5. Η ενημέρωση που παρέχει ο εργοδότης και τααποτελέσματα των διαβουλεύσεων καταγράφονται σεπρακτικό.».
\subsection*{ Άρθρο 43 }
\paragraph { 0. } Το πρώτο εδάφιο του άρθρου 5 του π.δ. 240/2006αντικαθίσταται ως εξής:«Οι κοινωνικοί εταίροι μπορούν, στο κατάλληλο επίπεδο, συμπεριλαμβανομένου του επιπέδου της επιχείρησηςή της εγκατάστασης, να καθορίζουν ελεύθερα και οποτεδήποτε, μέσω γραπτής συμφωνίας, τις πρακτικές λεπτομέρειες ενημέρωσης και διαβούλευσης των εργαζομένων.».
\subsection*{ Άρθρο 44 }
\paragraph { 1. } Στο τέλος της παρ. 1 του άρθρου 6 του ν. 4097/2012(Α΄ 235) προστίθεται δεύτερο εδάφιο ως εξής:«Το ως άνω επίδομα μπορεί να χορηγείται και στις κυοφόρους αυτοαπασχολούμενες γυναίκες, τις τεκμαιρόμενες αυτοαπασχολούμενες μητέρες του άρθρου 1464 τουΑστικού Κώδικα και στις αυτοαπασχολούμενες γυναίκεςπου υιοθετούν τέκνο ηλικίας έως δύο ετών.»
\paragraph { 2. } Η τεκμαιρόμενη μητέρα του άρθρου 1464 του Αστικού Κώδικα που αποκτά τέκνο με τη διαδικασία της παρένθετης μητρότητας δικαιούται το μεταγενέθλιο τμήματης άδειας μητρότητας που ορίζεται από τις διατάξειςτου άρθρου 7 της ΕΓΣΣΕ 1993 και του άρθρου 7 τηςΕΓΣΣΕ 2000-2001, το οποίο κυρώθηκε με το άρθρο 11του ν. 2874/2000, καθώς και τις πάσης φύσεως αποδοχέςκαι επιδόματα που συνδέονται με αυτήν, εφόσον πληροίτις προϋποθέσεις που ορίζονται στις επιμέρους καταστατικές διατάξεις του φορέα ασφάλισής της.
\paragraph { 3. } Στο άρθρο 142 του ν. 3655/2008 (Α΄ 58) προστίθεταιένατο εδάφιο ως εξής:«Την ειδική παροχή προστασίας μητρότητας, σύμφωνα με τις διατάξεις του παρόντος, δικαιούται και ητεκμαιρόμενη μητέρα του άρθρου 1464 του ΑστικούΚώδικα που αποκτά τέκνο με τη διαδικασία της παρένθετης μητρότητας.».
\subsection*{ Άρθρο 45 }
\paragraph { 0. } Στην παρ. 1 του άρθρου 51 του ν. 4075/2012 (Α΄ 89)προστίθεται δεύτερο εδάφιο ως εξής:«Η ανωτέρω άδεια χορηγείται, κατά την ως άνω διαδικασία, και στο φυσικό, θετό ή ανάδοχο γονέα παιδιούηλικίας έως δεκαοκτώ (18) ετών συμπληρωμένων, τοοποίο πάσχει από βαριά νοητική στέρηση ή σύνδρομοDown ή αυτισμό.».
\subsection*{ Άρθρο 46 }
\paragraph { 0. } Στο άρθρο 15 του ν. 1483/1984 (Α΄ 153) προστίθεταιπαράγραφος 3 ως εξής:«3. Η προστασία έναντι της καταγγελίας της σχέσηςεργασίας, σύμφωνα με τις διατάξεις του παρόντος άρθρου, ισχύει και για τις εργαζόμενες που υιοθετούν τέκνο ηλικίας έως έξι (6) ετών, με χρονική αφετηρία τηντοποθέτηση του τέκνου στην οικογένεια, καθώς και γιατις εργαζόμενες που εμπλέκονται στη διαδικασία τηςπαρένθετης μητρότητας, είτε ως τεκμαιρόμενες μητέρες,με χρονική αφετηρία τη γέννηση του παιδιού, είτε ωςκυοφόροι γυναίκες.».
\subsection*{ Άρθρο 47 }
\paragraph { 0. } Στο άρθρο 621 του Κώδικα Πολιτικής Δικονομίας προστίθεται παράγραφος 3 ως εξής:«3. Η συζήτηση των αγωγών και των τακτικών ένδικωνμέσων επί των διαφορών για άκυρη απόλυση, μισθούςυπερημερίας και καθυστερούμενους μισθούς προσδιορίζεται υποχρεωτικά μέσα σε εξήντα (60) ημέρες από τηνκατάθεσή τους. Αν αναβληθεί η συζήτηση, αυτή προσδιορίζεται υποχρεωτικά μέσα σε τριάντα (30) ημέρες.Η απόφαση του δικαστηρίου δημοσιεύεται υποχρεωτικά μέσα σε τριάντα (30) ημέρες από τη συζήτηση τηςαγωγής ή του τακτικού ένδικου μέσου.».
\subsection*{ Άρθρο 48 }
\paragraph { 0. } προστίθεται νέο άρθρο 636Α ως εξής:«Άρθρο 636Ααντικειμένου της αγωγής, της αίτησης ή άλλου δικογράφου που υποβάλλεται σε οποιοδήποτε δικαστήριοτου Κράτους και υπόκειται σε δικαστικό ένσημο κατά τιςοικείες διατάξεις.».ΜΕΡΟΣ Γ΄ΑΛΛΕΣ ΔΙΑΤΑΞΕΙΣ ΤΟΥ ΥΠΟΥΡΓΕΙΟΥΕΡΓΑΣΙΑΣ, ΚΟΙΝΩΝΙΚΗΣ ΑΣΦΑΛΙΣΗΣ
\subsection*{ Άρθρο 49 }
\paragraph { 1. } Στο δεύτερο εδάφιο της περίπτωσης α΄ της παρ.2 του άρθρου 44 του ν. 3986/2011 (Α΄ 152) η φράση«ασφαλισμένους στον ΟΑΕΕ και το ΕΤΑΠ-ΜΜΕ» αντικαθίσταται με τη φράση «ασφαλισμένους που υπάγονταιστον Ε.Φ.Κ.Α. και για τους οποίους, βάσει γενικών, ειδικών ή καταστατικών διατάξεων που ίσχυαν κατά την31.12.2016, προέκυπτε υποχρέωση υπαγωγής στον τέωςΟΑΕΕ και στο τέως ΕΤΑΠ-ΜΜΕ» και η φράση «ασφαλισμένους του ΕΤΑΑ» αντικαθίσταται με τη φράση «ασφαλισμένους που υπάγονται στον Ε.Φ.Κ.Α. και για τους οποίους, βάσει γενικών, ειδικών ή καταστατικών διατάξεωνπου ίσχυαν κατά την 31.12.2016, προέκυπτε υποχρέωσηυπαγωγής στο τέως ΕΤΑΑ».
\paragraph { 2. } Το τρίτο εδάφιο της περίπτωσης α΄ της παρ. 2 τουάρθρου 44 του ν. 3986/2011 αντικαθίσταται ως εξής:«Σκοπός του λογαριασμού είναι η χορήγηση βοηθήματος σε περιπτώσεις αποδεδειγμένης διακοπής τουεπαγγέλματος και για χρονικό διάστημα τουλάχιστοντριών (3) μηνών, καθώς και η ενίσχυση της απασχόλησης των προσώπων που καλύπτονται από τους ανωτέρωκλάδους.».
\paragraph { 3. } Στο τέταρτο εδάφιο της περίπτωσης α΄ της παρ. 2του άρθρου 44 του ν. 3986/2011 η φράση «ασφαλισμένους του ΕΤΑΑ» αντικαθίσταται με τη φράση «ασφαλισμένους του Ε.Φ.Κ.Α., για τους οποίους, βάσει γενικών,ειδικών ή καταστατικών διατάξεων που ίσχυαν κατά την31.12.2016, προέκυπτε υποχρέωση υπαγωγής στο τέωςΕΤΑΑ», η φράση «το ½ της 1ης ασφαλιστικής κατηγορίας» αντικαθίσταται με τη φράση «το 60% του ποσούπου αντιστοιχεί στο βασικό μισθό άγαμου μισθωτού άνωτων 25 ετών, αναγόμενο σε ετήσια βάση» και η φράση«το τελευταίο 8μηνο» αντικαθίσταται με τη φράση «τοτελευταίο 6μηνο».
\paragraph { 4. } Στο πρώτο εδάφιο της περίπτωσης β΄ της παρ. 2του άρθρου 44 του ν. 3986/2011 η φράση «των Ασφαλιστικών Οργανισμών ΟΑΕΕ, ΕΤΑΑ, ΕΤΑΠ-ΜΜΕ» αντικαθίσταται με τη φράση «που υπάγονται στον Ε.Φ.Κ.Α. καιγια τους οποίους, βάσει γενικών, ειδικών ή καταστατικώνδιατάξεων που ίσχυαν κατά την 31.12.2016, προέκυπτευποχρέωση υπαγωγής στον τέως ΟΑΕΕ, στο τέως ΕΤΑΠΜΜΕ και στο τέως ΕΤΑΑ».
\paragraph { 5. } Στο δεύτερο εδάφιο της περίπτωσης β΄ της παρ. 2του άρθρου 44 του ν. 3986/2011 η φράση «από τα οικείαταμεία» αντικαθίσταται με τη φράση «από τον Ε.Φ.Κ.Α.».
\paragraph { 6. } Στο τέταρτο εδάφιο της περίπτωσης β΄ της παρ. 2του άρθρου 44 του ν. 3986/2011 η φράση «του ΥπουρΤεύχος Α’ 137/13.09.2017γού Εργασίας, Κοινωνικής Ασφάλισης και Πρόνοιας μετάγνώμη των Διοικητικών Συμβουλίων των αρμόδιων οργανισμών και γνώμη του ΣΚΑ» αντικαθίσταται με τη φράση «του Υπουργού Εργασίας, Κοινωνικής Ασφάλισης καιΚοινωνικής Αλληλεγγύης».
\paragraph { 7. } Το πέμπτο εδάφιο της περίπτωσης β΄ της παρ. 2 τουάρθρου 44 του ν. 3986/2011 (Α΄ 152) αντικαθίσταται ωςεξής:«Με απόφαση του Υπουργού Εργασίας, ΚοινωνικήςΑσφάλισης και Κοινωνικής Αλληλεγγύης, μετά από γνώμη του Διοικητικού Συμβουλίου του ΟΑΕΔ, καταρτίζονται προγράμματα για την ενίσχυση της απασχόλησηςτων προσώπων που καλύπτονται από τους κλάδους τηςπερίπτωσης α΄, καθορίζονται και εξειδικεύονται οι προϋποθέσεις και τα κριτήρια για την ενίσχυση, η διαδικασία,ο χρόνος, ο τόπος και ο τρόπος της ενίσχυσης, το ύψοςαυτής, τα απαιτούμενα δικαιολογητικά για την απόδειξητων όρων και προϋποθέσεων χορήγησής της, καθώς καιτα δικαιολογητικά που απαιτούνται για την καταβολήτης, οι όροι, οι προϋποθέσεις και η διαδικασία για τηναναστολή ή τη διακοπή της και κάθε θέμα σχετικό μετην εφαρμογή του παρόντος, εφαρμοζόμενων αναλόγωςτων διατάξεων των παραγράφων 2 και 4 έως 8 του άρθρου 29 του ν. 1262/1982 (Α΄ 70). Με την ίδια απόφασηκαθορίζονται οι διαδικασίες μεταφοράς των διαθέσιμωνπόρων των δύο κλάδων της περίπτωσης α΄, καθώς και τοποσοστό συμμετοχής τους στο πρόγραμμα.».
\subsection*{ Άρθρο 50 }
\paragraph { 1. } Κατά τις διατάξεις των άρθρων 623 έως 636 μπορείνα ζητηθεί η έκδοση διαταγής πληρωμής οφειλόμενουμισθού, εφόσον η σύναψη της σύμβασης εξαρτημένηςεργασίας και το ύψος του μισθού αποδεικνύονται μεδημόσιο ή ιδιωτικό έγγραφο ή με απόφαση ασφαλιστικών μέτρων, η οποία εκδόθηκε μετά από ομολογία ήαποδοχή της αίτησης από τον οφειλέτη, και εφόσον έχειεπιδοθεί έγγραφη όχληση με δικαστικό επιμελητή δεκαπέντε (15) τουλάχιστον ημέρες πριν από την κατάθεσητης αίτησης. Η εργασία που αντιστοιχεί στο μισθό για τονοποίο ζητείται η έκδοση διαταγής πληρωμής τεκμαίρεταιότι έχει παρασχεθεί.
\paragraph { 2. } Η αίτηση για την έκδοση διαταγής πληρωμής υποβάλλεται και στον κατά τόπο αρμόδιο δικαστή της παραγράφου 1 του άρθρου 621.
\paragraph { 3. } Η συζήτηση της ανακοπής κατά διαταγής πληρωμής, που εκδόθηκε για απαιτήσεις της παραγράφου 1,προσδιορίζεται κατά προτεραιότητα στην συντομότερηδιαθέσιμη δικάσιμο και πάντως όχι πέραν των τριών (3)μηνών από την κατάθεσή της. Αν αναβληθεί η συζήτηση,αυτή προσδιορίζεται υποχρεωτικά μέσα σε τριάντα (30)ημέρες.
\paragraph { 4. } Τυχόν ειδικότερες διατάξεις που ισχύουν για τις διαφορές του άρθρου 614 αριθμ. 3 σχετικά με την καταβολήδικαστικού ενσήμου, τέλους απογράφου, παραβόλου καικάθε άλλου τέλους ή φόρου ισχύουν ανάλογα και για τηδιαταγή πληρωμής του παρόντος άρθρου, την ανακοπήπου ασκείται κατ’ αυτής, καθώς και για τα ένδικα μέσακαι βοηθήματα που ασκούνται κατά της απόφασης επίτης ανακοπής.».
\subsection*{ Άρθρο 51 }
\paragraph { 0. } Στο άρθρο 7 του ν. 1545/1985 (Α΄ 91) προστίθεται παράγραφος 6 ως εξής:«6.α) Ο άνεργος, ο οποίος έχει ασκήσει το δικαίωμαπου του παρέχει το άρθρο 7 του ν. 2112/1920 (Α΄ 67)και έχει θεωρήσει τη μονομερή βλαπτική μεταβολή τωνόρων εργασίας του ως καταγγελία της σχέσης εργασίαςαπό τον εργοδότη, επιδοτείται εάν, μαζί με την αίτησηγια επιδότηση, προσκομίσει στον ΟΑΕΔ την εξώδικη δήλωση με την οποία άσκησε το εν λόγω δικαίωμα, καθώςκαι αποδεικτικό της κοινοποίησής της στον εργοδότη,έγγραφο. Στην περίπτωση αυτή, η προθεσμία για τηνυποβολή αίτησης για επιδότηση ξεκινά από την κοινοποίηση της εξώδικης δήλωσης στον εργοδότη.β) Εντός έξι (6) μηνών από την υποβολή της αίτησηςγια επιδότηση, ο άνεργος οφείλει να προσκομίσει στηναρμόδια υπηρεσία του ΟΑΕΔ είτε έγγραφα που αποδεικνύουν την εξόφληση της αποζημίωσης απόλυσηςαπό τον εργοδότη είτε την αγωγή που άσκησε κατά τουεργοδότη με βάση το ασκηθέν δικαίωμα που του παρέχειτο άρθρο 7 του ν. 2112/1920. Σε περίπτωση παράβασηςτης υποχρέωσης του προηγούμενου εδαφίου, καθώς καιεάν η ασκηθείσα αγωγή δεν γίνει τελεσίδικα δεκτή, ηεπιδότηση ανεργίας διακόπτεται και τα ήδη καταβληθέντα ποσά επιστρέφονται αναδρομικά. Εντός προθεσμίαςεξήντα (60) ημερών από την τελεσιδικία της απόφασηςπου θα εκδοθεί επί της αγωγής, τόσο ο άνεργος όσο καιο εναγόμενος εργοδότης, εφόσον έχει ενημερωθεί απότον ΟΑΕΔ για την επιδότηση του ανέργου, υποχρεούνταινα προσκομίσουν στον ΟΑΕΔ την εκδοθείσα απόφαση,ειδάλλως επιβάλλεται πρόστιμο ύψους τριακοσίων (300)ευρώ σε βάρος καθενός από τους υπόχρεους.γ) Αντί να λάβει το επίδομα ανεργίας, κατά τα ανωτέρω, ο άνεργος, με σχετική αίτησή του προς τον ΟΑΕΔ, ηοποία κατατίθεται ταυτόχρονα με την προσκόμιση τηςεξώδικης δήλωσης της περίπτωση α΄, μπορεί να επιλέξεινα λάβει αναδρομικά το σύνολο της επιδότησης ανεργίας που δικαιούται, εφόσον προσκομίσει στον ΟΑΕΔδικαστική απόφαση που κάνει τελεσίδικα δεκτή την αγωγή που άσκησε κατά του εργοδότη με βάση το ασκηθένδικαίωμα που του παρέχει το άρθρο 7 του ν. 2112/1920.».
\subsection*{ Άρθρο 52 }
\paragraph { 0. } σταται ως εξής:«3. Για τη σύσταση, αλλοίωση, μετάθεση ή κατάργηση εμπράγματων δικαιωμάτων επί των πάσης φύσεωςακινήτων του καταργηθέντος ΟΕΚ που περιέρχονται σεαυτόν, ο ΟΑΕΔ απαλλάσσεται από κάθε δημόσιο, δημοτικό, κοινοτικό ή λιμενικό φόρο (άμεσο ή έμμεσο),από κάθε εισφορά και τέλος υπέρ Δημοσίου ή τρίτων καιαπολαύει εν γένει όλων ανεξαιρέτως των δικαστικών, διοικητικών και οικονομικών ατελειών και προνομίων τουΔημοσίου. Διατάξεις της κείμενης νομοθεσίας που εισάγουν ειδικότερες ρυθμίσεις, ατέλειες ή απαλλαγές υπέρτου καταργηθέντος ΟΕΚ για τη διαχείριση των ακινήτωντου ισχύουν και υπέρ του ΟΑΕΔ. Ο ΟΑΕΔ απαλλάσσεταιαπό την καταβολή κάθε είδους τελών για την εγγραφήστο Εθνικό Κτηματολόγιο εμπραγμάτων δικαιωμάτωντου καταργηθέντος ΟΕΚ που περιέρχονται σε αυτόν.».
\subsection*{ Άρθρο 53 }
\paragraph { 0. } Η παρ. 1 του άρθρου 44 του ν. 3863/2010 (Α΄ 115)εφαρμόζεται για την έγκριση των προϋπολογισμώνόλων των φορέων που εποπτεύονται από το Υπουργείο Εργασίας, Κοινωνικής Ασφάλισης και ΚοινωνικήςΑλληλεγγύης.
\subsection*{ Άρθρο 54 }
\paragraph { 0. } Η περίπτωση α΄ της παρ. 3 του άρθρου 30 τουν. 3918/2011 (Α΄ 31) αντικαθίσταται ως εξής:«α. Με απόφαση του Υπουργού Υγείας, ύστερα απόπρόταση των Διοικητών των Υγειονομικών Περιφερειών, συνιστώνται πρωτοβάθμιες και δευτεροβάθμιεςυγειονομικές επιτροπές, για την παραπομπή σε αυτέςθεμάτων υγειονομικής περίθαλψης, πιστοποίησης νόσουπου επιφέρει ανικανότητα για εργασία, χαρακτηρισμούανικανότητας προς εργασία οφειλόμενη σε εργατικόατύχημα ή επαγγελματική ασθένεια για αλλαγή θέσηςεργασίας, καθώς και κρίσης ικανότητας προς εργασία γιατην υπαγωγή στην προαιρετική ασφάλιση, με την οποίαορίζονται η συγκρότηση, οι αρμοδιότητές τους, καθώςκαι κάθε άλλη αναγκαία λεπτομέρεια.».
\subsection*{ Άρθρο 55 }
\paragraph { 1. } Η κυριότητα του επί της οδού Εθνικής Αντιστάσεως1 στον Πειραιά ακινήτου παραμένει στο ΝΑΤ, για τη στέγαση των υπηρεσιών του, και την άσκηση των μη ασφαλιστικών αρμοδιοτήτων του. Ομοίως στην κυριότητα τουΝΑΤ παραμένουν δύο αυτοκίνητα που εξυπηρετούσαντις ανάγκες μετακίνησης στο ΝΑΤ μέχρι 31.12.2016.
\paragraph { 2. } Η ισχύς του παρόντος άρθρου αρχίζει την 1.1.2017.
\subsection*{ Άρθρο 56 }
\paragraph { 0. } Στον Οργανισμό Απασχόλησης Εργατικού Δυναμικού(ΟΑΕΔ) συνιστώνται δέκα (10) οργανικές θέσεις μόνιμουπροσωπικού, κλάδου ΠΕ Πληροφορικής.Προσόντα διορισμού είναι τα προβλεπόμενα στοπ.δ. 50/2001 (Α΄ 39), και επιπλέον απαιτείται κατοχή μεταπτυχιακού τίτλου σπουδών ως εξής:α) Για τις τέσσερις (4) θέσεις, σε ένα από τα επιστημονικά πεδία, Πληροφορικής ή Πληροφορικής και Τηλεπικοινωνιών ή Εφαρμοσμένης Πληροφορικής ή ΕπιστήμηςΥπολογιστών ή Επιστήμης και Τεχνολογίας Υπολογιστώνκαιβ) για τις έξι (6) θέσεις, σε ένα από τα επιστημονικάπεδία, Ηλεκτρολόγου Μηχανικού και Μηχανικού Υπολογιστών ή Ηλεκτρολόγου Μηχανικού και ΤεχνολογίαςΥπολογιστών ή Μηχανικού Η/Υ και Πληροφορικής ήΗλεκτρονικού και Μηχανικού Υπολογιστών ή Μηχανικών Πληροφοριακών και Επικοινωνιακών Συστημάτωνή Μηχανικών Η/Υ Τηλεπικοινωνιών και Δικτύων.Πρόσθετα προσόντα μπορεί κάθε φορά να καθορίζονται με την προκήρυξη πλήρωσης των θέσεων, ύστερααπό γνώμη του Διοικητικού Συμβουλίου του ΟΑΕΔ.
\subsection*{ Άρθρο 57 }
\paragraph { 0. } Η ισχύς των συμβάσεων του επικουρικού προσωπικού των Κέντρων Κοινωνικής Πρόνοιας του άρθρου 9του ν. 4109/2013 (Α΄ 16), του Κέντρου Εκπαίδευσης καιΑποκατάστασης Τυφλών του π.δ. 265/1979 (Α΄ 74) καιτου Εθνικού Ιδρύματος Κωφών του α.ν. 726/1937 (Α΄ 228)παρατείνεται μέχρι τις 31.12.2018, κατά παρέκκλιση κάθεάλλης γενικής και ειδικής διάταξης.
\subsection*{ Άρθρο 58 }
\paragraph { 1. } Η προστασία των υπερηλίκων και χρονίως πασχόντων ατόμων εξασφαλίζεται με την ανάπτυξη προγραμμάτων ανοικτής κοινωνικής φροντίδας, καθώς καικλειστής περίθαλψης που καταρτίζει και εφαρμόζει τοΥπουργείο Εργασίας, Κοινωνικής Ασφάλισης και Κοινωνικής Αλληλεγγύης με σκοπό την βελτίωση της ποιότηταςζωής, την εξασφάλιση αξιοπρεπούς επιπέδου διαβίωσηςκαι περίθαλψης αυτών.
\paragraph { 2. } Τα ανωτέρω προγράμματα δύνανται να υλοποιούνται είτε από τους φορείς του άρθρου 3 του ν. 2646/1998(Α΄ 236) είτε από τις επιχειρήσεις των ΟΤΑ Α΄ βαθμούείτε από ιδιωτικούς φορείς κερδοσκοπικού χαρακτήραπου, σύμφωνα με τις καταστατικές τους διατάξεις έχουνσκοπό την κοινωνική ανάπτυξη και πρόνοια ή συναφήπρος τα ανωτέρω σκοπό.
\paragraph { 3. } Με απόφαση του Υπουργού Εργασίας, ΚοινωνικήςΑσφάλισης και Κοινωνικής Αλληλεγγύης καθορίζεται η μορφή και το περιεχόμενο των καταρτιζόμενωνπρογραμμάτων, οι φορείς υλοποίησης αυτών, οι προϋποθέσεις και τα κριτήρια, η διαδικασία υπαγωγής τωνωφελούμενων στα προβλεπόμενα προγράμματα τηςπροηγούμενης παραγράφου, καθώς και κάθε άλλο αναγκαίο σχετικό θέμα.».ΜΕΡΟΣ Δ΄ΚΑΤΕΥΘΥΝΤΗΡΙΕΣ - ΟΡΓΑΝΩΤΙΚΕΣΔΙΑΤΑΞΕΙΣ ΥΛΟΠΟΙΗΣΗΣ ΤΗΣ ΣΥΜΒΑΣΗΣΤΩΝ ΗΝΩΜΕΝΩΝ ΕΘΝΩΝ ΓΙΑ ΤΑ ΔΙΚΑΙΩΜΑΤΑΤΩΝ ΑΤΟΜΩΝ ΜΕ ΑΝΑΠΗΡΙΕΣΚΕΦΑΛΑΙΟ Α΄
\subsection*{ Άρθρο 59 }
\paragraph { 0. } Αντικείμενο του παρόντος μέρους αποτελεί η θέσπισηενός γενικού πλαισίου ρυθμίσεων κατ’ εφαρμογή διατάξεων της Σύμβασης των Ηνωμένων Εθνών για τα δικαιώματα των Ατόμων με Αναπηρίες και του ΠροαιρετικούΠρωτοκόλλου στη Σύμβαση για τα δικαιώματα των Ατόμων με Αναπηρίες, που κυρώθηκαν με το άρθρο πρώτοτου ν. 4074/2012 (Α΄ 88). Σκοπός του παρόντος μέρουςείναι η άρση των εμποδίων που δυσχεραίνουν την πλήρηκαι ισότιμη συμμετοχή των Ατόμων με Αναπηρίες (ΑμεΑ)στην κοινωνική, οικονομική και πολιτική ζωή της χώρας.
\subsection*{ Άρθρο 60 }
\paragraph { 1. } Ως «Άτομα με Αναπηρίες (ΑμεΑ)» νοούνται τα άτομα με μακροχρόνιες σωματικές, ψυχικές, διανοητικές ήαισθητηριακές δυσχέρειες, οι οποίες σε αλληλεπίδρασημε διάφορα εμπόδια, ιδίως θεσμικά, περιβαλλοντικά ήεμπόδια κοινωνικής συμπεριφοράς, δύναται να παρεμποδίσουν την πλήρη και αποτελεσματική συμμετοχήτων ατόμων αυτών στην κοινωνία σε ίση βάση με τουςάλλους.
\paragraph { 2. } Για τους σκοπούς του παρόντος νόμου:α) ως «Σύμβαση» ορίζεται η Σύμβαση των ΗνωμένωνΕθνών για τα δικαιώματα των ΑμεΑ που υπογράφηκεστη Νέα Υόρκη στις 30 Μαρτίου 2007 και κυρώθηκε μετο άρθρο πρώτο του ν. 4074/2012 (Α΄ 88),β) ως «Πρωτόκολλο» ορίζεται το Προαιρετικό Πρωτόκολλο στη Σύμβαση των Ηνωμένων Εθνών για τα δικαιΤεύχος Α’ 137/13.09.2017ώματα των ΑμεΑ που υπογράφηκε στη Νέα Υόρκη στις27 Σεπτεμβρίου 2010 και κυρώθηκε με το άρθρο πρώτοτου ν. 4074/2012,γ) ως «Επιτροπή» ορίζεται η Επιτροπή για τα δικαιώματα των ΑμεΑ που συστάθηκε σύμφωνα με το άρθρο34 της Σύμβασης.
\paragraph { 3. } Οι διατάξεις των άρθρων 2 και 3 της Σύμβασης διέπουν και το παρόν μέρος.ΚΕΦΑΛΑΙΟ Β΄ΚΑΤΕΥΘΥΝΤΗΡΙΕΣ ΔΙΑΤΑΞΕΙΣ ΥΛΟΠΟΙΗΣΗΣΤΗΣ ΣΥΜΒΑΣΗΣ ΤΩΝ ΗΝΩΜΕΝΩΝ ΕΘΝΩΝΓΙΑ ΤΑ ΔΙΚΑΙΩΜΑΤΑ ΤΩΝ ΑΤΟΜΩΝ
\subsection*{ Άρθρο 61 }
\paragraph { 1. } Κάθε φυσικό πρόσωπο ή νομικό πρόσωπο δημοσίου ή ιδιωτικού δικαίου υποχρεούται να διασφαλίζει τηνισότιμη άσκηση των δικαιωμάτων των ΑμεΑ στο πεδίοτων αρμοδιοτήτων ή δραστηριοτήτων του, λαμβάνονταςκάθε πρόσφορο μέτρο και απέχοντας από οποιαδήποτεενέργεια ή πρακτική που ενδέχεται να θίγει την άσκησητων δικαιωμάτων των ΑμεΑ.Ιδίως υποχρεούται:α) να αφαιρεί υφιστάμενα εμπόδια κάθε είδους,β) να τηρεί τις αρχές καθολικού σχεδιασμού σε κάθετομέα της αρμοδιότητάς του ή της δραστηριοποίησήςτου, προκειμένου να διασφαλίζει για τα ΑμεΑ την προσβασιμότητα των υποδομών, των υπηρεσιών ή των αγαθών που προσφέρει,γ) να παρέχει, όπου απαιτείται σε συγκεκριμένη περίπτωση, εύλογες προσαρμογές υπό τη μορφή εξατομικευμένων και κατάλληλων τροποποιήσεων, ρυθμίσεωνκαι ενδεδειγμένων μέτρων, χωρίς την επιβολή δυσανάλογου ή αδικαιολόγητου βάρους,δ) να απέχει από πρακτικές, κριτήρια, συνήθειες καισυμπεριφορές που συνεπάγονται διακρίσεις σε βάροςτων ΑμεΑ,ε) να προάγει με θετικά μέτρα την ισότιμη συμμετοχήκαι άσκηση των δικαιωμάτων των ΑμεΑ στον τομέα τηςαρμοδιότητας ή δραστηριότητάς του.
\paragraph { 2. } Οι κατά την παράγραφο 1 υπόχρεοι οφείλουν κατάτις συναλλαγές τους με ΑμεΑ να διασφαλίζουν βασική πληροφόρηση μέσω τρόπων, μορφών και μέσωνεπικοινωνίας, όπως ορίζονται στη Σύμβαση. Με κοινήαπόφαση των Υπουργών Εσωτερικών, Οικονομίας καιΑνάπτυξης, Εργασίας, Κοινωνικής Ασφάλισης και Κοινωνικής Αλληλεγγύης, Οικονομικών και Διοικητικής Ανασυγκρότησης έπειτα από συνεργασία με το ΣυντονιστικόΜηχανισμό του άρθρου 69 και κατόπιν διαβούλευσηςμε αναγνωρισμένες αντιπροσωπευτικές οργανώσεις τουαναπηρικού κινήματος, με άτομα και με ομάδες ατόμωνπου έχουν εύλογο ενδιαφέρον για τα δικαιώματα τωνΑμεΑ εξειδικεύονται οι υπόχρεοι φορείς, οι τρόποι, οιμορφές και τα μέσα προσβάσιμης επικοινωνίας, καθώςκαι κάθε άλλο θέμα τεχνικού ή λεπτομερειακού χαρακτήρα για την εφαρμογή των διατάξεων της παρούσαςπαραγράφου.
\subsection*{ Άρθρο 62 }
\paragraph { 1. } Τα διοικητικά όργανα και οι αρχές εντάσσουν τηδιάσταση της αναπηρίας σε κάθε δημόσια πολιτική, διοικητική διαδικασία, δράση, μέτρο και πρόγραμμα τηςαρμοδιότητάς τους με στόχο την εξάλειψη, αποκατάσταση και αποτροπή ανισοτήτων μεταξύ ατόμων με καιχωρίς αναπηρίες. Για το σκοπό αυτόν:α) Υποβάλλουν εκθέσεις στα οικεία Επιμέρους ΣημείαΑναφοράς του άρθρου 71 σχετικά με τις δράσεις, τα μέτρα και τα προγράμματα που υιοθετούν για την επίτευξητης ισότητας των ΑμεΑ,β) υιοθετούν ποσοτικούς και ποιοτικούς δείκτες για ταθέματα αναπηρίας, όπως αυτοί καθορίζονται από αρμόδια διεθνή και ευρωπαϊκά όργανα, ώστε να καθίσταταιδυνατή η μέτρηση και η αξιολόγηση της ένταξης τηςδιάστασης της αναπηρίας,γ) συλλέγουν και τηρούν επιμέρους στατιστικά στοιχεία για την αναπηρία ως προς τους τομείς ευθύνης τους.
\paragraph { 2. } Με κοινή απόφαση των καθ’ ύλην αρμόδιων Υπουργών κατόπιν συνεργασίας με το Συντονιστικό Μηχανισμότου άρθρου 69 και κατόπιν δημόσιας διαβούλευσης εξειδικεύονται οι υπόχρεοι φορείς, οι διαδικασίες υλοποίησης καθώς και κάθε άλλο θέμα τεχνικού ή λεπτομερειακού χαρακτήρα για την εφαρμογή των διατάξεων τουπαρόντος άρθρου.
\subsection*{ Άρθρο 63 }
\paragraph { 1. } Τα διοικητικά όργανα και οι αρχές, σε συνεργασία μετα Επιμέρους Σημεία Αναφοράς του άρθρου 71, υποχρεούνται να τηρούν τις αρχές του καθολικού σχεδιασμούτου άρθρου 2 της Σύμβασης, όπως εξειδικεύονται καιεπικαιροποιούνται κάθε φορά, κατά το σχεδιασμό δημοσίων πολιτικών, διοικητικών υπηρεσιών και προϊόντων,διαδικασιών, περιβαλλόντων και οργανωτικών δομών,που θα μπορούν να χρησιμοποιούνται από όλους στομεγαλύτερο δυνατό βαθμό, χωρίς να απαιτούνται ειδικές προσαρμογές ή εξειδικευμένος σχεδιασμός. Για τηνεπικαιροποίηση προδιαγραφών και οδηγιών εφαρμογής των αρχών του καθολικού σχεδιασμού οι υπόχρεοιτελούν σε διαβούλευση με αναγνωρισμένες αντιπροσωπευτικές οργανώσεις του αναπηρικού κινήματος, μεάτομα και με ομάδες ατόμων που έχουν εύλογο ενδιαφέρον για τα δικαιώματα των ΑμεΑ.
\paragraph { 2. } Τα διοικητικά όργανα και οι αρχές υποχρεούνται ναλαμβάνουν ενδεδειγμένα μέτρα προσαρμοσμένα στιςιδιαίτερες ανάγκες ενός ή περισσότερων ΑμεΑ προκειμένου να διασφαλιστεί η αρχή της ίσης μεταχείρισης. Στιςως άνω εύλογες προσαρμογές, οι οποίες παρέχονται υπότην προϋπόθεση της μη επιβολής δυσανάλογου ή αδικαιολόγητου βάρους, περιλαμβάνονται ενδεικτικά, μέτρα παροχής υποστηρικτικής τεχνολογίας, προσωπικήςβοήθειας και ενδιαμέσων, εξατομικευμένη προσαρμογήδιαδικασιών ή πρακτικών, εξειδικευμένες υπηρεσίες καιβοηθητικές υπηρεσίες για την επικοινωνία.
\subsection*{ Άρθρο 64 }
\paragraph { 1. } Τα διοικητικά όργανα και οι αρχές στο πλαίσιο τηςαρμοδιότητάς τους μεριμνούν για τη διασφάλιση τηςισότιμης πρόσβασης των ΑμεΑ στο φυσικό και δομημένο περιβάλλον, τόσο σε συνήθεις συνθήκες όσο και σεκαταστάσεις έκτακτης ανάγκης.
\paragraph { 2. } Τα διοικητικά όργανα και οι αρχές στο πλαίσιο τηςαρμοδιότητάς τους, διασφαλίζουν την ισότιμη πρόσβαση των ΑμεΑ στο ηλεκτρονικό περιβάλλον, ιδίως στιςηλεκτρονικές επικοινωνίες, πληροφορίες και υπηρεσίες, περιλαμβανομένων των μέσων ενημέρωσης και τωνυπηρεσιών διαδικτύου.
\subsection*{ Άρθρο 65 }
\paragraph { 1. } Τα διοικητικά όργανα και οι αρχές όταν συναλλάσσονται με ΑμεΑ υποχρεούνται να παρέχουν πρόσφοραμέσα επικοινωνίας και πρόσβαση στην πληροφόρηση.Στην ως άνω υποχρέωση περιλαμβάνονται ενδεικτικά,η πρόσβαση στα δημόσια έγγραφα και η κοινοποίησηδιοικητικών πράξεων σε ΑμεΑ σε προσβάσιμες μορφές,καθώς και η διασφάλιση της προηγούμενης ακρόασηςμε πρόσφορους τρόπους, υπό τους όρους και τις προϋποθέσεις του Κώδικα Διοικητικής Διαδικασίας, όπωςκυρώθηκε με το άρθρο πρώτο του ν. 2690/1999 (Α΄ 45).Με απόφαση του Υπουργού Διοικητικής Ανασυγκρότησης, σε συνεργασία με το Συντονιστικό Μηχανισμότου άρθρου 69 και με το Κεντρικό Σημείο Αναφοράςτου άρθρου 70 και κατόπιν δημόσιας διαβούλευσης μετα ενδιαφερόμενα μέρη, προσδιορίζονται τα είδη τωνπρόσφορων μέσων, των προσβάσιμων μορφών και τωντρόπων επικοινωνίας που υιοθετούνται από τις υπηρεσίες της Δημόσιας Διοίκησης, καθώς και κάθε άλλο θέματεχνικού ή λεπτομερειακού χαρακτήρα σχετικό με τηνεφαρμογή της παρούσας παραγράφου.
\paragraph { 2. } Η ελληνική νοηματική γλώσσα αναγνωρίζεται ωςισότιμη με την ελληνική γλώσσα. Το κράτος λαμβάνειμέτρα για την προώθησή της, καθώς και για την κάλυψηόλων των αναγκών επικοινωνίας των κωφών και βαρήκοων πολιτών.
\paragraph { 3. } Η ελληνική γραφή Μπράιγ (Braille) αναγνωρίζεταιως ο τρόπος γραφής των τυφλών Ελλήνων πολιτών. Τοκράτος υποχρεούται να λάβει μέτρα για την προώθησήτης, καθώς και για την κάλυψη των αναγκών επικοινωνίας των ως άνω πολιτών.
\subsection*{ Άρθρο 66 }
\paragraph { 1. } Τα διοικητικά όργανα και οι αρχές ενημερώνουν καιευαισθητοποιούν το προσωπικό, τους συναλλασσόμενους και τους αποδέκτες των υπηρεσιών τους σχετικάμε τα δικαιώματα των ΑμεΑ και τον καθολικό σχεδιασμό. Με κοινή απόφαση των Υπουργών Εσωτερικών,Ψηφιακής Πολιτικής, Τηλεπικοινωνιών και Ενημέρωσηςκαι Διοικητικής Ανασυγκρότησης, ύστερα από δημόσιαδιαβούλευση με αναγνωρισμένες αντιπροσωπευτικέςοργανώσεις του αναπηρικού κινήματος, με άτομα και μεομάδες ατόμων που έχουν εύλογο ενδιαφέρον για τα δικαιώματα των ΑμεΑ, εξειδικεύονται οι τρόποι ενημέρωσης, τα εργαλεία και τα μέσα ευαισθητοποίησης, καθώςκαι το περιεχόμενο των παρεμβάσεων για την υλοποίησητου περιεχομένου της παρούσας παραγράφου.
\paragraph { 2. } Προς το σκοπό της εκπαίδευσης φοιτητών, σπουδαστών αλλά και της κατάρτισης δικαστικών λειτουργώνκαι στελεχών του δημόσιου τομέα σε θέματα δικαιωμάτων και ίσης μεταχείρισης των ΑμεΑ, τα Πανεπιστήμιακαι τα ΤΕΙ, το Εθνικό Κέντρο Δημόσιας Διοίκησης και Αυτοδιοίκησης, η Εθνική Σχολή Δικαστικών Λειτουργώνκαι η Εθνική Σχολή Δημόσιας Υγείας μεριμνούν για τησυμπερίληψη στα προγράμματα σπουδών και στα επιμορφωτικά τους σεμινάρια εκπαιδευτικών ενοτήτων πουαφορούν στα δικαιώματα των ΑμεΑ, όπως αυτά απορρέουν από τη Σύμβαση. Με αποφάσεις των οικείων Υπουργών, σε συνεργασία με το Συντονιστικό Μηχανισμό τουάρθρου 69 και το Κεντρικό Σημείο Αναφοράς του άρθρου 70, ρυθμίζονται οι επιμέρους θεματικές ενότητες,το περιεχόμενο των εκπαιδευτικών ενοτήτων, καθώς καικάθε άλλο θέμα τεχνικού ή λεπτομερειακού χαρακτήρα.
\subsection*{ Άρθρο 67 }
\paragraph { 1. } Τα δημόσια και ιδιωτικά ΜΜΕ, έντυπα και ηλεκτρονικά, προωθούν την εμπέδωση και το σεβασμό της αρχήςτης μη διάκρισης. Για τον σκοπό αυτό, το Εθνικό Συμβούλιο Ραδιοτηλεόρασης (ΕΣΡ) στους Κώδικες Δεοντολογίας Ειδησεογραφικών Εκπομπών, Διαφημίσεων καιΨυχαγωγικών Προγραμμάτων που καταρτίζει, οφείλεινα συμπεριλαμβάνει ρυθμίσεις που αποβλέπουν στηνπραγμάτωση της αρχής της μη διάκρισης λόγω αναπηρίας, στην ανάπτυξη ενός πλουραλιστικού διαλόγου γιατα θέματα των ΑμεΑ και την προαγωγή της ουσιαστικήςισότητας μεταξύ ατόμων με και χωρίς αναπηρίες.
\paragraph { 2. } Οι πάροχοι υπηρεσιών μέσων ενημέρωσης και επικοινωνίας, συμπεριλαμβανομένου και του διαδικτύου,υποχρεούνται να αξιοποιούν τις νέες τεχνολογίες, όπωςομιλούσες ιστοσελίδες, υποτιτλισμό, ακουστική περιγραφή, διερμηνεία νοηματικής, προκειμένου να διασφαλίσουν την πρόσβαση των ΑμεΑ σε αυτά. Με απόφασητου Υπουργού Ψηφιακής Πολιτικής, Τηλεπικοινωνιών καιΕνημέρωσης καθορίζονται τα μέσα, η διαδικασία καθώςκαι κάθε άλλο θέμα τεχνικού ή λεπτομερειακού χαρακτήρα για την εφαρμογή των διατάξεων της παρούσαςπαραγράφου.
\subsection*{ Άρθρο 68 }
\paragraph { 1. } Κατά το στάδιο της νομοπαραγωγικής διαδικασίας τα αρμόδια όργανα συνεκτιμούν τα δικαιώματα τωνΑμεΑ, όπως αυτά περιγράφονται στη Σύμβαση και κατάτη διάρκεια της κατάρτισης σχεδίων νόμου, συνεργάζονται με το Συντονιστικό Μηχανισμό του άρθρου 69 και μεΤεύχος Α’ 137/13.09.2017το Κεντρικό Σημείο Αναφοράς του άρθρου 70 και τελούνσε διαβούλευση με αναγνωρισμένες αντιπροσωπευτικέςοργανώσεις του αναπηρικού κινήματος, με άτομα και μεομάδες ατόμων που έχουν εύλογο ενδιαφέρον για ταδικαιώματα των ΑμεΑ.
\paragraph { 2. } Στην ανάλυση συνεπειών ρυθμίσεων που συνοδεύεικάθε σχέδιο νόμου, προσθήκη ή τροπολογία, καθώς καικανονιστικές αποφάσεις μείζονος οικονομικής ή κοινωνικής σημασίας, συμπεριλαμβάνεται ειδική ενότητα τεκμηρίωσης της συμβατότητας των προτεινόμενων ρυθμίσεων με τη Σύμβαση, καθώς και των ειδικών συνεπειώντων προτεινόμενων ρυθμίσεων στα ΑμεΑ.
\paragraph { 3. } Η Ελληνική Στατιστική Αρχή και οι υπηρεσίες καιφορείς που περιλαμβάνονται στο Ελληνικό ΣτατιστικόΣύστημα αναπτύσσουν, παράγουν και διαδίδουν σύμφωνα με τις αρχές του Ελληνικού Στατιστικού Συστήματος και την κείμενη νομοθεσία επίσημες στατιστικέςσχετικά με τα ΑμεΑ, καθώς και με τα εμπόδια που αυτάαντιμετωπίζουν κατά την άσκηση των δικαιωμάτων τους.Για τους σκοπούς του σχεδιασμού των ως άνω στατιστικών και της διάχυσης των παραγόμενων δεδομένωντελούν σε διαβούλευση με το Παρατηρητήριο για ταΘέματα Αναπηρίας της Εθνικής Συνομοσπονδίας Ατόμων με Αναπηρία (Ε.Σ.Α.με Α.).ΚΕΦΑΛΑΙΟ Γ΄
\subsection*{ Άρθρο 69 }
\paragraph { 1. } Ο Υπουργός Επικρατείας, αρμόδιος για τη συνοχήτου κυβερνητικού έργου, ορίζεται ως Συντονιστικός Μηχανισμός σύμφωνα με την παράγραφο 1 του άρθρου33 της Σύμβασης. Στην αρμοδιότητά του ανήκουν ειδικότερα:α) Η παρακολούθηση θεμάτων που αφορούν στα δικαιώματα των ΑμεΑ,β) ο συντονισμός των αρμόδιων Υπουργείων για τηχάραξη και την εφαρμογή δημόσιων πολιτικών που προάγουν τα δικαιώματα των ΑμεΑ,γ) η συνεργασία με το Κεντρικό Σημείο Αναφοράς τουάρθρου 70,δ) ο συντονισμός και η παρακολούθηση του έργουτων συναρμόδιων Υπουργείων για την εφαρμογή τηςΣύμβασης στον ιδιωτικό τομέα,ε) οι αρμοδιότητες που καθορίζονται από τη Σύμβασηκαι το Πρωτόκολλο σε σχέση με την παρακολούθησηεφαρμογής της Σύμβασης.
\paragraph { 2. } Με απόφαση του Πρωθυπουργού και εφόσον δενυπάρχει Υπουργός Επικρατείας με αρμοδιότητα τη συνοχή του κυβερνητικού έργου, μπορεί η αρμοδιότητατου παρόντος άρθρου να ανατίθεται σε άλλον Υπουργό.
\paragraph { 3. } Με απόφαση του Υπουργικού Συμβουλίου, ύστερααπό πρόταση του Υπουργού Επικρατείας της παραγράφου 1, μπορεί να συνιστώνται στο γραφείο του ως άνωΥπουργού μέχρι δύο (2) θέσεις μετακλητών διοικητικώνυπαλλήλων ή ειδικών συνεργατών εξειδικευμένων σεθέματα αναπηρίας.
\subsection*{ Άρθρο 70 }
\paragraph { 1. } Η Γενική Γραμματεία Διαφάνειας και ΑνθρωπίνωνΔικαιωμάτων του Υπουργείου Δικαιοσύνης, Διαφάνειαςκαι Ανθρωπίνων Δικαιωμάτων ορίζεται ως Κεντρικό Σημείο Αναφοράς για θέματα σχετιζόμενα με την εφαρμογήτης Σύμβασης.
\paragraph { 2. } Στην αρμοδιότητα του Κεντρικού Σημείου Αναφοράς ανήκει:α) Η υποδοχή και ο χειρισμός ζητημάτων που άπτονταιτης εφαρμογής της Σύμβασης σε κεντρικό, περιφερειακόκαι τοπικό επίπεδο,β) η συνεργασία με τα Επιμέρους Σημεία Αναφοράςτου άρθρου 71,βασης του άρθρου 72,γ) η διασύνδεση με το Πλαίσιο Προαγωγής της Σύμδ) η διαβούλευση με αναγνωρισμένες αντιπροσωπευτικές οργανώσεις του αναπηρικού κινήματος, με άτομακαι με ομάδες ατόμων, οργανωμένες ή μη που έχουνεύλογο ενδιαφέρον για τα δικαιώματα των ΑμεΑ,ε) η παροχή ενημέρωσης και κατευθύνσεων για θέματα σχετικά με τα δικαιώματα των ΑμεΑ,στ) η εκπόνηση και υποβολή στη Βουλή ΕθνικούΣχεδίου Δράσης για τα ΑμεΑ, καθώς και η σύνταξη καιυποβολή στην Επιτροπή για τα δικαιώματα των ΑμεΑτου άρθρου 34 της Σύμβασης των εκθέσεων που προβλέπονται στο άρθρο 35 αυτής.
\subsection*{ Άρθρο 71 }
\paragraph { 1. } Με απόφαση των οικείων Υπουργών ορίζεται σεκάθε Υπουργείο ο Γενικός ή Διοικητικός Γραμματέας ωςΣημείο Αναφοράς, για την παρακολούθηση υλοποίησης των υιοθετούμενων κατ’ εφαρμογή της Σύμβασηςδημόσιων πολιτικών στα οικεία Υπουργεία και στουςεποπτευόμενους από αυτά φορείς, την προώθηση στοΚεντρικό Σημείο Αναφοράς του άρθρου 70, προτάσεωνκαι νομοθετικών ρυθμίσεων βέλτιστης εφαρμογής τηςΣύμβασης, τη σύνταξη ετήσιων εκθέσεων προόδου καιτην προώθηση δημόσιας διαβούλευσης για τα ως άνωθέματα.
\paragraph { 2. } Με απόφαση του Γενικού ή Διοικητικού Γραμματέα οι αρμοδιότητες του Σημείου Αναφοράς μπορεί ναμεταβιβασθούν σε οργανική μονάδα επιπέδου ΓενικήςΔιεύθυνσης, Διεύθυνσης ή Τμήματος.
\paragraph { 3. } Στην έδρα κάθε Περιφέρειας και κάθε Δήμου ορίζεται ως Σημείο Αναφοράς ο Περιφερειάρχης και ο Δήμαρχος αντίστοιχα για την παρακολούθηση εφαρμογήςτης Σύμβασης σε περιφερειακό και τοπικό επίπεδο. Μεαπόφαση του Περιφερειάρχη και του Δημάρχου αντίστοιχα οι αρμοδιότητες του Σημείου Αναφοράς μπορείνα μεταβιβασθούν σε οργανική μονάδα επιπέδου Γενικής Διεύθυνσης, Διεύθυνσης ή Τμήματος.
\subsection*{ Άρθρο 72 }
\paragraph { 1. } Για την προώθηση της εφαρμογής της Σύμβασης,κατά τα διαλαμβανόμενα στην παράγραφο 2 του άρθρου 33 αυτής, ορίζεται ο Συνήγορος του Πολίτη ως ησυνταγματικά κατοχυρωμένη Ανεξάρτητη Αρχή πουαποτελεί το Πλαίσιο για την Προαγωγή της εφαρμογήςτης Σύμβασης (εφεξής Πλαίσιο Προαγωγής). Για την εκπλήρωση της αποστολής του το Πλαίσιο Προαγωγής τελεί σε συνεργασία με την Εθνική Συνομοσπονδία Ατόμωνμε Αναπηρία (Ε.Σ.Α.με Α.), η οποία αποτελεί τριτοβάθμιαοργάνωση των Ατόμων με Αναπηρίες και ανεξάρτητομηχανισμό της κοινωνίας των πολιτών.
\paragraph { 2. } Αποστολή του Πλαισίου Προαγωγής είναι η παρακολούθηση, προαγωγή και προστασία της εφαρμογής τηςΣύμβασης και των δημόσιων πολιτικών, για την προώθηση των δικαιωμάτων των ΑμεΑ. Το Πλαίσιο Προαγωγήςμεταξύ άλλων:α) Εκφράζει γνώμη για τη συμβατότητα των δημοσίωνπολιτικών που προωθούνται σε κεντρικό, περιφερειακόή τοπικό επίπεδο, καθώς και για τη συμβατότητα τηςισχύουσας εθνικής νομοθεσίας με τις διατάξεις της Σύμβασης,β) χειρίζεται και διερευνά αναφορές που υποβάλλονται ενώπιόν του σε σχέση με την παραβίαση δικαιωμάτων των ΑμεΑ,γ) αναλαμβάνει δράσεις ευαισθητοποίησης σε θέματασχετικά με τα δικαιώματα και τις υποχρεώσεις που απορρέουν από τη Σύμβαση,δ) εκπονεί μελέτες και έρευνες, σχετικά με την υλοποίηση άρθρων της Σύμβασης σε επιμέρους τομείς,ε) υποβάλλει ετήσια έκθεση με την αξιολόγηση τωνδημόσιων πολιτικών, της εφαρμογής της κείμενης νομοθεσίας και προτεινόμενα μέτρα αντιμετώπισης ελλείψεων και αναγκών που διαπιστώθηκαν, η οποία μπορείνα περιλαμβάνει προτάσεις νομοθετικών ρυθμίσεων ήτροποποιήσεων,στ) λαμβάνει κάθε άλλο πρόσφορο για την εξυπηρέτηση της αποστολής του μέτρο.ΚΕΦΑΛΑΙΟ Δ΄
\subsection*{ Άρθρο 73 }
\paragraph { 0. } Όπου προβλέπεται κατ’ εξουσιοδότηση του παρόντοςμέρους η έκδοση κανονιστικών ή κοινών κανονιστικώνπράξεων, αυτή λαμβάνει χώρα το αργότερο εντός τεσσάρων (4) μηνών από τη θέση αυτού σε ισχύ.
\subsection*{ Άρθρο 74 }
\paragraph { 1. } Οι διατάξεις του παρόντος μέρους δεν θίγουν ευνοϊκότερες διατάξεις του εθνικού, κοινοτικού ή διεθνούςδικαίου σχετικές με την υλοποίηση δικαιωμάτων τωνΑμεΑ και δεν αποτελούν λόγο περιορισμού του υφιστάμενου επιπέδου παρεχόμενης προστασίας.
\paragraph { 2. } Το άρθρο δεύτερο του ν. 4074/2012 αντικαθίσταταιως εξής:«Οι διατάξεις της παραγράφου 1 του άρθρου 27 τηςΣύμβασης των Ηνωμένων Εθνών για τα δικαιώματα τωνΑμεΑ δεν εφαρμόζονται ως προς την εργασία και τηναπασχόληση στις ένοπλες δυνάμεις καθόσον αφορά σεδιαφορετική μεταχείριση λόγω αναπηρίας, ή χρόνιαςπάθησης σχετικής με την Υπηρεσία, όπως προβλέπεταιστη διάταξη της παρ. 5 του άρθρου 3 του ν. 4443/2016(Α΄ 232)».
\paragraph { 3. } Το άρθρο τρίτο του ν. 4074/2012 καταργείται.
\paragraph { 4. } Η έκδοση του προβλεπόμενου στο άρθρο 24 τουν. 4443/2016 (Α΄ 232) προεδρικού διατάγματος για τηνεπέκταση της προστασίας που παρέχεται για διακρίσειςμεταξύ άλλων λόγω αναπηρίας ή χρόνιας πάθησης καιστους τομείς της κοινωνικής προστασίας, συμπεριλαμβανομένης της κοινωνικής ασφάλισης και της υγειονομικήςπερίθαλψης, των κοινωνικών παροχών και φορολογικώνδιευκολύνσεων, της εκπαίδευσης και της πρόσβασης στηδιάθεση και παροχή αγαθών και υπηρεσιών που διατίθενται συναλλακτικά στο κοινό, συμπεριλαμβανομένηςκαι της στέγης, λαμβάνει χώρα σε συνεργασία με το Συντονιστικό Μηχανισμό του άρθρου 69 και ύστερα απόδιαβούλευση με αναγνωρισμένες αντιπροσωπευτικέςοργανώσεις του αναπηρικού κινήματος, με άτομα και μεομάδες ατόμων, οργανωμένες ή μη που έχουν εύλογοενδιαφέρον για τα δικαιώματα των ΑμεΑ, το αργότεροεντός δώδεκα (12) μηνών από την έναρξη ισχύος τουπαρόντος νόμου.
\paragraph { 5. } Από την έναρξη ισχύος του παρόντος καταργείταικάθε γενική ή ειδική διάταξη που είναι αντίθετη ή ρυθμίζει διαφορετικά τα προβλεπόμενα με τις διατάξεις τουπαρόντος θέματα.ΜΕΡΟΣ Ε΄ΟΡΓΑΝΩΣΗ ΚΑΙ ΛΕΙΤΟΥΡΓΙΑ ΝΟΜΙΚΟΥΠΡΟΣΩΠΟΥ ΔΗΜΟΣΙΟΥ ΔΙΚΑΙΟΥΜΕ ΤΗΝ ΕΠΩΝΥΜΙΑ «ΣΥΝΔΕΣΜΟΣ
\subsection*{ Άρθρο 75 }
\paragraph { 1. } Το Ν.Π.Δ.Δ. με την επωνυμία «Σύνδεσμος Κοινωνικών Λειτουργών Ελλάδος» (ΣΚΛΕ) εδρεύει στην Αθήνακαι τελεί υπό την εποπτεία του Υπουργείου Εργασίας,Κοινωνικής Ασφάλισης και Κοινωνικής Αλληλεγγύης.
\paragraph { 2. } Περιφερειακά Τμήματα (ΠΤ) του ΣΚΛΕ δύνανταινα ιδρύονται σε κάθε Περιφέρεια του άρθρου 3 τουν. 3852/2010 (Α΄ 87). Ως έδρα των ΠΤ ορίζεται η πρωτεύουσα του Νομού που ανήκει στην οικεία Περιφέρεια καιέχει τα περισσότερα μέλη. Ειδικώς το ΠΤ Νοτίου Αιγαίουθα έχει έδρα την Ρόδο.
\subsection*{ Άρθρο 76 }
\paragraph { 1. } Κατά την πρώτη εφαρμογή του παρόντος τα Περιφερειακά Τμήματα (ΠΤ) ιδρύονται με απόφαση της Προσωρινής Διοικούσας Επιτροπής. Μετά την ανάδειξη τουπρώτου Διοικητικού Συμβουλίου (ΔΣ) η ίδρυση, κατάργηση και συγχώνευση των ΠΤ ενεργείται με απόφασητης Γενικής Συνέλευσης (ΓΣ). Με την απόφαση ίδρυσης,κατάργησης και συγχώνευσης των ΠΤ, προσδιορίζεται ηέδρα και η τοπική αρμοδιότητα εκάστου ΠΤ, με γνώμονατον Νομό όπου ασκεί κάθε μέλος του ΣΚΛΕ το επάγγελμα.Τεύχος Α’ 137/13.09.2017
\paragraph { 2. } Με την απόφαση ίδρυσης νέου ΠΤ ορίζεται η αντίστοιχη Προσωρινή Διοικούσα Επιτροπή (ΠΔΕ). Η ΠΔΕείναι τριμελής, και υποχρεούται εντός τριών (3) μηνώναπό τον ορισμό της να εγγράψει στο μητρώο του οικείουΠΤ τους κοινωνικούς λειτουργούς που υπάγονται στηντοπική του αρμοδιότητα. Εντός δεκαπέντε (15) ημερώναπό την επομένη της εκπνοής της τρίμηνης προθεσμίαςγια την εγγραφή των μελών, υποχρεούται να συγκαλέσειΓΣ, για την εκλογή της αιρετής διοίκησης του Περιφερειακού Τμήματος. Η θητεία ειδικώς της πρώτης αιρετήςδιοίκησης νεοϊδρυθέντος ΠΤ, λήγει με την πάροδο τηςτριετούς θητείας του τρέχοντος ΔΣ.
\paragraph { 3. } Για τη λειτουργία τους τα ΠΤ επιχορηγούνται απότο ΣΚΛΕ ετησίως. Για την καταβολή της ετήσιας επιχορήγησής τους ή τα ΠΤ αποστέλλουν στο ΣΚΛΕ κατά τουςπρώτους δύο (2) μήνες κάθε έτος: α) προτεινόμενο Προϋπολογισμό και β) Ετήσιο Σχέδιο Δράσης. Το ύψος τηςεπιχορήγησης καθορίζεται με απόφαση του ΔΣ.
\paragraph { 4. } Για τη αντιμετώπιση εκτάκτων γεγονότων, ή τηνυλοποίηση ειδικών δράσεων με απόφαση του ΔΣ είναιδυνατή και η έκτακτη επιχορήγησή τους.
\subsection*{ Άρθρο 77 }
\paragraph { 1. } Τα μέλη του Συνδέσμου διακρίνονται σε τακτικά καιεπίτιμα, πάρεδρα και έκτακτα.
\paragraph { 2. } Τακτικά μέλη του ΣΚΛΕ είναι υποχρεωτικά όλοι οικοινωνικοί λειτουργοί που είναι απόφοιτοι: α) ΤμημάτωνΚοινωνικής Εργασίας πανεπιστημιακής και τεχνολογικήςεκπαίδευσης της ημεδαπής ή του Τμήματος ΚοινωνικήςΔιοίκησης του Δημοκρίτειου Πανεπιστημίου της Θράκηςμε κατεύθυνση Κοινωνικής Εργασίας και β) Σχολών ήτμημάτων της αλλοδαπής, των οποίων τα διπλώματαέχουν αναγνωριστεί ως ισότιμα με τα διπλώματα τωνημεδαπών σχολών και τμημάτων.
\paragraph { 3. } Κοινωνικοί λειτουργοί που δικαιούνται σύμφωναμε τις διατάξεις της Ε.Ε. να ασκούν το επάγγελμα στηνΕλλάδα, υποχρεούνται να γίνουν μέλη του ΣΚΛΕ με ταίδια δικαιώματα και υποχρεώσεις. Το Υπουργείο Εργασίας, Κοινωνικής Ασφάλισης και Κοινωνικής Αλληλεγγύηςδύναται να αναθέτει στα μέλη του ΣΚΛΕ την υλοποίησησυγκεκριμένου έργου, σχετικού με τις αρμοδιότητες τωνΚοινωνικών Λειτουργών, μέσω προγραμματικών συμβάσεων, σύμφωνα με την κείμενη νομοθεσία.
\paragraph { 4. } Επίτιμα μέλη του ΣΚΛΕ είναι πρόσωπα που έχουνσυμβάλει στην προαγωγή και την ανάπτυξη του επαγγέλματος του κοινωνικού λειτουργού. Τα επίτιμα μέληορίζονται μετά από απόφαση του ΔΣ που λαμβάνεταιμε πλειοψηφία των δύο τρίτων (2/3) των μελών.
\paragraph { 5. } Τα τακτικά μέλη, μετά την συνταξιοδότησή τους, μεταπίπτουν σε πάρεδρα μέλη χωρίς δικαίωμα εκλέγειν καιεκλέγεσθαι και χωρίς υποχρέωση καταβολής εισφοράς.Τα πάρεδρα μέλη δύνανται να μετέχουν σε κάθε μορφήςεπιστημονικές, κοινωνικές και πάσης άλλης φύσεως δράσεις του Συνδέσμου.
\paragraph { 6. } Έκτακτα μέλη, χωρίς δικαίωμα εκλέγειν και εκλέγεσθαι και χωρίς υποχρέωση καταβολής εισφοράς, δύνανται να γίνονται οι σπουδαστές των αναγνωρισμένωνσχολών Κοινωνικής Εργασίας, με το πέρας του δεύτερουέτους σπουδών τους. Τα έκτακτα μέλη δύνανται να παρευρίσκονται στις ΓΣ, και να μετέχουν σε επιστημονικέςή κοινωνικές δράσεις του Συνδέσμου.
\paragraph { 7. } Παύουν να είναι μέλη του ΣΚΛΕ όσοι τους ανακλήθηκε, ακυρώθηκε, καταργήθηκε η άδεια άσκησης επαγγέλματος, σύμφωνα με τις διατάξεις του π.δ. 23/1992 (Α΄6),καθώς και όσοι τους επιβάλλεται η πειθαρχική ποινή τηςπροσωρινής διαγραφής, για όσο χρονικό διάστημα διαρκεί αυτή, και της οριστικής διαγραφής.
\subsection*{ Άρθρο 78 }
\paragraph { 1. } Κάθε κοινωνικός λειτουργός υποχρεούται να υποβάλλει αίτηση εγγραφής - δήλωση στον ΣΚΛΕ με ταεξής στοιχεία: ονοματεπώνυμο, αριθμό αστυνομικήςταυτότητας, όνομα πατέρα, όνομα μητέρας, ΑΦΜ, διεύθυνση κατοικίας και εργασίας, εργασιακή κατάσταση,φορέα απασχόλησης, στοιχεία απασχόλησης, στοιχείαεκπαίδευσης εγγραφής στο ΣΚΛΕ. Η αίτηση εγγραφήςσυνοδεύεται από δύο (2) φωτογραφίες του αιτούντος καιεπικυρωμένο φωτοαντίγραφο του τίτλου σπουδών του.Για την εγγραφή απαιτείται η εφάπαξ καταβολή ποσούείκοσι πέντε (25,00) ευρώ. Για κάθε εγγραφή τηρείται στοΣΚΛΕ ατομικός ηλεκτρονικός φάκελος.
\paragraph { 2. } Κάθε κοινωνικός λειτουργός υποχρεούται, έως τοτέλος Φεβρουαρίου κάθε χρόνου, να υποβάλει στο ΣΚΛΕ,δήλωση με τα εξής στοιχεία: όνομα, επώνυμο, όνομαπατέρα, όνομα μητέρας, τόπο γέννησης, ιθαγένεια, διεύθυνση κατοικίας και εργασίας. Μαζί με τη δήλωσηυποχρεούται να υποβάλει αντίγραφο της Άδειας Άσκησης Επαγγέλματος Κοινωνικού Λειτουργού, και δήλωσηπραγματικής άσκησης του επαγγέλματος. Το έντυπο τηςδήλωσης αναρτάται στην ιστοσελίδα του Συνδέσμου. ΤοΔΣ του Συνδέσμου μπορεί με απόφασή του να τροποποιήσει τη μορφή της δήλωσης. Η δήλωση καταχωρείται στο ηλεκτρονικό μητρώο του ΣΚΛΕ και τον ατομικόφάκελο του κοινωνικού λειτουργού. Η ετήσια εισφοράκάθε κοινωνικού λειτουργού στον ΣΚΛΕ ορίζεται στα είκοσι πέντε (25) ευρώ, και καταβάλλεται στο ΔΣ του ΣΚΛΕμέχρι το τέλος Φεβρουαρίου εκάστου έτους. Τα επίτιμαμέλη δεν υποχρεούνται να καταβάλλουν τα ποσά τηςετήσιας εισφοράς και της εγγραφής. Άνεργοι κοινωνικοίλειτουργοί, θεωρείται ότι είναι οικονομικά τακτοποιημένοι με την προσκόμιση Βεβαίωσης Ανεργίας του ΟΑΕΔ.
\paragraph { 3. } Τα ποσά της ετήσιας εισφοράς και της εγγραφής,μπορούν να αναπροσαρμόζονται με απόφαση τηςΓΣ του ΣΚΛΕ, που λαμβάνεται με πλειοψηφία των δύοτρίτων (2/3) των μελών του. Εντός μηνός, από την εγγραφή μελών στα ΠΤ, τα Περιφερειακά Συμβούλια τωνΠΤ, υποχρεούνται να αποστείλουν τις αιτήσεις εγγραφής,ετήσιες δηλώσεις και τα ποσά που έχουν καταβληθείγια την εγγραφή και την ετήσια εισφορά, με σκοπό τηντήρηση Ενιαίου Μητρώου Κοινωνικών Λειτουργών.Σε κάθε κοινωνικό λειτουργό που υποβάλλει εμπρόθεσμα τη δήλωση και καταβάλλει την ετήσια συνδρομή,χορηγείται δελτίο ταυτότητας, που ισχύει μέχρι το τέλοςΦεβρουαρίου του επόμενου έτους. Το δελτίο υπογράφεται από τον Πρόεδρο και τον Γενικό Γραμματέα του ΣΚΛΕ.Η υποβολή εκπρόθεσμης ή ανειλικρινούς δήλωσηςαποτελεί πειθαρχικό παράπτωμα.
\paragraph { 4. } Κάθε κοινωνικός λειτουργός που απασχολείται ωςελεύθερος επαγγελματίας αναρτά στην είσοδο του καταστήματος βεβαίωση με την ένδειξη «Νόμιμος ΧώροςΠαροχής Κοινωνικής Εργασίας», που χορηγείται από τοΣΚΛΕ και αναφέρει τον αριθμό μητρώου του μέλους.Η βεβαίωση ισχύει μέχρι το τέλος του μηνός Φεβρουαρίου του επόμενου έτους, οπότε και ανανεώνεται με τηνκαταβολή της ετήσιας εισφοράς.
\subsection*{ Άρθρο 79 }
\paragraph { 1. } Πόροι του Συνδέσμου είναι:α) Τα έσοδα από την εγγραφή νέων μελών.β) Η ετήσια εισφορά των τακτικών μελών.γ) Τα δικαιώματα από την έκδοση πιστοποιητικών καιβεβαιώσεων.δ) Τυχόν έκτακτες εισφορές των τακτικών μελών, μετάαπό απόφαση του ΔΣ, που λαμβάνεται με πλειοψηφίαδύο τρίτων (2/3) των μελών του και επικυρώνεται απότην Συνέλευση των Αντιπροσώπων.ε) Επιχορηγήσεις, από φυσικά ή νομικά πρόσωπα.στ) Δωρεές ή κληροδοτήματα.ζ) Έσοδα από την υλοποίηση προγραμμάτων, με εθνική ή κοινοτική χρηματοδότηση.η) Έσοδα από την πραγματοποίηση εκδηλώσεων.θ) Έσοδα από τις διαφημιστικές καταχωρήσεις στοπεριοδικό, τα ενημερωτικά έντυπα και την ιστοσελίδατου Συνδέσμου.ι) Κάθε ποσό που εισπράττεται από το Σύνδεσμο, γιανόμιμη αιτία και προκειμένου να προαχθούν οι καταστατικοί σκοποί του.
\paragraph { 2. } Η οικονομική διαχείριση του Συνδέσμου διέπεταιαπό τις διατάξεις του ν.δ. 496/1974 (Α΄ 204).
\subsection*{ Άρθρο 80 }
\paragraph { 1. } Ο ΣΚΛΕ διαρθρώνεται σε Κεντρική Διοίκηση και Περιφερειακά Τμήματα (ΠΤ).
\paragraph { 2. } Τα όργανα διοίκησης του ΣΚΛΕ είναι η Γενική Συνέλευση (ΓΣ) και το Διοικητικό Συμβούλιο (ΔΣ). Στο Σύνδεσμο λειτουργεί Πρωτοβάθμιο και ΔευτεροβάθμιοΠειθαρχικό Συμβούλιο, Κεντρική Εξελεγκτική Επιτροπήκαι Συνέλευση Προέδρων.
\paragraph { 3. } Όργανα διοίκησης κάθε ΠΤ είναι η ΠεριφερειακήΣυνέλευση και το Περιφερειακό Συμβούλιο.
\paragraph { 4. } Η θητεία των μελών του ΔΣ, και των ΠεριφερειακώνΣυμβουλίων, είναι τριετής.
\subsection*{ Άρθρο 81 }
\paragraph { 1. } Η ΓΣ αποτελείται από το σύνολο των οικονομικώςτακτοποιημένων μελών του ΣΚΛΕ.
\paragraph { 2. } Η ΓΣ είναι το ανώτατο όργανο και εξασφαλίζει τησυμμετοχή όλων των μελών στη λήψη αποφάσεων. ΗΓΣ αποφασίζει για κάθε θέμα που ανάγεται στους σκοπούς της, ιδίως, δε, ελέγχει τις πράξεις του ΔΣ, εγκρίνειτον ετήσιο προϋπολογισμό και απολογισμό του ΣΚΛΕ,εγκρίνει τον εσωτερικό κανονισμό λειτουργίας του ΣΚΛΕ.Οι αποφάσεις της ΓΣ υπερισχύουν πάντοτε των αποφάσεων του ΔΣ, των Περιφερειακών Συμβουλίων και τωνΠεριφερειακών Γενικών Συνελεύσεων.
\paragraph { 3. } Η ΓΣ βρίσκεται σε απαρτία, όταν παρευρίσκονταιτουλάχιστον τα μισά εκ των οικονομικώς τακτοποιημένων μελών της. Αν δεν επιτευχθεί απαρτία, η συνεδρίαση αναβάλλεται για την ίδια ημέρα της επόμενηςεβδομάδας, οπότε θεωρείται ότι υπάρχει απαρτία εάνπαρίσταται το ήμισυ του οριζόμενου στο προηγούμενοεδάφιο αριθμού μελών.
\paragraph { 4. } Η ΓΣ συνέρχεται τακτικά μία (1) φορά κάθε έτος,εντός δύο (2) μηνών από τη λήξη του οικονομικού έτουςκαι εκτάκτως, εφόσον κρίνεται αυτό αναγκαίο από τοΔΣ ή εάν αυτό ζητείται από το ένα τρίτο (1/3) των μελών της, με έγγραφη αίτηση που υποβάλλεται στο ΔΣκαι στην οποία αναγράφονται τα θέματα που πρέπει νασυζητηθούν.
\paragraph { 5. } Τη σύγκληση της ΓΣ αποφασίζει το ΔΣ. Τα μέλητης ΓΣ προσκαλούνται από τον Πρόεδρο και τον Γενικό Γραμματέα του ΣΚΛΕ, με ατομικές προσκλήσεις πουαποστέλλονται ταχυδρομικώς ή ηλεκτρονικώς ή με τηλεομοιοτυπία ή με κάθε άλλο πρόσφορο μέσο, όπως ανάρτηση στην ιστοσελίδα του ΣΚΛΕ, το αργότερο τριάντα(30) ημέρες πριν από τη συνεδρίαση. Στις προσκλήσειςορίζεται ο τόπος, ο χρόνος και η ημερήσια διάταξη τηςσυνεδρίασης. Εκτός από τα θέματα που αναγράφονταιστην πρόσκληση, μπορεί να συζητηθεί και άλλο θέμα,εφόσον προταθεί και γίνει δεκτό από την πλειοψηφίατων παρόντων κατά τη συνεδρίαση μελών.
\paragraph { 6. } Οι αποφάσεις της ΓΣ λαμβάνονται με απόλυτη πλειοψηφία των παρόντων μελών. Σε περίπτωση ισοψηφίας,δεν λαμβάνεται απόφαση.
\paragraph { 7. } Η ψηφοφορία για κάθε προσωπικό θέμα ή για τηνεκλογή προσώπου είναι μυστική.
\paragraph { 8. } Στην έναρξη κάθε συνεδρίασης η ΓΣ εκλέγει Πρόεδρο και δύο (2) Γραμματείς.
\paragraph { 9. } Στις συνεδριάσεις της ΓΣ τηρούνται Πρακτικά απότου δύο Γραμματείς, τα οποία υπογράφονται και σφραγίζονται από τον Πρόεδρο της ΓΣ, καθώς και από τονΠρόεδρο και τον Γενικό Γραμματέα του ΣΚΛΕ.
\subsection*{ Άρθρο 82 }
\paragraph { 1. } Η ημερήσια διάταξη της ΓΣ περιέχει υποχρεωτικάτα εξής θέματα: α) απολογισμό των πεπραγμένων τουέτους, β) έγκριση προϋπολογισμού και απολογισμούύστερα από ανακοίνωση της Έκθεσης της ΕξελεγκτικήςΕπιτροπής, γ) προτάσεις για τον προϋπολογισμό και τοπρόγραμμα δράσης. Στην ημερήσια διάταξη συμπεριλαμβάνεται και κάθε άλλο ζήτημα που ορίζεται από τοΔΣ στην πρόσκληση. Ειδικά στη ΓΣ του έτους των αρχαιρεσιών εκλέγεται η Εφορευτική Επιτροπή που θα τιςδιενεργήσει.
\paragraph { 2. } Το ΔΣ διατηρεί τη δυνατότητα να καλεί οποιοδήποτεπρόσωπο κρίνει αναγκαίο προκειμένου να εκφέρει γνώμη επί θεμάτων της ημερήσιας διάταξης.
\subsection*{ Άρθρο 83 }
\paragraph { 1. } Το ΔΣ του ΣΚΛΕ αποτελείται από δεκατρία (13) μέλη,που εκλέγονται από τα οικονομικά τακτοποιημένα μέληΤεύχος Α’ 137/13.09.2017με τριετή θητεία. Μέσα σε επτά (7) ημέρες από τη διεξαγωγή των εκλογών, ο πλειοψηφήσας σύμβουλος τουπλειοψηφήσαντος συνδυασμού και σε περίπτωση ισοψηφίας το αρχαιότερο μέλος βάσει του χρόνου εγγραφήςστο Μητρώο του Περιφερειακού Τμήματος που ανήκει,καλεί τους συμβούλους που έχουν εκλεγεί προς εκλογήΠροέδρου, Αντιπροέδρου Α΄ και Β΄, Γενικού Γραμματέα,Οργανωτικού Γραμματέα, Ταμία, Υπευθύνου Διαδικτύουκαι Ηλεκτρονικών Εκδόσεων και Υπευθύνου ΔημοσίωνΣχέσεων. Η εκλογή γίνεται με μυστική ψηφοφορία και μεαπόλυτη πλειοψηφία των παριστάμενων μελών.Αν δεν επιτευχθεί απόλυτη πλειοψηφία, η εκλογή επαναλαμβάνεται μεταξύ των δύο (2) πρώτων και εκλέγεταιαυτός που έλαβε τις περισσότερες ψήφους.
\paragraph { 2. } Το ΔΣ συγκαλείται από τον Πρόεδρο του και σε περίπτωση άρνησης ή αδυναμίας του Προέδρου από τονΑντιπρόεδρο Α΄ ή οποιοδήποτε άλλο μέλος. Ο Πρόεδροςυποχρεούται να συγκαλέσει το ΔΣ μέσα σε οκτώ (8) ημέρες, εφόσον αυτό ζητηθεί από έξι (6) τουλάχιστον μέλη.
\paragraph { 3. } Το ΔΣ συνεδριάζει τακτικώς τουλάχιστον δύο (2)φορές το μήνα και εκτάκτως όποτε κρίνεται αναγκαίο, μετην επιφύλαξη της παραγράφου 2 του παρόντος.
\paragraph { 4. } Το ΔΣ βρίσκεται σε απαρτία εάν παρίστανται επτά (7)τουλάχιστον από τα μέλη του και λαμβάνει αποφάσεις μεπλειοψηφία των παρόντων. Προκειμένου περί προσωπικών ζητημάτων οι αποφάσεις λαμβάνονται με μυστικήψηφοφορία. Σε περίπτωση ισοψηφίας, η ψηφοφορίαεπαναλαμβάνεται και σε περίπτωση δεύτερης ισοψηφίαςυπερισχύει η ψήφος του Προέδρου. Για τη συνεδρίαση του ΔΣ τηρούνται πρακτικά που υπογράφονται απόόλους όσοι παρίστανται.
\subsection*{ Άρθρο 84 }
\paragraph { 1. } Το ΔΣ έχει τις ακόλουθες αρμοδιότητες:α) Διοικεί τον ΣΚΛΕ, συγκαλεί τη ΓΣ και εκτελεί τις αποβ) Αποφασίζει για κάθε θέμα που ανατίθεται σε αυτόφάσεις της.από τη ΓΣ.γ) Εποπτεύει και συντονίζει τις ενέργειες των ΠΤ.δ) Καταρτίζει τον προϋπολογισμό και απολογισμό.ε) Αποφασίζει για την ίδρυση, κατάργηση και συγχώνευση των ΠΤ.στ) Διοργανώνει κάθε χρόνο συνέδρια.ζ) Συστήνει επιτροπές με συγκεκριμένο έργο.η) Εκδίδει περιοδικό, το οποίο αποστέλλεται έντυπαή/και ηλεκτρονικά σε όλα τα μέλη του ΣΚΛΕ.θ) Συνεργάζεται με φορείς στην Ελλάδα ή/και το εξωτερικό για θέματα που αφορούν τα μέλη του ΣΚΛΕ.ι) Αποφασίζει για την προσχώρηση ή αποχώρησητου ΣΚΛΕ από άλλους συλλογικούς φορείς, όπως Συνδέσμους, Ενώσεις, Οργανώσεις με συναφείς σκοπούςκαι δράσεις.ια) Δύναται να συγκαλεί σε κοινή σύσκεψη τους Προέδρους των ΠΤ, οι οποίοι υποχρεούνται να παραστούν.Τα πορίσματα των συσκέψεων αυτών αξιολογούνται απότο ΔΣ του ΣΚΛΕ.ιβ) Δύναται να συστήνει επιτροπές εργασίας για τηνπροώθηση των σκοπών του. Οι επιτροπές έχουν ως αντικείμενο την έρευνα και μελέτη των προβλημάτων τηςαρμοδιότητάς τους και συνεπικουρούν το έργο του ΔΣμε τη διατύπωση εισηγήσεων.
\paragraph { 2. } Μέλος του ΔΣ που απουσιάζει αδικαιολόγητα επίτέσσερις (4) συνεχείς τακτικές συνεδριάσεις εκπίπτει αυτοδίκαια από τη θέση του και αντικαθίσταται από τονπρώτο αναπληρωματικό. Μετά από τρεις (3) συνεχείςαπουσίες ο Γενικός Γραμματέας του ΔΣ οφείλει να ειδοποιήσει εγγράφως το μέλος του ΔΣ που απουσίαζε γιατις συνέπειες της τέταρτης συνεχόμενης απουσίας του.
\paragraph { 3. } Αν κενωθεί κάποια θέση στο ΔΣ καταλαμβάνεταιαπό τον πρώτο επιλαχόντα του αντίστοιχου συνδυασμού και σε περίπτωση κένωσης θέσεως μεμονωμένουυποψηφίου, από άλλον μεμονωμένο υποψήφιο με τουςπερισσότερους σταυρούς, ελλείψει, δε, άλλου μεμονωμένου υποψηφίου από τον πρώτο επιλαχόντα τουπλειοψηφήσαντος συνδυασμού. Σε περίπτωση κένωσηςτριών (3) θέσεων του ΔΣ, συγκαλείται έκτακτη ΓΣ για τηνεκλογή των μελών που λείπουν και για τον υπολειπόμενοχρόνο θητείας του τρέχοντος ΔΣ.
\paragraph { 4. } Ο Πρόεδρος, οι Αντιπρόεδροι, ο Γενικός Γραμματέας,ο Οργανωτικός Γραμματέας, ο Ταμίας, ο Υπεύθυνος Διαδικτύου και Ηλεκτρονικών Εκδόσεων και ο ΥπεύθυνοςΔημοσίων Σχέσεων του ΔΣ του ΣΚΛΕ, εφόσον είναι υπάλληλοι του δημόσιου τομέα δικαιούνται άδεια απουσίαςαπό τον αρμόδιο Υπουργό, κατ’ ανώτατο όριο δεκαπέντε(15) εργάσιμες ημέρες μηνιαίως, όσο διαρκεί η θητείατους. Κατά το χρόνο αυτόν διατηρούν όλα τα δικαιώματαπου απορρέουν από την υπαλληλική τους ιδιότητα. Ταάλλα μέλη του ΔΣ του ΣΚΛΕ δικαιούνται άδεια απουσίαςεννέα (9) ημερών το μήνα.
\paragraph { 5. } Οι θέσεις στο ΔΣ είναι τιμητικές και άμισθες. Τα μέλητου ΔΣ του ΣΚΛΕ, εφόσον διαμένουν εκτός Αττικής, δικαιούνται να αποζημιώνονται για τις δαπάνες μετακίνησης και διαμονής τους προκειμένου να παραστούνστις συνεδριάσεις του ΔΣ ή για να εκπροσωπήσουν τονΣΚΛΕ σε πάσης φύσεως εκδηλώσεις, εφόσον έχουν ειδικώς εξουσιοδοτηθεί προς τούτο από το ΔΣ. Το ύψοςτης αποζημίωσης καθορίζεται με απόφαση του ΔΣ καιη καταβολή της γίνεται από αυτό. Τα μέλη του ΔΣ τουΣΚΛΕ, εφόσον είναι υπάλληλοι δημόσιου ή ιδιωτικούφορέα, δεν μετατίθενται κατά τη διάρκεια της θητείαςτους χωρίς τη συγκατάθεσή τους.
\subsection*{ Άρθρο 85 }
\paragraph { 1. } Ο Πρόεδρος του ΔΣ του ΣΚΛΕ εκπροσωπεί το φορέαενώπιον κάθε διοικητικής και δικαστικής αρχής. Συγκαλείκαι διευθύνει τις συνεδριάσεις του ΔΣ, καταρτίζει τηνημερήσια διάταξη μαζί με τον Γενικό Γραμματέα, υπογράφει μαζί με τον Γενικό Γραμματέα την αλληλογραφία,καθώς και κάθε άλλο έγγραφο ή πιστοποιητικό. υπογράφει μαζί με τον Ταμία τα εντάλματα πληρωμών καικάθε έγγραφο που αφορά την κίνηση των κεφαλαίωντου ΣΚΛΕ.
\paragraph { 2. } Ο Πρόεδρος, όταν κωλύεται ή απουσιάζει, αναπληρώνεται από τον Αντιπρόεδρο Α΄, τον οποίο αναπληρώνει σε περίπτωση κωλύματος ή απουσίας ο Αντιπρόεδρος Β΄.
\paragraph { 3. } Ο Πρόεδρος του ΔΣ του ΣΚΛΕ δεν μπορεί να είναισυγχρόνως και Πρόεδρος Περιφερειακού Συμβουλίου.
\paragraph { 4. } Εφόσον ο Πρόεδρος του ΔΣ του ΣΚΛΕ είναι και Πρόεδρος Περιφερειακού Συμβουλίου, επιλέγει μέσα σε τρεις(3) ημέρες το αξίωμα το οποίο επιθυμεί να διατηρήσει.Αν η προθεσμία αυτή παρέλθει άπρακτη, τεκμαίρεταιότι παραιτήθηκε από το αξίωμα του Προέδρου του Περιφερειακού Συμβουλίου.
\subsection*{ Άρθρο 86 }
\paragraph { 0. } Ο Αντιπρόεδρος Α΄ έχει την ευθύνη για το συντονισμόκαι την αποτελεσματική λειτουργία των Επιτροπών Εργασίας, εισηγείται τη συγκρότησή τους και παρουσιάζειτα αποτελέσματα των εργασιών τους στο ΔΣ, έχει τηνευθύνη για τη διασύνδεση του ΣΚΛΕ με άλλους φορείς,συλλόγους, οργανώσεις, καθώς και για τη συμμετοχήτου ΣΚΛΕ σε Εθνικά Δίκτυα.
\subsection*{ Άρθρο 87 }
\paragraph { 0. } Ο Αντιπρόεδρος Β΄ έχει την ευθύνη επικοινωνίας με άλλους διεθνείς οργανισμούς ή ομοειδή νομικά πρόσωπαάλλων χωρών και συμμετοχής σε ευρωπαϊκά και διεθνήόργανα. Στα καθήκοντά του αυτά δύναται να συνεπικουρείται από Επιτροπή μελών. Επιπλέον, έχει την ευθύνηπαρακολούθησης του οικονομικού αντικειμένου συγχρηματοδοτούμενων προγραμμάτων στη υλοποίησητων οποίων συμμετέχει ο ΣΚΛΕ.
\subsection*{ Άρθρο 88 }
\paragraph { 1. } Ο Γενικός Γραμματέας έχει την εποπτεία και τηνευθύνη της ομαλής λειτουργίας των υπηρεσιών τουΣΚΛΕ. Τηρεί την αλληλογραφία, είναι υπεύθυνος για τοαρχείο, το μητρώο και τη σφραγίδα του, συντάσσει σεσυνεργασία με τον Πρόεδρο την ημερήσια διάταξη τουΔΣ και κάθε άλλο έγγραφο. Τηρεί τα πρακτικά του ΔΣ,τα οποία και συνυπογράφει με τον Πρόεδρο, καθώς καιπρωτόκολλο εισερχόμενων και εξερχόμενων εγγράφων.
\paragraph { 2. } Ο Γενικός Γραμματέας, όταν κωλύεται ή απουσιάζει,αναπληρώνεται από τον Αναπληρωτή Γραμματέα.
\subsection*{ Άρθρο 89 }
\paragraph { 1. } Ο Ταμίας είναι υπεύθυνος για τη διαχείριση της κινητής και ακίνητης περιουσίας του ΣΚΛΕ. Φροντίζει για τηνείσπραξη των ποσών εγγραφής και ετήσιας εισφοράς τωνμελών, καθώς και για κάθε τακτικό ή έκτακτο έσοδο τουΣΚΛΕ. Τηρεί βιβλίο εσόδων - εξόδων και ενημερώνει το ΔΣγια τα οικονομικά θέματα του ΣΚΛΕ. Ο Ταμίας καταρτίζειτον προϋπολογισμό και τον απολογισμό εσόδων και εξόδων κάθε ημερολογιακού έτους και τον υποβάλλει προςέγκριση στο ΔΣ και στη συνέχεια στη ΓΣ. Τηρεί βιβλίο όπουκαταχωρούνται οι εκθέσεις της Εξελεγκτικής Επιτροπής.
\paragraph { 2. } Κάθε πληρωμή ενεργείται με ένταλμα που υπογράφεται από τον Πρόεδρο, τον Γενικό Γραμματέα καιτον Ταμία. Ο Ταμίας μπορεί να κρατά «εις χείρας» για τιςεπείγουσες δαπάνες του ΣΚΛΕ ποσό καθοριζόμενο κάθεφορά από το ΔΣ και υποχρεώνεται να καταθέτει κάθεεπιπλέον ποσό σε τραπεζικό λογαριασμό που ορίζεταιαπό το ΔΣ.
\paragraph { 3. } Κάθε είσπραξη για λογαριασμό του ΣΚΛΕ διενεργείται από τον Ταμία ή εντεταλμένο υπάλληλο του ΣΚΛΕ.
\paragraph { 4. } Ο Ταμίας, όταν κωλύεται ή απουσιάζει, αναπληρώνεται από άλλο μέλος του ΔΣ που ορίζεται ως αναπληρωτής του από αυτό.
\subsection*{ Άρθρο 90 }
\paragraph { 0. } Ο Οργανωτικός Γραμματέας έχει την ευθύνη επικοινωνίας του ΔΣ με τα Περιφερειακά Τμήματα του ΣΚΛΕ, διεκπεραιώνει την αλληλογραφία με αυτά και ενημερώνειτον Γενικό Γραμματέα για τα θέματα της αρμοδιότητάςτου. Όταν απουσιάζει η κωλύεται αναπληρώνεται απόάλλο μέλος του ΔΣ που ορίζεται από αυτό.
\subsection*{ Άρθρο 91 }
\paragraph { 0. } Ο Υπεύθυνος Διαδικτύου και Ηλεκτρονικών Εκδόσεωνέχει την ευθύνη για τη δημιουργία και συνεχή ανανέωσητης ιστοσελίδας του Ν.Π.Δ.Δ., τη λειτουργία του ηλεκτρονικού περιοδικού και την παρουσία του ΣΚΛΕ στα μέσαηλεκτρονικής δικτύωσης.
\subsection*{ Άρθρο 92 }
\paragraph { 0. } Ο Υπεύθυνος Δημοσίων Σχέσεων έχει την ευθύνη γιατην οργάνωση εκδηλώσεων και ειδικών δράσεων τουΝ.Π.Δ.Δ.. Σε περίπτωση κωλύματος ή απουσίας αναπληρώνεται από τον Οργανωτικό Γραμματέα.
\subsection*{ Άρθρο 93 }
\paragraph { 0. } Τα υπόλοιπα μέλη του ΔΣ του ΣΚΛΕ αναλαμβάνουν συγκεκριμένα καθήκοντα που τους ανατίθενται με απόφαση του ΔΣ και μπορούν να αποτελούν τους συνδέσμουςτου ΔΣ με τις επιτροπές που λειτουργούν στα πλαίσιατου Ν.Π.Δ.Δ..
\subsection*{ Άρθρο 94 }
\paragraph { 1. } Στο ΔΣ του ΣΚΛΕ ασκείται έλεγχος νομιμότητας καιοικονομικός έλεγχος από πενταμελή Κεντρική Εξελεγκτική Επιτροπή, τα τακτικά μέλη της οποίας μαζί με δύο (2)αναπληρωματικά εκλέγονται ταυτόχρονα με την εκλογήτου ΔΣ και με ισόχρονη τριετή θητεία.
\paragraph { 2. } Η Κεντρική Εξελεγκτική Επιτροπή ενημερώνεται γιατην έκθεση πεπραγμένων που υποβάλλει στη ΓΣ ο Γραμματέας και ο Ταμίας. Στη συνέχεια συντάσσει τη δική τηςέκθεση σε ειδική συνεδρία και τη θέτει υπόψη της ΓΣ.
\paragraph { 3. } Η Κεντρική Εξελεγκτική Επιτροπή συνέρχεται τακτικά μία (1) φορά το χρόνο και προ της τακτικής ΓενικήςΣυνέλευσης και έκτακτα όποτε χρειαστεί. Οι αποφάσειςτης λαμβάνονται κατά πλειοψηφία.
\subsection*{ Άρθρο 95 }
\paragraph { 1. } Η Συνέλευση των Προέδρων αποτελείται από ταμέλη του ΔΣ και τους Προέδρους και ένα (1) μέλος τωνΤεύχος Α’ 137/13.09.2017Περιφερειακών Συμβουλίων των ΠΤ. Το μέλος ορίζεταιμε απόφαση του οικείου Περιφερειακού Συμβουλίου.
\paragraph { 2. } Η Συνέλευση των Προέδρων συνεδριάζει στις αρχέςΝοεμβρίου εκάστου έτους. Πρόεδρος της Συνέλευσηςείναι ο Πρόεδρος του ΔΣ, που έχει την ευθύνη για τησύγκλησή της.
\paragraph { 3. } Αρμοδιότητα της Συνέλευσης των Προέδρων είναιο καθορισμός της γενικής πολιτικής του ΣΚΛΕ, η επίλυσηπροβλημάτων που ανακύπτουν από τη λειτουργία τωνΠΤ και η παροχή οδηγιών και κατευθύνσεων για την αρμονική συνεργασία μεταξύ ΠΤ και ΔΣ.
\paragraph { 4. } Η Συνέλευση των Προέδρων βρίσκεται σε απαρτία,όταν παρευρίσκονται τουλάχιστον τα μισά εκ των μελώντης. Αν δεν επιτευχθεί απαρτία, η συνεδρίαση αναβάλλεται για την ίδια ημέρα της επόμενης εβδομάδας, οπότεθεωρείται ότι υπάρχει απαρτία εάν παρίσταται το ήμισυτου οριζόμενου στο προηγούμενο εδάφιο αριθμού μελών.Οι αποφάσεις της Συνέλευσης των Προέδρων λαμβάνονται κατά πλειοψηφία.
\subsection*{ Άρθρο 96 }
\paragraph { 1. } Η Γενική Συνέλευση των ΠΤ του ΣΚΛΕ αποτελείταιαπό όλα τα μέλη του ΣΚΛΕ στη συγκεκριμένη Περιφέρεια, που είναι γραμμένα στα μητρώα του αντίστοιχουΠΤ και έχουν εκπληρώσει τις οικονομικές τους υποχρεώσεις.
\paragraph { 2. } Η άσκηση του εκλογικού δικαιώματος για την εκλογή των αιρετών οργάνων του ΣΚΛΕ είναι υποχρεωτικήγια όλα τα μέλη των ΠΤ.
\paragraph { 3. } Η Γενική Συνέλευση του ΠΤ έχει τις εξής αρμοδιότητες:α) Εκλέγει τα μέλη του ΔΣ, της Κεντρικής ΕξελεγκτικήςΕπιτροπής και των Πειθαρχικών Συμβουλίων.β) Εκλέγει τα μέλη του Περιφερειακού Συμβουλίου καιελέγχει τις πράξεις τους.γ) Αποφασίζει για κάθε ζήτημα που υποβάλλει σε αυτήτο Περιφερειακό Συμβούλιο και ανάγεται στους σκοπούςτου ΠΤ ή όταν το ζητά εγγράφως το ένα τρίτο των μελών.
\paragraph { 4. } Η Γενική Συνέλευση συνέρχεται τακτικά μέσα σε έναδίμηνο από τη λήξη του οικονομικού έτους και έκτακταμε απόφαση του Περιφερειακού Συμβουλίου ή ύστερααπό έγγραφη αίτηση του ενός τρίτου (1/3) των μελών.Οι αιτούντες πρέπει να έχουν εκπληρώσει τις οικονομικές τους υποχρεώσεις και οφείλουν να παρίστανται κατάτη Γενική Συνέλευση. Τυχόν αδικαιολόγητη απουσία τουςαπό αυτήν αποτελεί πειθαρχικό παράπτωμα.
\paragraph { 5. } Τα μέλη του ΠΤ καλούνται από τον Πρόεδρο και τονΓενικό Γραμματέα με γενική πρόσκληση που αναγράφει κατά σειρά τα θέματα της ημερήσιας διάταξης, τοντόπο και το χρόνο διεξαγωγής της Γενικής Συνέλευσης.Η πρόσκληση αναρτάται στην ιστοσελίδα του ΣΚΛΕ καιτοιχοκολλείται στα γραφεία του ΠΤ τουλάχιστον δεκαπέντε (15) ημέρες πριν από την ημέρα συνεδρίασης τηςΓενικής Συνέλευσης.
\paragraph { 6. } Τα μέλη που προσέρχονται στη Γενική Συνέλευσηυπογράφουν σε ιδιαίτερο κατάλογο που τηρείται σε κάθεσυνεδρίαση και χρησιμεύει για να βεβαιώνει την απαρτίαή σε ειδικό βιβλίο.
\paragraph { 7. } Η Γενική Συνέλευση βρίσκεται σε απαρτία, ότανπαρευρίσκονται σε αυτή τουλάχιστον τα μισά από ταμέλη της. Αν δεν επιτευχθεί απαρτία, η συνεδρίαση αναβάλλεται για την ίδια ημέρα της επόμενης εβδομάδας,οπότε θεωρείται ότι υπάρχει απαρτία εάν παρίσταται τοήμισυ του οριζόμενου στο προηγούμενο εδάφιο αριθμού μελών. Οι αποφάσεις της Γενικής Συνέλευσης, μεεξαίρεση τις αρχαιρεσίες, λαμβάνονται με πλειοψηφίατων παρόντων και ανακοινώνονται στο ΔΣ του ΣΚΛΕ.
\subsection*{ Άρθρο 97 }
\paragraph { 1. } Το Περιφερειακό Συμβούλιο, ανάλογα με τον αριθμότων μελών του ΠΤ, αποτελείται:α) Για ΠΤ που αριθμούν έως διακόσια (200) μέλη, απότον Πρόεδρο, τον Γενικό Γραμματέα, τον Ταμία και δύο(2) μέλη (5μελές).β) Για ΠΤ που αριθμούν πάνω από διακόσια ένα (201)μέλη, από τον Πρόεδρο, τον Αντιπρόεδρο, τον ΓενικόΓραμματέα, τον Αναπληρωτή Γραμματέα, τον Ταμία καιδύο (2) μέλη (7μελές).
\paragraph { 10. } 0. Τα μέλη των Περιφερειακών Συμβουλίων του ΣΚΛΕ,εφόσον διαμένουν έξω από την έδρα του ΠΤ, δικαιούνται να αποζημιώνονται για τις δαπάνες μετακίνησηςκαι διαμονής, προκειμένου να πάρουν μέρος στις συνεδριάσεις των Περιφερειακών Συμβουλίων. Το ύψος τηςαποζημίωσης καθορίζεται από το ΔΣ του ΣΚΛΕ το οποίοείναι αρμόδιο και για την καταβολή της.
\paragraph { 11. } 1. Τα μέλη των Περιφερειακών Συμβουλίων του ΣΚΛΕ,εφόσον είναι υπάλληλοι του δημόσιου ή ιδιωτικού φορέα, δεν μετατίθενται κατά τη διάρκεια της θητείας τους,χωρίς τη συγκατάθεσή τους.
\paragraph { 2. } Μέσα σε οκτώ (8) ημέρες από την επικύρωση τηςεκλογής των μελών του Περιφερειακού Συμβουλίου, οπλειοψηφήσας υποψήφιος του πλειοψηφήσαντος συνδυασμού ή ο πλειοψηφήσας μεμονωμένος υποψήφιοςκαι, σε περίπτωση ισοψηφίας, ο αρχαιότερος, βάσει τουχρόνου εγγραφής στο Μητρώο του ΠΤ όπου ανήκει, καλεί τους συμβούλους που έχουν εκλεγεί προς εκλογήΠροέδρου, Αντιπροέδρου, Γενικού Γραμματέα, Αναπληρωτή Γραμματέα και Ταμία. Αυτοί εκλέγονται με μυστικήψηφοφορία και με απόλυτη πλειοψηφία των παριστάμενων μελών. Αν δεν επιτευχθεί απόλυτη πλειοψηφία, ηεκλογή επαναλαμβάνεται μεταξύ των δύο πρώτων καιαρκεί σχετική πλειοψηφία για την εκλογή.
\paragraph { 3. } Το Περιφερειακό Συμβούλιο συγκαλείται από τονΠρόεδρό του. Ο Πρόεδρος υποχρεώνεται να συγκαλέσειτο Περιφερειακό Συμβούλιο μέσα σε οκτώ (8) ημέρες,εφόσον αυτό ζητηθεί από δύο τρίτα (2/3) των μελών του.
\paragraph { 4. } Το Περιφερειακό Συμβούλιο συνεδριάζει τουλάχιστον μία (1) φορά το μήνα και εκτάκτως όποτε κρίνεταιαναγκαίο.
\paragraph { 5. } Το Περιφερειακό Συμβούλιο βρίσκεται σε απαρτίαεάν οι παρόντες είναι περισσότεροι από τους απόντεςκαι λαμβάνει αποφάσεις με πλειοψηφία των παρόντων.Προκειμένου περί προσωπικών ζητημάτων ή περί άλλων σοβαρών θεμάτων, οι αποφάσεις λαμβάνονται μεμυστική ψηφοφορία. Σε περίπτωση ισοψηφίας δεν λαμβάνεται απόφαση. Για τη συνεδρίαση του ΠεριφερειακούΣυμβουλίου τηρούνται πρακτικά που υπογράφονται απόόλους όσοι παρίστανται.
\paragraph { 6. } Το Περιφερειακό Συμβούλιο έχει τις εξής αρμοδιότητες:α) Διοικεί το ΠΤ.β) Εκτελεί τις αποφάσεις της ΓΣ του ΠΤ.γ) Ενημερώνει για τις δραστηριότητες και τις αποφάσεις του το ΔΣ του ΣΚΛΕ.δ) Καταρτίζει τον προϋπολογισμό - απολογισμό και τοετήσιο Σχέδιο Δράσης του ΠΤ, που υποβάλλονται προςέγκριση στο ΔΣ.
\paragraph { 7. } Οι αποφάσεις του Περιφερειακού Συμβουλίου υπόκεινται σε έλεγχο νομιμότητας από τη ΓΣ του ΠΤ και τοΔΣ του ΣΚΛΕ.
\paragraph { 8. } Μέλος του Περιφερειακού Συμβουλίου που απουσιάζει αδικαιολόγητα επί τρεις (3) συνεχείς τακτικέςσυνεδριάσεις, εκπίπτει αυτοδίκαια από τη θέση του καιαναπληρώνεται από το πρώτο στη σειρά αναπληρωματικό μέλος. Μετά από δύο (2) συνεχείς απουσίες ο ΓενικόςΓραμματέας του Περιφερειακού Συμβουλίου οφείλει ναειδοποιεί εγγράφως το μέλος που απουσίαζε για τις συνέπειες της τρίτης συνεχόμενης απουσίας του.
\paragraph { 9. } Οι Πρόεδροι και οι Γενικοί Γραμματείς των Περιφερειακών Συμβουλίων του ΣΚΛΕ, εφόσον είναι υπάλληλοιτου δημόσιου τομέα, λαμβάνουν άδεια απουσίας απότην υπηρεσία τους εννέα (9) ημερών το μήνα για όσοχρονικό διάστημα διαρκεί η θητεία τους. Κατά το χρόνοαυτόν διατηρούν όλα τα δικαιώματα που απορρέουναπό την υπαλληλική τους ιδιότητα. Τα άλλα μέλη τωνΠεριφερειακών Συμβουλίων του ΣΚΛΕ δικαιούνται άδειααπουσίας τεσσάρων (4) ημερών το μήνα.
\subsection*{ Άρθρο 98 }
\paragraph { 1. } Ο Πρόεδρος του Περιφερειακού Συμβουλίου τουΣΚΛΕ εκπροσωπεί το ΠΤ ενώπιον κάθε διοικητικής ήδικαστικής αρχής. Συγκαλεί και διευθύνει τις συνεδριάσεις του Περιφερειακού Συμβουλίου και τη ΓΣ του ΠΤ,εισάγει στο ΔΣ την ετήσια έκθεση πεπραγμένων του Περιφερειακού Συμβουλίου, υπογράφει μαζί με τον ΓενικόΓραμματέα την αλληλογραφία, καθώς και κάθε άλλο έγγραφο ή πιστοποιητικό, υπογράφει μαζί με τον Ταμία ταεντάλματα των πληρωμών και κάθε έγγραφο που αφοράτην κίνηση των κεφαλαίων του ΠΤ του ΣΚΛΕ.
\paragraph { 2. } Σε περίπτωση κωλύματος ή απουσίας, ο Πρόεδροςαναπληρώνεται από τον Αντιπρόεδρο ή από άλλο μέλοςτου ΔΣ που ορίζεται με απόφασή του.
\subsection*{ Άρθρο 99 }
\paragraph { 1. } Ο Γενικός Γραμματέας έχει την εποπτεία και την ευθύνη της ομαλής λειτουργίας των υπηρεσιών του ΠΤ τουΣΚΛΕ. Τηρεί την αλληλογραφία, είναι υπεύθυνος για τοαρχείο, το μητρώο και τη σφραγίδα της, συντάσσει τηνετήσια έκθεση πεπραγμένων των Περιφερειακών Συμβουλίων του ΣΚΛΕ, την οποία υποβάλλει στον Πρόεδρομαζί με κάθε άλλο έγγραφο. Τηρεί τα πρακτικά των Περιφερειακών Συμβουλίων και των Γενικών Συνελεύσεωντων ΠΤ, τα οποία συνυπογράφει με τους Προέδρους καιτο πρωτόκολλο εισερχομένων και εξερχόμενων εγγράφων.
\paragraph { 2. } Σε περίπτωση κωλύματος ή απουσίας ο ΓενικόςΓραμματέας αναπληρώνεται από τον Αναπληρωτή Γραμματέα και ελλείψει τούτου από άλλο μέλος του Περιφερειακού Συμβουλίου που ορίζεται από αυτό.
\subsection*{ Άρθρο 100 }
\paragraph { 1. } Ο Ταμίας, μετά από απόφαση του ΔΣ, είναι υπεύθυνος για την εν γένει διαχείριση της κινητής και ακίνητηςπεριουσίας του ΠΤ. Τηρεί βιβλίο εσόδων - εξόδων καιενημερώνει το Περιφερειακό Συμβούλιο για τα οικονομικά θέματα. Ο Ταμίας συντάσσει τον απολογισμό εσόδωνκαι εξόδων κάθε ημερολογιακού έτους και τον υποβάλλει προς κατάρτιση στο Περιφερειακό Συμβούλιο και στησυνέχεια προς έγκριση στη Γενική Συνέλευση του ΠΤ.Επίσης καταθέτει τον απολογισμό και προϋπολογισμόστο ΔΣ του ΣΚΛΕ. Τηρεί βιβλίο όπου καταχωρούνται οιεκθέσεις της Εξελεγκτικής Επιτροπής.
\paragraph { 2. } Κάθε πληρωμή γίνεται με ένταλμα που υπογράφεταιαπό τον Πρόεδρο, τον Γενικό Γραμματέα και τον Ταμία.Ο Ταμίας μπορεί να κρατά «εις χείρας» για τις επείγουσες δαπάνες του ΠΤ του ΣΚΛΕ ποσό καθοριζόμενο κάθεφορά από το Περιφερειακό Συμβούλιο και υποχρεώνεταινα καταθέτει κάθε επιπλέον ποσό σε τραπεζικό λογαριασμό που ορίζεται από το Περιφερειακό Συμβούλιο.
\paragraph { 3. } Κάθε είσπραξη για λογαριασμό του ΠΤ του ΣΚΛΕ διενεργείται από τον Ταμία ή εντεταλμένο υπάλληλο του ΠΤ.
\paragraph { 4. } Όταν ο Ταμίας κωλύεται ή απουσιάζει αναπληρώνεται από άλλο μέλος του Περιφερειακού Συμβουλίουπου ορίζεται από αυτό.
\subsection*{ Άρθρο 101 }
\paragraph { 0. } Το ΔΣ του ΣΚΛΕ και τα Περιφερειακά Συμβούλια έχουντο δικαίωμα να καταρτίζουν επιτροπές αποτελούμενεςαπό ένα (1) μέλος αυτών ως Πρόεδρο και δύο (2) ή περισσότερα μέλη. Οι επιτροπές αυτές καταρτίζονται γιασυγκεκριμένο έργο ή συγκεκριμένη περίοδο, προκειμένου να συμβάλουν στη μελέτη και επίλυση ειδικήςφύσεως υφιστάμενων ή εκτάκτως εμφανιζόμενων στονΣΚΛΕ θεμάτων, απαρτίζονται δε από μέλη που έχουνειδικότητα που σχετίζεται προς τα υπό μελέτη θέματα.Το ΔΣ και τα Περιφερειακά Συμβούλια μπορεί να συνεδριάζουν από κοινού με την επιτροπή για την καλύτερηενημέρωσή τους επί του υπό κρίση θέματος. Με τις αυτέςδιαδικασίες μπορούν να δημιουργούνται ομάδες εργασίας ή ερευνητικές ομάδες με συγκεκριμένο σκοπό, όπωςη εκπόνηση μίας ερευνητικής μελέτης.
\subsection*{ Άρθρο 102 }
\paragraph { 1. } Τα μέλη του ΔΣ, της Κεντρικής Εξελεγκτικής Επιτροπής και των Πειθαρχικών Συμβουλίων εκλέγονταιανά τριετία από τη ΓΣ του ΣΚΛΕ με καθολική, μυστικήψηφοφορία.Τεύχος Α’ 137/13.09.2017Τα Περιφερειακά Συμβούλια εκλέγονται ανά τριετίααπό τη Γενική Συνέλευση εκάστου ΠΤ με καθολική, μυστική ψηφοφορία.Ειδικά κατά την πρώτη εφαρμογή του παρόντος, ηθητεία των πρώτων αιρετών οργάνων του ΣΚΛΕ λήγειτο πρώτο δεκαπενθήμερο του έτους κατά το οποίο συμπληρώνεται η τριετής θητεία τους.
\paragraph { 2. } Στη Γενική Τακτική Συνέλευση του τρίτου χρόνου θητείας του ΔΣ, μετά τη συζήτηση των θεμάτων της ημερησίας διάταξης, εκλέγεται η Κεντρική Εφορευτική Επιτροπή (ΚΕΕ) με την παρουσία αντιπροσώπου της δικαστικήςαρχής, με το εκλογικό σύστημα της απλής αναλογικής,με μυστικότητα, με τη μέριμνα της ψηφολεκτικής τριμελούς επιτροπής, η οποία εκλέγεται από τα μέλη της ΓΣ με«ανάταση της χειρός». Ταυτόχρονα ορίζεται η μέρα τωναρχαιρεσιών. Σε πέντε (5) ημέρες από την εκλογή της, ηΕφορευτική Επιτροπή συνέρχεται. Πρόεδρός της είναι οαντιπρόσωπος της δικαστικής αρχής.
\paragraph { 3. } Η ΚΕΕ είναι το κεντρικό όργανο εποπτείας στη διενέργεια των αρχαιρεσιών για την ανάδειξη των αιρετώνοργάνων του ΣΚΛΕ. Αποτελείται από πέντε (5) τακτικάμέλη, εκ των οποίων τα τέσσερα (4) μαζί με τα αναπληρωματικά τους είναι τακτικά μέλη του ΣΚΛΕ, ενώ τοπέμπτο και το αναπληρωματικό του μέλος είναι Πρωτοδίκης των Πολιτικών Δικαστηρίων, ο οποίος ορίζεταιΠρόεδρος, κατόπιν υποβολής σχετικής αίτησης στο οικείο Πρωτοδικείο. Υποψήφιοι για την ΚΕΕ δύνανται ναείναι όλα τα οικονομικώς τακτοποιημένα μέλη του ΣΚΛΕ.Τα μέλη της ΚΕΕ δεν μπορεί να είναι και υποψήφιοι στηνεκλογική διαδικασία που εποπτεύουν. Για τη διενέργειατων εκλογών στα ΠΤ εκλέγονται ανά ΠΤ ΠεριφερειακέςΕφορευτικές Επιτροπές με την ίδια διαδικασία από τιςοικείες Περιφερειακές Γενικές Συνελεύσεις. Οι Περιφερειακές Εφορευτικές Επιτροπές δύνανται να λειτουργούνακόμη και μόνον με την παρουσία των Προέδρων τους.
\paragraph { 4. } Οι εκλογές για την ανάδειξη όλων των αιρετών οργάνων του ΣΚΛΕ, κεντρικών και περιφερειακών, διεξάγονται την ίδια ημέρα.
\paragraph { 5. } Οι υποψηφιότητες για όλα τα αιρετά όργανα τουΣΚΛΕ υποβάλλονται το αργότερο είκοσι (20) ημέρες πριναπό τις εκλογές, γραπτώς, στην ΚΕΕ. Οι υποψηφιότητεςτων συνδυασμών υποβάλλονται από ειδικώς εξουσιοδοτημένο προς τούτο εκπρόσωπο του συνδυασμού. Οιμεμονωμένοι υποψήφιοι υποβάλλουν υποψηφιότηταείτε αυτοπροσώπως είτε μέσω ειδικώς εξουσιοδοτημένου προς τούτο εκπροσώπου τους. Η ΚΕΕ μετά τονέλεγχο των νομίμων προϋποθέσεων και της ταμειακήςτακτοποίησης των υποψηφίων, ανακηρύσσει τις υποψηφιότητες μέσα στο επόμενο πενθήμερο με σύνταξησχετικού πρακτικού. Το πρακτικό αυτό αποστέλλεται σεόλα τα ΠΤ, αναρτάται επί πέντε (5) ημέρες στα γραφείατου ΣΚΛΕ και των ΠΤ και προσβάλλεται με ένσταση απόκάθε οικονομικώς τακτοποιημένο μέλος. Μετά την πάροδο του πενθημέρου και εφόσον δεν υποβληθεί ένσταση, η ανακήρυξη των υποψηφίων καθίσταται οριστική.Επί των ενστάσεων αποφαίνεται αιτιολογημένα η ΚΕΕεντός τριών (3) ημερών από την πάροδο της πενθήμερηςανάρτησης, τροποποιώντας ενδεχομένως το πρακτικόανακήρυξης και καθιστώντας αυτό οριστικό. Με την ίδιαακριβώς διαδικασία η ΚΕΕ συντάσσει εκλογικούς καταλόγους ανά ΠΤ, οι οποίοι αποστέλλονται έγκαιρα στιςοικείες Περιφερειακές Εφορευτικές Επιτροπές.Η ΚΕΕ έχει την ευθύνη εκτύπωσης των ψηφοδελτίωντων συνδυασμών υποψηφίων και των μεμονωμένωνυποψηφίων, ακολούθως προς το πρακτικό ανακήρυξης,σε χαρτί όμοιας διάστασης και χρώματος για όλους. Γιακάθε συνδυασμό ή μεμονωμένο υποψήφιο εκτυπώνεταιένα ψηφοδέλτιο για το ΔΣ, την Κεντρική ΕξελεγκτικήΕπιτροπή και τα Πειθαρχικά Συμβούλια. Αντίστοιχα, σεκάθε ΠΤ για κάθε συνδυασμό ή μεμονωμένο υποψήφιο εκτυπώνεται ένα ψηφοδέλτιο για το ΠεριφερειακόΣυμβούλιο. Ακολούθως, τα ψηφοδέλτια προωθούνταιέγκαιρα στις Περιφερειακές Εφορευτικές Επιτροπές καιπάντως τουλάχιστον δέκα (10) ημέρες πριν από τις εκλογές. Οι φάκελοι που χρησιμοποιούνται σφραγίζονται καιμονογράφονται από τον Πρόεδρο κάθε ΠεριφερειακήςΕφορευτικής Επιτροπής.
\paragraph { 6. } Εκλογικό σύστημα για την ανάδειξη των μελών τωναιρετών οργάνων του ΣΚΛΕ είναι η απλή αναλογική. Οιυποψήφιοι κατεβαίνουν είτε σε συνδυασμούς είτε ωςμεμονωμένοι σε χωριστά ψηφοδέλτια. Η εκλογή τωνοργάνων διοίκησης του ΣΚΛΕ, γίνεται ως εξής:α) Οι έδρες όλων των αιρετών οργάνων κατανέμονταιμεταξύ των συνδυασμών και των μεμονωμένων υποψηφίων ανάλογα με την εκλογική τους δύναμη.β) Το εκλογικό σύστημα για την ανάδειξη των οργάνωνδιοίκησης είναι αναλογικό. Το εκλογικό μέτρο προκύπτειαπό την διαίρεση του συνόλου των έγκυρων ψηφοδελτίων με τον αριθμό των εδρών του ΔΣ ή της ΚεντρικήςΕξελεγκτικής Επιτροπής ή του Πειθαρχικού Συμβούλιουή του Περιφερειακού Συμβουλίου ή της ΠεριφερειακήςΕξελεγκτικής Επιτροπής.γ) Κάθε συνδυασμός καταλαμβάνει τόσες έδρες στο αιρετό όργανο, όσες φορές χωρεί το εκλογικό μέτρο στοναριθμό των έγκυρων ψηφοδελτίων που έλαβε.δ) Μεμονωμένος υποψήφιος, που έλαβε τον ίδιο ήμεγαλύτερο αριθμό ψήφων από το εκλογικό μέτρο καταλαμβάνει μια (1) έδρα στο όργανο, για το οποίο έχειθέσει υποψηφιότητα.ε) Συνδυασμός που περιλαμβάνει υποψήφιους λιγότερους από τις έδρες που του αναλογούν, καταλαμβάνειτόσες έδρες, όσοι είναι οι υποψήφιοί του.στ) Οι έδρες των αιρετών οργάνων που παραμένουναδιάθετες μετά την εφαρμογή της περίπτωσης γ΄ του παρόντος άρθρου, κατανέμονται σε δεύτερη κατανομή ανάμία (1) κατά τη σειρά μεγέθους των υπολοίπων, στουςσυνδυασμούς ή στους μεμονωμένους υποψηφίους.ζ) Κάθε συνδυασμός υποψηφίων καταρτίζεται με έγγραφη δήλωση, υπογεγραμμένη κατά σειρά επωνύμου,από τους αποτελούντες το συνδυασμό υποψηφίους συμβούλους.η) Η εκλογή των υποψηφίων γίνεται με βάση το σύνολοτων σταυρών προτίμησης για τον καθένα τους.Ο εκλογέας δηλώνει την προτίμησή του, ανάμεσα στουςυποψηφίους του συνδυασμού της επιλογής του, σημειώνοντας ένα σταυρό δίπλα στο ονοματεπώνυμό του.θ) Η ψηφοφορία είναι μυστική και διεξάγεται σε κατάλληλα διαρρυθμισμένο χώρο από τις 8:00 π.μ. έως τις8:00μ.μ.. Η διάρκεια τη ψηφοφορίας μπορεί να παραταθεί με απόφαση της αρμόδιας Εφορευτικής Επιτροπής. Η ψηφοφορία στα ΠΤ εκτός Αττικής διεξάγεται στιςέδρες των Τμημάτων και στα γραφεία τους ή σε τόποπου προτείνεται από την κάθε Περιφερειακή ΕφορευτικήΕπιτροπή, με απόφαση της ΚΕΕ. Ειδικά η ψηφοφορίαγια το ΠΤ Αττικής διεξάγεται στα κεντρικά γραφεία τουΣΚΛΕ. ενώπιον της ΚΕΕ. Σε όλες τις Εφορευτικές Επιτροπές προεδρεύουν εκπρόσωποι της Δικαστικής Αρχής.ι) Ειδικά οι ψηφοφόροι που διαμένουν σε νήσο ή πόληόπου δεν λειτουργεί εκλογικό τμήμα έχουν δικαίωμα ναψηφίσουν με επιστολή, η οποία αποστέλλεται συστημένη στην ΚΕΕ επτά (7) τουλάχιστον ημέρες πριν από τηνημέρα των εκλογών. Το απόρρητο της επιστολικής ψήφου διασφαλίζεται διά της τοποθετήσεως σφραγισμένουτου φακέλου με το ψηφοδέλτιο επιλογής του ψηφοφόρου εντός άλλου φακέλου, επί του οποίου αναγράφονται τα στοιχεία του ψηφοφόρου. Η επιστολική ψήφοςδιαβιβάζεται αμελλητί στην ΚΕΕ, η οποία την ημέρα τωνεκλογών, αφού καταγράψει στο βιβλίο ψηφισάντων τονψηφοφόρο, ρίπτει τον φάκελο με το ψηφοδέλτιο τηςεπιλογής του εντός της κάλπης. Όποιος επιθυμεί να ψηφίσει με επιστολική ψήφο οφείλει να ενημερώσει τηνΚΕΕ είκοσι (20) τουλάχιστον ημέρες πριν από τις εκλογές.ια) Κάθε ψηφοφόρος ψηφίζει ένα (1) μόνο συνδυασμόή μεμονωμένο υποψήφιο. Εφόσον ψηφίζει συνδυασμό,έχει δικαίωμα να βάλει μέχρι δεκατρείς (13) σταυρούςγια το ΔΣ, μέχρι πέντε (5) στην εξελεγκτική επιτροπή καιμέχρι τρείς (3) για το πρωτοβάθμιο και δευτεροβάθμιοπειθαρχικό συμβούλιο.ιβ) Ενστάσεις των υποψηφίων, των εξουσιοδοτημένωναντιπροσώπων τους ή των εκλογέων υποβάλλονται γραπτά στην οικεία Εφορευτική Επιτροπή καθ’ όλη τη διάρκεια της ψηφοφορίας και της διαλογής. Κάθε ΕφορευτικήΕπιτροπή αποφαίνεται επί των ενστάσεων με απόφασηπου λαμβάνεται με απλή πλειοψηφία των παρόντων. Οιυποψήφιοι ή οι ειδικώς εξουσιοδοτημένοι εκπρόσωποίτους έχουν δικαίωμα να παρευρίσκονται στην ψηφοφορία και τη διαλογή.ιγ) Τα μέλη ψηφίζουν αυτοπρόσωπα στα εκλογικά κέντρα των ΠΤ όπου ανήκουν. Οι Εφορευτικές Επιτροπέςκατά το χρόνο διεξαγωγής των αρχαιρεσιών τηρούν μητρώο ψηφισάντων, ελέγχοντας τα στοιχεία ταυτότηταςκαι την ταμειακή ενημερότητα των εκλογέων.ιδ) Μετά το πέρας της ψηφοφορίας ανοίγεται η κάλπη, καταμετρούνται οι φάκελοι και διαπιστώνεται αν οαριθμός τους συμφωνεί με τον αριθμό της κατάστασηςψηφισάντων. Στη συνέχεια αποσφραγίζονται οι φάκελοικαι τα ψηφοδέλτια υπογράφονται από τον Πρόεδρο τηςΕφορευτικής Επιτροπής, αφού αναγραφεί στο επάνω αριστερό σημείο τους ο αύξων αριθμός τους και στο επάνωδεξιά ολογράφως ο αριθμός των σταυρών προτίμησηςπου έχουν τεθεί δίπλα από τα ονόματα των υποψηφίων.Ακολουθεί η διαλογή των ψήφων και η σύνταξη-υπογραφή του πρακτικού εκλογής. Οι Εφορευτικές Επιτροπέςτων ΠΤ αμέσως μετά τη διαλογή και αφού έχουν συντάξειτο πρακτικό εκλογής, οφείλουν, το συντομότερο δυνατό,εντός της ημέρας διεξαγωγής των εκλογών, να ανακοινώσουν με κάθε πρόσφορο τρόπο τα αποτελέσματα στηνΚΕΕ και σε καμία περίπτωση πριν την προκαθορισμένηγια όλα τα ΠΤ ώρα λήξης της ψηφοφορίας.ιε) Το υλικό της εκλογής (ψηφοδέλτια, καταστάσειςψηφισάντων, ενδεχόμενες ενστάσεις με τις επ’ αυτώναποφάσεις των εφορευτικών επιτροπών, επικυρωμέναπρακτικά της εκλογής) σφραγίζεται μέσα σε δέμα καιαποστέλλεται στην έδρα του ΣΚΛΕ.
\paragraph { 7. } Στο αρχείο του Τμήματος τηρούνται αντίγραφαπρακτικών των αρχαιρεσιών και των καταστάσεων ψηφισάντων.
\paragraph { 8. } Η ΚΕΕ αμέσως προβαίνει στη κατάρτιση πίνακα συγκεντρωτικών αποτελεσμάτων και στη σύνταξη πρακτικού με το οποίο ανακηρύσσει τα εκλεγμένα μέλη όλωντων αιρετών οργάνων, με παράλληλη μνεία της σειράςτων αναπληρωματικών μελών.
\subsection*{ Άρθρο 103 }
\paragraph { 1. } Πρωτοβάθμιο Πειθαρχικό Συμβούλιο: Το Πρωτοβάθμιο Πειθαρχικό Συμβούλιο είναι αρμόδιο για τα πειθαρχικά παραπτώματα των μελών του ΣΚΛΕ σε πρώτο βαθμό.Αποτελείται από τρία (3) μέλη και ισάριθμα αναπληρωματικά, που εκλέγονται με τριετή θητεία ταυτόχρονα μετην εκλογή των λοιπών αιρετών οργάνων του ΣΚΛΕ.
\paragraph { 2. } Δευτεροβάθμιο Πειθαρχικό Συμβούλιο: Αποτελείταιαπό πέντε (5) μέλη και ισάριθμα αναπληρωματικά, εκτων οποίων ένας Εφέτης Πολιτικών Δικαστηρίων με τοναναπληρωματικό του, ο οποίος και προεδρεύει. Τα μέλητου Δευτεροβάθμιου Πειθαρχικού Συμβουλίου, πλην τουΠροέδρου, εκλέγονται μετά των αναπληρωματικών τουςταυτόχρονα με την εκλογή των λοιπών αιρετών οργάνωντου ΣΚΛΕ. Χρέη Γραμματέα εκτελεί ο Γραμματέας του ΔΣτου ΣΚΛΕ, ο οποίος τηρεί τα πρακτικά της συνεδρίασης.Το Δευτεροβάθμιο Πειθαρχικό Συμβούλιο εκδικάζει σεδεύτερο βαθμό τα πειθαρχικά παραπτώματα των μελώντου ΣΚΛΕ, μετά από έφεση. Είναι αρμόδιο για την εκδίκαση των πειθαρχικών παραπτωμάτων των μελών τουΔΣ του ΣΚΛΕ και των Περιφερειακών Συμβουλίων. Στηνπερίπτωση αυτή δικάζει σε πρώτο και τελευταίο βαθμό.
\subsection*{ Άρθρο 104 }
\paragraph { 0. } Πειθαρχικά παραπτώματα είναι ιδίως:Α. Η παράβαση των καθηκόντων και υποχρεώσεωνπου προβλέπονται για τους κοινωνικούς λειτουργούςαπό τις διατάξεις του παρόντος, του Κώδικα Δεοντολογίας, του εσωτερικού κανονισμού λειτουργίας του ΣΚΛΕ,των αποφάσεων του ΔΣ του ΣΚΛΕ και του ΠεριφερειακούΣυμβουλίου του ΠΤ στο οποίο ανήκει ο κοινωνικός λειτουργός. Το παράπτωμα κρίνεται και τιμωρείται από τοΠειθαρχικό Συμβούλιο με πειθαρχική ποινή, ανεξάρτητααπό ενδεχόμενη ποινική ευθύνη ή άλλη συνέπεια.Β. Η αποδεδειγμένη αμέλεια κατά την εκτέλεση τωνκαθηκόντων του κοινωνικού λειτουργού, ακόμα και ότανδεν αποτελεί βαρύτερη ποινικά κολάσιμη πράξη.
\subsection*{ Άρθρο 105 }
\paragraph { 0. } Οι ποινές που επιβάλλονται από το Πειθαρχικό Συμβούλιο και το Δευτεροβάθμιο Πειθαρχικό ΣυμβούλιοΤεύχος Α’ 137/13.09.2017για τα πειθαρχικά παραπτώματα του άρθρου 104 είναιοι ακόλουθες:Α) Επίπληξη.Β) Πρόστιμο από εξήντα (60) ευρώ έως το ένα τρίτο(1/3) του βασικού μισθού ενός μήνα του κοινωνικού λειτουργού που τιμωρείται. Το πρόστιμο καταβάλλεται στοΔΣ του ΣΚΛΕ και εισπράττεται σύμφωνα με τις διατάξειςτου ΚΕΔΕ.Γ) Προσωρινή διαγραφή από μέλος, η οποία δεν μπορεί να υπερβεί τους έξι (6) μήνες.Δ) Οριστική διαγραφή, εφόσον το μέλος είναι υπότροπο.
\subsection*{ Άρθρο 106 }
\paragraph { 1. } Τα πειθαρχικά παραπτώματα παραγράφονται μετάαπό τριετία από την τέλεσή τους. Η κίνηση της πειθαρχικής διαδικασίας, καθώς και η άσκηση της ποινικήςδίωξης διακόπτει την παραγραφή. Σε κάθε περίπτωσηο χρόνος για την οριστική παραγραφή δεν μπορεί ναυπερβεί τα πέντε (5) χρόνια από το χρόνο τέλεσης τηςπράξης. Το Πειθαρχικό Συμβούλιο μπορεί με απόφασητου να διατάξει την αναστολή της πειθαρχικής δίωξης,εφόσον εκκρεμεί ποινική δίωξη και μέχρι το τέλος αυτής. Σε αυτήν την περίπτωση ο χρόνος παραγραφής τουπειθαρχικού παραπτώματος δεν συμπληρώνεται πρινπεράσει ένα (1) έτος από την τελεσιδικία της απόφασηςτου ποινικού δικαστηρίου.
\paragraph { 2. } Ο Γραμματέας της Εισαγγελίας Πρωτοδικών ή οΓραμματέας αρμόδιου δικαστηρίου αποστέλλει στονΣΚΛΕ αντίγραφα των υποβαλλόμενων κατά των Κοινωνικών Λειτουργών μηνύσεων και των εκδιδόμενων επίαυτών βουλευμάτων και αποφάσεων.
\subsection*{ Άρθρο 107 }
\paragraph { 1. } Η πειθαρχική δίωξη ασκείται από το ΠρωτοβάθμιοΠειθαρχικό Συμβούλιο αυτεπάγγελτα ή με έγγραφη ήπροφορική αναφορά ή ανακοίνωση δημόσιας αρχήςή μετά από αίτηση - καταγγελία κάθε ενδιαφερομένου.
\paragraph { 10. } 0. Μετά την απολογία ή την ερημοδικία του διωκομένου, το Πειθαρχικό Συμβούλιο εκδίδει οριστική απόφασητο αργότερο μέσα σε οκτώ (8) ημέρες από την ημέρασυνεδρίασης.
\paragraph { 11. } 1. Η απόφαση πρέπει να είναι αιτιολογημένη, συντάσσεται από τον εισηγητή, υπογράφεται από τον Πρόεδρο και τον Γραμματέα και κοινοποιείται μέσα σε οκτώ(8) ημέρες από την έκδοση στον κοινωνικό λειτουργό.Ομοίως, κοινοποιείται στο ΔΣ και στο ΠΤ όπου ανήκει οπειθαρχικώς διωκόμενος κοινωνικός λειτουργός.
\paragraph { 12. } 2. Η διαδικασία του παρόντος άρθρου ακολουθείταικαι ενώπιον του Δευτεροβάθμιου Πειθαρχικού Συμβουλίου, οσάκις αυτό εκδικάζει πειθαρχική υπόθεση σε πρώτο και τελευταίο βαθμό.
\paragraph { 2. } Σε διάστημα τριών (3) μηνών το αργότερο από τηνέναρξη της πειθαρχικής δίωξης, το Πειθαρχικό Συμβούλιο οφείλει να εκδώσει οριστική απόφαση.
\paragraph { 3. } Εάν για την ίδια πράξη έχει ασκηθεί ποινική δίωξηκατά του κοινωνικού λειτουργού, το Πειθαρχικό Συμβούλιο δεν εμποδίζεται να εξετάσει την ίδια πράξη καιδικαιούται να αναστείλει, κατά την κρίση του, την πειθαρχική δίωξη μέχρι την έκδοση τελεσίδικης απόφασηςαπό τα ποινικά δικαστήρια. Η αθωωτική ή καταδικαστικήαπόφαση του δικαστηρίου δεν αποτελεί δεδικασμένογια το Πειθαρχικό Συμβούλιο.
\paragraph { 4. } Το Πρωτοβάθμιο Πειθαρχικό Συμβούλιο ύστερα απότην υποβολή σε αυτό καταγγελίας κατά κοινωνικού λειτουργού ή με τη διαπίστωση οποιουδήποτε παραπτώματος, αποφαίνεται αιτιολογημένα μέσα σε δεκαπέντε(15) ημέρες αν θα ασκηθεί πειθαρχική δίωξη ή όχι. Σεκαταφατική περίπτωση το Πρωτοβάθμιο ΠειθαρχικόΣυμβούλιο ασκεί πρωτοβάθμια πειθαρχική δικαιοδοσία.
\paragraph { 5. } Το Πρωτοβάθμιο Πειθαρχικό Συμβούλιο ενεργείκάθε αναγκαία εξέταση με κάποιο από τα μέλη του, πουορίζεται ως εισηγητής. Ο εισηγητής έχει την εξουσία νακαλεί και να εξετάζει μάρτυρες ενόρκως.
\paragraph { 6. } Πειθαρχική ποινή δεν επιβάλλεται πριν απολογηθείή κληθεί εμπρόθεσμα προς απολογία και δεν εμφανιστεί ο κοινωνικός λειτουργός που διώκεται πειθαρχικά.Στην κλήση σε απολογία, η οποία επιδίδεται με απόδειξη,περιγράφεται σαφώς το αποδιδόμενο παράπτωμα καιτάσσεται προθεσμία πέντε (5) ημερών για την υποβολήαπολογητικού υπομνήματος, η οποία μπορεί να παραταθεί άπαξ μέχρι και το διπλάσιο κατόπιν αιτιολογημένηςαίτησης του διωκομένου.
\paragraph { 7. } Μετά την υποβολή έγγραφου απολογητικού υπομνήματος ή την πάροδο της ταχθείσας προθεσμίας, εφόσον έχει ολοκληρωθεί η ανάκριση, ο εισηγητής ενημερώνει τον Πρόεδρο του Πειθαρχικού Συμβουλίου, ο οποίοςορίζει την ημέρα και ώρα συνεδρίασης του ΠειθαρχικούΣυμβουλίου.
\paragraph { 8. } Ο διωκόμενος καλείται τουλάχιστον δέκα (10) ημέρες πριν σε παράσταση ενώπιον του Πειθαρχικού Συμβουλίου. Εάν το Συμβούλιο θεωρεί αναγκαία τη συμπλήρωση των στοιχείων του ανακριτικού υλικού, καλεί μέσαστην ίδια προθεσμία τον διωκόμενο σε συμπληρωματικήαπολογία.
\paragraph { 9. } Το Πειθαρχικό Συμβούλιο, κατά την ημέρα που έχειπροσδιορίσει, μπορεί να εξετάζει μάρτυρες κατά την κρίση του και, μετά την απολογία του διωκομένου ή σε περίπτωση που αυτός δεν εμφανίζεται, αφού διαπιστωθείότι αυτός έχει νόμιμα κληθεί, εκδίδει απόφαση. Εφόσονκριθεί αναγκαίο, μπορεί να διατάξει τη συμπλήρωση τουκατηγορητηρίου και της ανάκρισης. Σε κάθε περίπτωσηο διωκόμενος δικαιούται να παρίσταται με πληρεξούσιοδικηγόρο.
\subsection*{ Άρθρο 108 }
\paragraph { 1. } Ο κοινωνικός λειτουργός που καταδικάστηκε καιτο ΔΣ ή το ΠΤ όπου ανήκει έχουν δικαίωμα μέσα σετρεις (3) μήνες από την επίδοση της απόφασης να τηνεκκαλέσουν ενώπιον του Δευτεροβάθμιου ΠειθαρχικούΣυμβουλίου.
\paragraph { 2. } Η έφεση κατατίθεται στον Γραμματέα του Πρωτοβάθμιου Πειθαρχικού Συμβουλίου και συντάσσεται πράξη κατάθεσης, η οποία υπογράφεται από τον εκκαλούντακαι τον Γραμματέα. Ο Γραμματέας υποχρεούται μέσα σεδέκα (10) ημέρες να τη μεταβιβάσει μαζί με όλα τα σχετικά έγγραφα στη Γραμματεία του ΔΣ. Η προθεσμία καιη άσκηση της έφεσης έχουν ανασταλτική ισχύ.
\paragraph { 3. } Για το παραδεκτό της έφεσης απαιτείται η καταβολήτου ποσού των δέκα (10) ευρώ ως έξοδα έφεσης, εκτόςεάν εκκαλών είναι το ΔΣ ή το Περιφερειακό Συμβούλιο.Η απόδειξη καταβολής εκδίδεται από το ΔΣ, ενώ το ποσότων δέκα (10) ευρώ αποδίδεται στον εκκαλούντα σε περίπτωση αποδοχής της έφεσης.
\paragraph { 4. } Το Δευτεροβάθμιο Πειθαρχικό Συμβούλιο με αιτιολογημένη απόφασή του μπορεί να διατάξει νέα ανάκριση που διενεργείται κατά τις διατάξεις του άρθρου 107,να καλεί τον τιμωρηθέντα κοινωνικό λειτουργό και να μεταρρυθμίζει ή να εξαφανίζει την εκκαλούμενη απόφαση.
\paragraph { 5. } Το Δευτεροβάθμιο Πειθαρχικό Συμβούλιο αποφασίζει αμετάκλητα και εκδίδει την απόφασή του εντόςτριμήνου από τη διαβίβαση σε αυτό της υπόθεσης. Ηαπόφαση του Δευτεροβάθμιου Πειθαρχικού Συμβουλίουδιαβιβάζεται στον Πρόεδρο του οικείου ΠΤ, ο οποίοςοφείλει χωρίς καθυστέρηση να την κοινοποιήσει προςτον τιμωρηθέντα κοινωνικό λειτουργό.
\paragraph { 6. } Οι τελεσίδικες αποφάσεις εκτελούνται από τον Πρόεδρο του Περιφερειακού Συμβουλίου.
\subsection*{ Άρθρο 109 }
\paragraph { 1. } Οι διατάξεις του Κώδικα Πολιτικής Δικονομίας γιατην εξαίρεση δικαστών ισχύουν αναλογικά και για ταμέλη των Πειθαρχικών Συμβουλίων.
\paragraph { 2. } Η αίτηση εξαίρεσης επιδίδεται στον Πρόεδρο τουΠειθαρχικού Συμβουλίου.
\paragraph { 3. } Όταν ζητείται η εξαίρεση ολόκληρου του Πειθαρχικού Συμβουλίου ή τόσων μελών του, ώστε να μην καθίσταται εφικτή η νόμιμη συγκρότησή του, η αίτηση διαβιβάζεται από τον Πρόεδρο του Πειθαρχικού Συμβουλίουστο ΔΣ του ΣΚΛΕ και η διαδικασία αναστέλλεται μέχριτην έκδοση της απόφασης του ΔΣ επί της αιτήσεως.
\paragraph { 4. } Σε περίπτωση αποδοχής της αιτήσεως, αν δεν επαρκεί ο αριθμός των μελών για την ανασυγκρότηση τουΠειθαρχικού Συμβουλίου, η υπόθεση παραπέμπεται απότο ΔΣ του ΣΚΛΕ στο Ανώτατο Πειθαρχικό Συμβούλιο.
\paragraph { 5. } Κοινωνικός λειτουργός που διώκεται πειθαρχικάμπορεί να ζητήσει την εξαίρεση των μελών του Πειθαρχικού Συμβουλίου μία (1) φορά μόνο κατά βαθμόδικαιοδοσίας.
\subsection*{ Άρθρο 110 }
\paragraph { 1. } Για την έκδοση από τις αρμόδιες υπηρεσίες της βεβαίωσης νομίμου άσκησης του επαγγέλματος, οι κοινωνικοί λειτουργοί οφείλουν να συμπεριλάβουν, μεταξύτων δικαιολογητικών, και βεβαίωση εγγραφής στο οικείοΠΤ του ΣΚΛΕ. Με την παραλαβή της βεβαίωσης νομίμουάσκησης επαγγέλματος υποχρεούνται να την καταθέσουν άμεσα στο Περιφερειακό Τμήμα όπου ανήκουν.
\paragraph { 2. } Όποιος ασκεί το επάγγελμα του κοινωνικού λειτουργού, χωρίς να έχει βεβαίωση νομίμου άσκησης επαγγέλματος, διώκεται ποινικά σύμφωνα με το άρθρο 458 τουΠοινικού Κώδικα. Καταγγελία για παράνομη άσκηση τουεπαγγέλματος μπορεί να κάνει οποιοσδήποτε ιδιώτηςστα Περιφερειακά Συμβούλια ή στο ΔΣ του ΣΚΛΕ, τοοποίο στη συνέχεια υποχρεούται να γνωστοποιήσει τογεγονός στις αρμόδιες δικαστικές αρχές.
\paragraph { 3. } Εντός έξι (6) μηνών από τη δημοσίευση του παρόντος υποχρεούνται όλοι οι κοινωνικοί λειτουργοί ναεγγραφούν στα μητρώα του ΣΚΛΕ. Μετά την πάροδοαυτού του χρονικού διαστήματος η άσκηση του επαγγέλματος χωρίς εγγραφή στον ΣΚΛΕ συνιστά πειθαρχικόπαράπτωμα.
\paragraph { 4. } Στις περιπτώσεις που επιβάλλεται πειθαρχική ποινήπροσωρινής ή οριστικής διαγραφής από τον ΣΚΛΕ αναστέλλεται αυτοδίκαια η άδεια άσκησης επαγγέλματος.
\subsection*{ Άρθρο 111 }
\paragraph { 0. } Η σφραγίδα του ΣΚΛΕ αποτελείται από δύο επάλληλους και ομόκεντρους κύκλους, ο εξωτερικός των οποίων έχει διάμετρο 0,04 εκατοστά. Στο εσωτερικό τουςυπάρχει το έμβλημα της Ελληνικής Δημοκρατίας, ενώγύρω από αυτό αναγράφονται κυκλικά στο επάνω μέροςοι λέξεις «ΣΥΝΔΕΣΜΟΣ ΚΟΙΝΩΝΙΚΩΝ ΛΕΙΤΟΥΡΓΩΝ ΕΛΛΑΔΟΣ», οι οποίες προκειμένου για ΠΤ συμπληρώνονταιμε τις λέξεις «ΠΕΡ. ΤΜΗΜΑ.....». Στον εξωτερικό κύκλοαναγράφονται οι λέξεις «ΕΛΛΗΝΙΚΗ ΔΗΜΟΚΡΑΤΙΑ».
\subsection*{ Άρθρο 112 }
\paragraph { 1. } Μέχρι την ανάδειξη των οργάνων διοίκησης τουΣΚΛΕ, σύμφωνα με τις διατάξεις του παρόντος, μετάαπό πρόταση του σωματείου «Σύνδεσμος ΚοινωνικώνΛειτουργών Ελλάδος», ορίζεται, με απόφαση του Υπουργού Εργασίας, Κοινωνικής Ασφάλισης και Κοινωνικής Αλληλεγγύης, Προσωρινή Διοικούσα Επιτροπή του ΣΚΛΕ,αποτελούμενη από δεκατρία (13) μέλη.
\paragraph { 2. } Η Προσωρινή Διοικούσα Επιτροπή έχει τις εξής αρμοδιότητες έως ότου αναδειχθεί το πρώτο ΔΣ του ΣΚΛΕ:α) Ενημερώνει τους κοινωνικούς λειτουργούς σε όλεςτις Περιφέρειες της χώρας.β) Καταγράφει και εγγράφει όλους τους κοινωνικούςλειτουργούς στον ΣΚΛΕ.γ) Συγκαλεί Γενική Συνέλευση σε όλες τις Περιφέρειεςμε σκοπό τη διενέργεια εκλογών για την ανάδειξη Διοικητικού Συμβουλίου, Περιφερειακών Συμβουλίων, Εξελεγκτικών Επιτροπών, Αντιπροσώπων των Περιφερειώνκαι μελών του Ανώτατου Πειθαρχικού Συμβουλίου.
\paragraph { 3. } Τα μέλη της Προσωρινής Διοικούσας Επιτροπής,εφόσον απασχολούνται στο δημόσιο τομέα, δικαιούνταιάδεια απουσίας από την υπηρεσία τους. Κατά το χρόνοαυτόν διατηρούν όλα τα δικαιώματα που απορρέουναπό την υπαλληλική τους ιδιότητα.
\paragraph { 4. } Τα έξοδα μετακίνησης του προσωπικού του ΣΚΛΕκαι των μελών της Προσωρινής Διοικούσας Επιτροπήςκαταβάλλονται από τον ΣΚΛΕ.
\paragraph { 5. } Η Προσωρινή Διοικούσα Επιτροπή, μετά την ανάδειξη των μελών κάθε Περιφερειακού Συμβουλίου, παραδίδει σε αυτό το μητρώο του Περιφερειακού Τμήματος
\paragraph { 6. } Έδρα της Προσωρινής Διοικούσας Επιτροπής ορίζεται η Αθήνα. Η θητεία της ορίζεται διετής και αρχίζει απόΤεύχος Α’ 137/13.09.2017την ημερομηνία διορισμού των μελών της. Η ΠροσωρινήΔιοικούσα Επιτροπή υποχρεούται να κινήσει αμέσως τιςδιαδικασίες ανάδειξης των οργάνων διοίκησης. Παράταση της χρονικής διάρκειας της θητείας της είναι εφικτήμόνο με απόφαση του Υπουργού Εργασίας, ΚοινωνικήςΑσφάλισης και Κοινωνικής Αλληλεγγύης και μόνο για τοδιάστημα που απαιτείται για την ολοκλήρωση των διαδικασιών ανάδειξης των οργάνων διοίκησης του ΣΚΛΕ.
\paragraph { 7. } Ειδικά, για την εκλογή των πρώτων οργάνων διοίκησης του ΣΚΛΕ λειτουργεί τεκμήριο αρμοδιότητας υπέρτης Προσωρινής Διοικούσας Επιτροπής για την έγκυρηδιενέργεια όλων των προπαρασκευαστικών πράξεων καιτην έκδοση των απαιτουμένων αποφάσεων.
\subsection*{ Άρθρο 113 }
\paragraph { 0. } Με απόφαση του Υπουργού Εργασίας, ΚοινωνικήςΑσφάλισης και Κοινωνικής Αλληλεγγύης, ύστερα απόπρόταση του Ν.Π.Δ.Δ. ΣΚΛΕ, θεσπίζεται κώδικας δεοντολογίας των κοινωνικών λειτουργών. Ο κώδικας δεοντολογίας δεσμεύει, ως προς την εφαρμογή των αρχώντου, όλους τους κοινωνικούς λειτουργούς που ασκούντο επάγγελμα σε οποιοδήποτε πλαίσιο.
\subsection*{ Άρθρο 114 }
\paragraph { 1. } Κεντρική ΔιοίκησηΣτην Κεντρική Διοίκηση συνιστώνται οργανικές θέσειςως εξής:α) Κλάδος ΠΕ/ΤΕ Κοινωνικών Λειτουργών: Μία (1) θέση.β) Κλάδος ΠΕ/ΤΕ Διοικητικός - Λογιστικός: Μία (1) θέση.γ) Κλάδος ΔΕ Διοικητικός: Δύο (2) θέσεις.δ) Κλάδος ΥΕ Γενικών Καθηκόντων: Μία (1) θέση.ε) Νομικός Σύμβουλος: Μία (1) θέση.Οι αποδοχές, αποζημιώσεις και λοιπά έξοδα των υπαλλήλων του ΣΚΛΕ καλύπτονται από ίδιους πόρους τουΣΚΛΕ και σε καμία περίπτωση δεν βαρύνεται ο ΚρατικόςΠροϋπολογισμός.
\paragraph { 2. } Προσόντα θέσεων κατά κλάδοΤα προσόντα που απαιτούνται για τη θέση του κάθεκλάδου είναι τα εξής:Α. Κλάδος ΠΕ/ΤΕ Κοινωνικών Λειτουργών:α) Πτυχίο Τμήματος Κοινωνικής Εργασίας εσωτερικούή ισότιμου εξωτερικού ή του Τμήματος Κοινωνικής Διοίκησης του ΔΠΘ.β) Αναγνωρισμένος τίτλος μεταπτυχιακών σπουδώνσε θέματα σχετικά με το αντικείμενο της θέσης, της ημεδαπής ή αλλοδαπής.γ) Πολύ καλή γνώση τουλάχιστον της αγγλικήςγλώσσας, που θα πιστοποιείται με πτυχίο Advanced ήProficiency.δ) Άριστη γνώση χειρισμού Η/Υ.ε) Τριετής προϋπηρεσία σε θέση κοινωνικού λειτουργού.Β. Κλάδος ΠΕ/ΤΕ Διοικητικός - Λογιστικός:α) Πτυχίο Α.Ε.Ι. Τμήματος Νομικής ή Οικονομικών καιΝομικών Επιστημών ή Οικονομικών και Εμπορικών Επιστημών ή πτυχίο Α.Σ.Ο.Ε.Ε. ή Π.Α.Σ.Π.Ε. ή ισότιμου τηςαλλοδαπής.β) Πολύ καλή γνώση τουλάχιστον της αγγλικήςγλώσσας, που θα πιστοποιείται με πτυχίο Advanced ήProficiency.γ) Άριστη γνώση χειρισμού Η/Υ.δ) Τριετής προϋπηρεσία σε αντικείμενο οικονομικό -λογιστικό.Γ. Κλάδος ΔΕ Διοικητικού:α) Απολυτήριο Λυκείου.β) Πολύ καλή γνώση τουλάχιστον της αγγλικής, πουθα πιστοποιείται με πτυχίο Advanced ή Proficiency ήσπουδές στο εξωτερικό.γ) Άριστη γνώση χειρισμού και εφαρμογής προγραμμάτων Η/Υ.Δ. Κλάδος ΥΕ Γενικών Καθηκόντων:α) Απολυτήριο Γυμνασίου.β) Πολύ καλή γνώση της αγγλικής, που θα πιστοποιείται με πτυχίο Lower.E. Νομικός Σύμβουλος:α) Δικηγόρος παρ’ Αρείω Πάγω.β) Αναγνωρισμένο τίτλο μεταπτυχιακών σπουδώννομικής.
\paragraph { 3. } Αρμοδιότητες προσωπικούΟι αρμοδιότητες των υπαλλήλων των πιο κάτω κλάδωνορίζονται ως εξής:Α. Κλάδος ΠΕ/ΤΕ Κοινωνικών Λειτουργών:α) Ευθύνεται για την εκτέλεση των αποφάσεων του ΔΣ.β) Συντονίζει τις δραστηριότητες μεταξύ ΔΣ και Περιφερειακών Συμβουλίων.γ) Συμμετέχει χωρίς δικαίωμα ψήφου στις συνεδριάσεις του ΔΣ.δ) Έχει την ευθύνη των διεθνών σχέσεων.Ο υπάλληλος του κλάδου αυτού, όταν απουσιάζει,αναπληρώνεται από μέλος του Δ.Σ. με απόφασή του.Β. Κλάδος ΠΕ/ΤΕ Διοικητικός - Λογιστικός:α) Μελετά τις οικονομικές ανάγκες της Κεντρικής καιΠεριφερειακής Διοίκησης του ΣΚΛΕ και τις εισηγείται σταΠεριφερειακά Συμβούλια και ΔΣ του ΣΚΛΕ.β) Καταρτίζει τον οικονομικό προϋπολογισμό της Κεντρικής Διοίκησης του ΣΚΛΕ και των ΠΤ.γ) Πραγματοποιεί πάσης φύσεως δαπάνες, τις αποδοχές του προσωπικού και λοιπές αποζημιώσεις τηςΚεντρικής Διοίκησης.δ) Προβαίνει στις απαιτούμενες διατυπώσεις της προμήθειας πάσης φύσεως υλικού της Κεντρικής Διοίκησηςτου ΣΚΛΕ, της διαχείρισης και της φύλαξης του υλικούπου χρειάζεται για τη λειτουργία της.ε) Καταρτίζει τον οικονομικό ισολογισμό και απολογισμό του έτους, τηρεί τα λογιστικά βιβλία της ΚεντρικήςΔιοίκησης, τα αποδεικτικά εισαγωγής υλικού και τα λοιπάαπαραίτητα δικαιολογητικά.στ) Παρέχει συμβουλές για πάσης φύσεως οικονομικάθέματα που προκύπτουν στην Περιφερειακή Διοίκηση.ζ) Εισπράττει τα έσοδα της Κεντρικής Διοίκησης τουΣΚΛΕ και έχει την ευθύνη της διαφύλαξης της περιουσίας της.η) Τηρεί οικονομικά στατιστικά στοιχεία.Τον υπάλληλο του κλάδου αυτού, όταν απουσιάζει,αναπληρώνει, με απόφαση του Διοικητικού Συμβουλίου,ένας από τους υπαλλήλους του κλάδου ΔΕ Διοικητικού.Γ. Κλάδος ΔΕ Διοικητικός:α) Επικουρεί τον Οικονομικό Διευθυντή στην εκτέλεσητων αρμοδιοτήτων του, εφόσον παρίσταται ανάγκη.β) Εκτελεί όλη τη γραφική εργασία.γ) Τηρεί το αρχείο, μητρώο και πρωτόκολλο.δ) Φροντίζει για κάθε θέμα που έχει σχέση με την οργάνωση της γραμματειακής στήριξης, σύμφωνα με τιςαποφάσεις του ΔΣ.ε) Συντάσσει κάθε εν γένει έγγραφο, τηρεί τα πρακτικάτων συνεδριάσεων του ΔΣ και του Συνεδρίου του ΣΚΛΕ,διακινεί και επιδίδει υπηρεσιακά έγγραφα, αλληλογραφία και άλλα αρμοδίως.στ) Τηρεί βιβλία στατιστικής.ζ) Συνεπικουρεί στην οργάνωση του Συνεδρίου τωνΑντιπροσώπων του ΣΚΛΕ.γλώσσα και αντίστροφα.η) Μεταφράζει κείμενα από την ελληνική σε αγγλικήΔ. Κλάδος ΥΕ Γενικών Καθηκόντων:Πραγματοποιεί όλες τις απαραίτητες εργασίες για τηνλειτουργία του ΔΣ καθ’ υπόδειξη και σύμφωνα με τιςαποφάσεις του ΔΣ.Ε. Νομικός Σύμβουλος:α) Παρέχει νομική κάλυψη και υποστήριξη στο ΣΚΛΕ.β) Συντάσσει γνωμοδοτήσεις επί θεμάτων νομικής φύσεως για κάθε ζήτημα οργάνωσης και λειτουργίας τουΣΚΛΕ και προώθησης των καταστατικών σκοπών του.γ) Εκπροσωπεί τον ΣΚΛΕ ενώπιον διοικητικών και δικαστικών αρχών.
\paragraph { 4. } Περιφερειακά Τμήματα:Σε κάθε Περιφερειακό Τμήμα συνιστάται μία (1) οργανική θέση κλάδου ΔΕ Διοικητικού με ανάλογα προσόντακαι αρμοδιότητες.
\subsection*{ Άρθρο 115 }
\paragraph { 0. } Με τη δημοσίευση του παρόντος νόμου καταργείταιτο τελευταίο εδάφιο της παραγράφου 2 και η παρ. 5 τουάρθρου 111 του ν. 4387/2016 (Α΄ 85).
\subsection*{ Άρθρο 116 }
\paragraph { 1. } Το Υπουργείο Εργασίας, Κοινωνικής Ασφάλισης καιΚοινωνικής Αλληλεγγύης δύναται να αναθέτει την υλοποίηση, την εκτέλεση και τη διαχείριση προγραμμάτων,δράσεων και ενεργειών που αφορούν προνοιακές πολιτικές της Γενικής Γραμματείας Πρόνοιας στις αρμόδιεςυπηρεσίες του Οργανισμού Γεωργικών Ασφαλίσεων(ΟΓΑ), ο οποίος θα ενεργεί ως εντολοδόχος του Υπουργείου Εργασίας, Κοινωνικής Ασφάλισης και ΚοινωνικήςΑλληλεγγύης με μεταβίβαση των αναγκαίων πιστώσεωνστον προϋπολογισμό του ΟΓΑ.
\paragraph { 2. } Με απόφαση του Υπουργού Εργασίας, ΚοινωνικήςΑσφάλισης και Κοινωνικής Αλληλεγγύης ανατίθενταιστον ΟΓΑ τα προγράμματα, δράσεις και ενέργειες τηςπαραγράφου 1 και καθορίζονται οι όροι και οι προϋποθέσεις υλοποίησης και διαχείρισής τους, καθώς και κάθεαναγκαία λεπτομέρεια για την εφαρμογή του παρόντος.
\subsection*{ Άρθρο 117 }
\paragraph { 0. } Η παρ. 1 του άρθρου 101 του ν. 4483/2017 (Α΄ 107)αντικαθίσταται ως εξής:«1. Οι ΟΤΑ και τα νομικά τους πρόσωπα, εφόσον η χρηματοδότηση της Δράσης «Εναρμόνιση οικογενειακής καιεπαγγελματικής ζωής» δεν επαρκεί για την κάλυψη τωνδαπανών μισθοδοσίας και ασφαλιστικών εισφορών τουπροσωπικού τους, που απασχολείται σε αυτή με συμβάσεις εργασίας ιδιωτικού δικαίου ορισμένου χρόνου,για τους ετήσιους κύκλους 2016-2017 και 2017-2018,καλύπτουν τις δαπάνες αυτές από κάθε είδους τακτικάή έκτακτα ανειδίκευτα έσοδά τους.».
\subsection*{ Άρθρο 118 }
\paragraph { 1. } Στο άρθρο 51 του ν. 4465/2017 (Α΄ 47) προστίθεταιπαράγραφος 2 ως εξής:«2. Παραχωρείται κατά χρήση για ενενήντα εννέα (99)έτη, από το Ταμείο Εθνικής Άμυνας στο Δήμο ΠαύλουΜελά της Περιφερειακής Ενότητας Θεσσαλονίκης, προςεξυπηρέτηση λόγων γενικού συμφέροντος και με τουςειδικότερους όρους που προβλέπονται στη μεταξύ τουςσύμβαση, έκταση 332.103,98 τ.μ., που αποτελεί μέροςτου ακινήτου του πρώην στρατοπέδου «Παύλου Μελά»,όπως εμφαίνεται και ορίζεται από τα σημεία 1-2-3...- 84-1στο από 17 Ιουνίου 2017 τοπογραφικό διάγραμμα τουΔιπλωματούχου Τοπογράφου Μηχανικού ΕΜΠ ΑντωνίουΚαραλή, που προσαρτάται στο παρόν ως ΠαράρτημαΑ΄, με αντάλλαγμα την παραχώρηση προς το ΤαμείοΕθνικής Άμυνας της κυριότητας ογδόντα τριών (83)διαμερισμάτων κυριότητας του Δήμου Παύλου Μελά,όπως αυτά εμφαίνονται στο συνημμένο στο παρόν Πίνακα, ως Παράρτημα Β΄. Σκοπός της παραχώρησης τηςως άνω έκτασης του στρατοπέδου είναι η αξιοποίησητης παραχωρούμενης έκτασης του πρώην στρατοπέδου «Παύλου Μελά», με στόχο την ενσωμάτωση τουχώρου αυτού στον αστικό ιστό και την κοινωνική ζωήτης πόλης, ως χώρου υπερτοπικού πρασίνου, για χρήσεις πολιτισμού και αναψυχής μικρής κλίμακας και τηνταυτόχρονη προστασία και ανάδειξή του ως τόπο πολιτιστικής κληρονομιάς. Σκοπός δε, της παραχώρησηςτης κυριότητας των ως άνω 83 διαμερισμάτων είναι ηικανοποίηση των αναγκών στέγασης των στρατιωτικώνκαι των οικογενειών τους και η εξυπηρέτηση του σκοπούτου Ταμείου Εθνικής Άμυνας γενικότερα, όπως καθορίζεται στο άρθρο 3 του ν. 4407/1929. Η σχετική σύμβασησυνάπτεται εντός τριών (3) μηνών από τη δημοσίευσητου παρόντος.»
\paragraph { 2. } Το εδάφιο β΄ της παρ. 1 του άρθρου 51 τουν. 4465/2017 (Α΄ 47) καταργείται.Τεύχος Α’ 137/13.09.2017Τεύχος Α’ 137/13.09.2017Τεύχος Α’ 137/13.09.2017
\subsection*{ Άρθρο 119 }
\paragraph { 1. } Δημιουργούνται Ηλεκτρονικές Βάσεις καταχώρισηςδεδομένων τεχνικών ασφαλείας και ιατρών εργασίας,που λειτουργούν στο πλαίσιο του Ολοκληρωμένου Πληροφοριακού Συστήματος του ΣΕΠΕ (ΟΠΣ-ΣΕΠΕ) και στιςοποίες εντάσσονται τα φυσικά πρόσωπα που πληρούντις απαραίτητες, κατά τα οριζόμενα στις διατάξεις τηςκείμενης νομοθεσίας, προϋποθέσεις για την ανάληψηκαθηκόντων τεχνικού ασφαλείας και ιατρού εργασίας.Με τη θέση σε λειτουργία των Ηλεκτρονικών Βάσεων, ηανάθεση καθηκόντων τεχνικού ασφαλείας και ιατρού εργασίας από τους εργοδότες γίνεται αποκλειστικά μεταξύτων εγγεγραμμένων στις εν λόγω Ηλεκτρονικές Βάσεις,με τους όρους και τη διαδικασία που προβλέπουν οιαποφάσεις του Υπουργού Εργασίας, Κοινωνικής Ασφάλισης και Κοινωνικής Αλληλεγγύης της παραγράφου 2του παρόντος.
\paragraph { 2. } Με αποφάσεις του Υπουργού Εργασίας, ΚοινωνικήςΑσφάλισης και Κοινωνικής Αλληλεγγύης καθορίζεται τοπεριεχόμενο και τα στοιχεία των Ηλεκτρονικών Βάσεωνκαταχώρισης δεδομένων τεχνικών ασφαλείας και ιατρώνεργασίας, ο τρόπος και η διαδικασία εγγραφής σε αυτές,οι ενέργειες των εμπλεκομένων μερών για την ανάθεσηκαθηκόντων τεχνικού ασφαλείας και ιατρού εργασίαςκαι την αποδοχή αυτής ή τη μεταβολή της ή τη λήξη ήτη διακοπή της, οι αρμόδιες υπηρεσίες, οι προθεσμίες, οικυρώσεις και κάθε άλλο σχετικό ειδικότερο, οργανωτικό,τεχνικό ή λεπτομερειακού χαρακτήρα θέμα που αφοράστις εν λόγω Ηλεκτρονικές Βάσεις.
\paragraph { 3. } Ρυθμίσεις του παρόντος Κώδικα ή άλλων νόμων καικανονιστικών πράξεων που δεν έρχονται σε αντίθεσημε τις διατάξεις των παραγράφων 1 και 2 του παρόντος,εξακολουθούν να ισχύουν.».
\subsection*{ Άρθρο 120 }
\paragraph { 0. } Η παρ. 2 του άρθρου 80 του ν. 4316/2014 αντικαθίσταται ως ακολούθως:«2. Οι ανωτέρω οφειλές, ληξιπρόθεσμες ή μη, απαλλάσσονται πλήρως από κάθε είδους φόρους, πρόστιμα,τόκους και προσαυξήσεις εκπρόθεσμης καταβολής. Γιατις εν λόγω οφειλές οι ΔΕΥΑ και οι νόμιμοι εκπρόσωποιαυτών είναι φορολογικά ενήμερες. Τυχόν καταλογιστικές πράξεις που έχουν εκδοθεί και μέτρα αναγκαστικήςείσπραξης που έχουν επιβληθεί σε βάρος των παραπάνωπροσώπων ή και των επιχειρήσεων ακυρώνονται.».
\subsection*{ Άρθρο 121 }
\paragraph { 0. } Στην περίπτωση γ΄ της παρ. 2 του άρθρου 79 τουν. 4316/2014 (Α΄ 270) προστίθεται εδάφιο «ν», ως εξής:«ν) αναστέλλεται η ποινική δίωξη σε βάρος των υπευθύνων κατά τις διατάξεις των παραγράφων 1 και 2 τουάρθρου 1 του α.ν. 86/1967 (Α΄ 136), όπως ισχύει, καιαναβάλλεται η εκτέλεση της ποινής που επιβλήθηκε ή,εφόσον άρχισε η εκτέλεσή της, διακόπτεται».
\subsection*{ Άρθρο 122 }
\paragraph { 0. } Από την έναρξη ισχύος του ν. 4487/2017 (Α΄ 116), στηνπερίπτωση β΄ της παρ. 3 του άρθρου 2 του ν. 3548/2007(Α΄ 68), όπως το άρθρο αυτό αντικαταστάθηκε με το άρθρο 14 του ν. 4487/2017 (Α΄ 116) και ισχύει, η φράση«τριάντα τριών χιλιάδων οκτακοσίων εβδομήντα δύοτετραγωνικών εκατοστόμετρων (33.872 cm2)» αντικαθίσταται από τη φράση «δεκαέξι χιλιάδων εννιακοσίων τριάντα έξι τετραγωνικών εκατοστόμετρων (16.936 cm2)».Τεύχος Α’ 137/13.09.2017
\end{document}